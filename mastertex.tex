\documentclass{article}
\usepackage[english]{babel}
\usepackage{enumerate, latexsym, amssymb, amsmath}
\usepackage{framed, multicol}
\newenvironment{forthel}{\begin{leftbar}}{\end{leftbar}}

%%%%%%%%%% Start TeXmacs macros
\newcommand{\tmaffiliation}[1]{\\ #1}
\newcommand{\tmem}[1]{{\em #1\/}}
\newenvironment{enumeratenumeric}{\begin{enumerate}[1.] }{\end{enumerate}}
\newenvironment{proof}{\noindent\textbf{Proof\ }}{\hspace*{\fill}$\Box$\medskip}
\newenvironment{quoteenv}{\begin{quote} }{\end{quote}}
\newtheorem{axiom}{Axiom}
\newtheorem{lemma}{Lemma}
\newtheorem{theorem}{Theorem}
\newtheorem{definition}{Definition}
\newtheorem{signature}{Signature}
\newtheorem{proposition}{Proposition}
%%%%%%%%%% End TeXmacs macros

\newcommand{\event}{UITP 2018}
\newcommand{\dom}{Dom}
\newcommand{\fun}{aFunction}
\newcommand{\sym}{sym}
\newcommand{\halfline}{{\vspace{3pt}}}
\newcommand{\tab}{{\hspace{1cm}}}
\newcommand{\ball}[2]{B_{#1}(#2)}
\newcommand{\llbracket}{[}
\newcommand{\rrbracket}{]}
\newcommand{\less}[1]{<_{#1}}
\newcommand{\greater}[1]{>_{#1}}
\newcommand{\leeq}[1]{{\leq}_{#1}}
\newcommand{\supr}[1]{\mathrm{sup}_{#1}}
\newcommand{\RR}{\mathbb{R}}
\newcommand{\QQ}{\mathbb{Q}}
\newcommand{\ZZ}{\mathbb{Z}}
\newcommand{\NN}{\mathbb{N}}

\begin{document}

\title{An SAD3 Formalization of the Basic Number Systems for Walter Rudin's
\it{Principles of Mathematical Analysis}}

\author{Roman Pl}

\date{August 25, 2018}

\maketitle


\section{Introduction}
SAD3 kann formalieserte Texte verifizieren, die für Menschen angenehm lesbar sind.
Dies wird in diesem Text dargestellt, es folgen die Kapitel von Ordered Set, Field, Ordered Field von {\it Principles of Mathematical Analysis} by Walter Rudin \cite{Rudin}.
Die Kapitel Real Field and Complex Field sind in seperaten Texten.

Die Datein ord-set.ftl, field.ftl, ord-field.ftl, complex sind nah am Rudin geschrieben. Für die Abgabe und zusamenführung mit den anderen Arbeitgruppen wurden die Dateine in den spezialen Fall von reelenzahlen umgeschrieben. Praktisch gesehen war das umschreiben nicht viel, aber das wiederherstellen der funktionalität war mühsam. Da durch den erhöhten Kontext (4 Kapitel) es in manchen Beweisen zu problemen führt.
Lösung dafür ist die Beweise mit Atrubuten zu ergänzen z.B. "(by Satz-name)".
Ich möchte mit der Anmerkung: "wenig Umschreiben", anmerken wie gut die Modularität von SAD3 funktioniert.

Die ForThel-Datein existieren für fast jedes Kapitel in zwei Versionen.
Die zwei Versionen unterscheiden sich so, das z.B. OrdSet.ftl sind die Relationen mit R benannt und beliebige Relationen, das heißt man kann beweisen, falls man eine 2-stelliges-fkt-symbol hat das diese eine Relation ist falls Sie eine ist.
In der Version in Master.ftl is OrdSet in der Schreibweise mit < geschrieben sodass später Kapitel, dies benutzen können.

Am Dramatischsten ist diese Änderung in Field.ftl zu sehen da mul(AM)[( - , - )] zu einem einfachen * geworden ist.


\subsection{Some Set-Theoretic Terminology}

Formalisierung von Rudins Ordered Set-Kapitel und grundlegende Mengentheorie, Begriffe.

\begin{forthel}
[set/-s] [element/-s] [belong/-s] [subset/-s] [relation/-s] [number/-s]

\begin{signature} A real number is a notion.

\end{signature}
\begin{definition} $\mathbb{R}$ is the set of real numbers. 

\end{definition}
\begin{axiom} every real numbers is element of $\mathbb{R}$.

\end{axiom}


Let S,T,A,B,C denote sets.

\begin{definition} DefSubset.   A subset of S is a set T

\end{definition}
such that every element of T is a element of S.

\begin{definition} DefEmpty.    S is empty iff S has no elements.

\end{definition}

\begin{signature} A relation is a notion.

\end{signature}

\begin{signature} An order is a relation.

\end{signature}

\begin{signature}Let $<$ be a relation.

\end{signature}

\begin{signature} Let x,y be elements of $\mathbb{R}$. x$<$y is an atom.

\end{signature}

Let x,y,z denote real numbers.

\begin{axiom} x$<$y or y$<$x or x = y.

\end{axiom}

\begin{axiom} Then not ((x$<$y and y$<$x) or (x$<$y and y=x) or (x=y and y$<$x)).

\end{axiom}
\begin{axiom} If x$<$y and y$<$z then x$<$z.

\end{axiom}



\begin{definition} upperBound.

\end{definition}
Assume A is a subset of $\mathbb{R}$.
An upper bound of A is an element b of $\mathbb{R}$ such that ( y$<$b or y = b ) for every element y of A.

\begin{definition} lowerBound.

\end{definition}
Assume A is a subset of $\mathbb{R}$.
A lower bound of A is an element b of $\mathbb{R}$ such that ( b$<$y or y = b ) for every element y of A.

\begin{definition} Supremum.

\end{definition}
Assume A is a subset of $\mathbb{R}$.
Let s be an element of $\mathbb{R}$.
s is supremum of A  iff s is an upper bound of A 
and for every element x of $\mathbb{R}$ if x$<$s then x is not an upper bound of A .

\begin{definition} Infimum.

\end{definition}
Assume A is a subset of $\mathbb{R}$.
Let s be an element of $\mathbb{R}$.
s is infimum of A  iff s is an lower bound of A 
and for every element x of $\mathbb{R}$ if s$<$x then x is not an lower bound of A .

\begin{definition} BoundedBelow.

\end{definition}
Assume A is a subset of $\mathbb{R}$.
A is bounded below  iff 
there exists an element b of $\mathbb{R}$ such that b is a lower bound of A .

\begin{definition} BoundedAbove.

\end{definition}
Assume A is a subset of $\mathbb{R}$.
A is bounded above  iff 
there exists an element b of $\mathbb{R}$ such that b is an upper bound of A .

\begin{definition} leastupperboundproperty.

\end{definition}
Assume R is a subset of $\mathbb{R}$.
R is lub iff for every subset A of R
if (A is bounded above and not empty) then (there exists an element s of R such that s is supremum of A ).

\begin{theorem}
.
\end{theorem}
Assume $\mathbb{R}$ is lub.
Assume B is a subset of $\mathbb{R}$ and not empty and bounded below.
Let L = {a in $\mathbb{R}$ | a is lower bound of B}.
Then there exists an element a of $\mathbb{R}$ such that a is supremum of L and a is infimum of B .

\begin{proof}
L is not empty.
Every element x of B is an upper bound of L.
Then L is bounded above.
Take an element a of $\mathbb{R}$ such that a is supremum of L.
a is an element of L.
\end{proof}





\end{forthel}

\subsection{The Field}

Im Kapitel Field und ord-field werden aus den Axiomen viele einfache Aussagen der Körpertheorie gezeigt und Umformung-, Rechen- und Kürzungsregen gezeigt, Diese sind essenziell für das weiter Rechnen in anderen Beweisen.

\begin{forthel}

\begin{signature} A field is a notion.

\end{signature}

Let x,y,z denote real numbers.

\begin{signature} Add. x + y is real number.

\end{signature}
\begin{axiom} A1. Let x, y be real numbers. Then x + y is a real number.

\end{axiom}
\begin{axiom} A2. x + y = y + x.

\end{axiom}
\begin{axiom} A3. (x + y) + z = x + (y + z). 

\end{axiom}
\begin{signature} A4. 0 is element of $\mathbb{R}$ such that for every real number x x + 0 = x.

\end{signature}
\begin{signature} A5. -x is element of $\mathbb{R}$ such that x+(-x) = 0.

\end{signature}

\begin{signature} M1. x * y is a real number.

\end{signature}
\begin{axiom} M2. x * y = y * x.

\end{axiom}
\begin{axiom} M3. (x * y) * z = x * (y * z).

\end{axiom}
\begin{signature} M4. 1 is element of $\mathbb{R}$ such that for every real number x x * 1 = x.

\end{signature}
\begin{axiom} M42. 0 is not equal to 1.

\end{axiom}
\begin{signature} M5. Assume not x = 0. inv(x) is a real number such that inv(x)*x = 1.

\end{signature}

\begin{axiom} D. x*(y + z) = (x*y) + (x*z).

\end{axiom}



\begin{theorem}
 P114a.
If x + y = x + z then y = z.\end{theorem}
\begin{proof}


Assume x + y = x + z.
y  = 0 + y
= ((-x)+x) + y
= -x + (x+y)
= -x + (x+z)
= ((-x)+x) + z
= 0 + z
= z.
\end{proof}


\begin{theorem}
 P114b.
If x + y = x then y = 0.
\end{theorem}
\begin{theorem}
 P114c. 
If x + y = 0 then y = -x.\end{theorem}
\begin{proof}

Assume x + y = 0.
Take z = -x. Then x + z = 0.
Then x + z = x + y. Then y = -x (by P114a).
\end{proof}

\begin{theorem}
 P114d. 
Let x be elements of $\mathbb{R}$. -(-x) = x.
\end{theorem}
\begin{proof}

Then (-x)+x = 0.
Then -(-x) = x (by P114c).
\end{proof}



\begin{theorem}
 P115a.
Let x,y,z be elements of $\mathbb{R}$. If x is not equal to 0 and x*y = x*z then y = z.\end{theorem}
\begin{theorem}
Let x,y be elements of $\mathbb{R}$. If x is not equal to 0 and x*y = x then y = 1.\end{theorem}
\begin{theorem}

Let x,y be elements of $\mathbb{R}$. If x is not equal to 0 and x*y = 1 then y = inv(x).\end{theorem}
\begin{theorem}

Let x be elements of $\mathbb{R}$. If x is not equal to 0 then inv(inv(x)) = x.\end{theorem}



\begin{theorem}
 P116a.
Let x be elements of $\mathbb{R}$. 0*x = 0.\end{theorem}
\begin{proof}

Then (0*x) + (0*x)
= (0+0)*x
= 0*x.
Then 0*x = 0 (by P114b).
\end{proof}

\begin{theorem}
 P116b. 
If x is not equal to 0 and y is not equal to 0 then x*y is not equal to 0.\end{theorem}
\begin{proof}


Let x is not equal to 0 and y is not equal to 0.
Assume the contrary.

Then 1 = (inv(x)*inv(y))*(x*y)
= (inv(x)*inv(y))*0
= 0 .
Contradiction.
\end{proof}


\begin{lemma} L1.


(-x)*y = -(x*y).\end{lemma}
\begin{proof}


((-x)*y) + (x*y) .= (y*(-x)) + (y*x) (by M2)
.= y*(-x+x) (by D) 
.= (-x+x)*y (by M2)
.= 0*y (by A2, A5)
.= 0 (by P116a, M2).

\end{proof}
 
\begin{theorem}
 P116c.
(-x)*y = -(x*y) = x*(-y).\end{theorem}
\begin{proof}

Then(-x)*y = -(x*y) (by L1).

Then -(x*y) = -(y*x) = (-y)*x = x*(-y).

\end{proof}
 

\begin{theorem}
 P116d.
(-x)*(-y)= x*y.\end{theorem}


Let x - y stand for x + (-y).


Let x $>$ y stand for y $<$ x.
Let x $\leq$ y stand for x$<$y or x=y.
Let x $\geq$ y stand for y $\leq$ x.



\end{forthel}

\subsection{The Ordered Real Field}
Im Kapitel Orderen Field werden sowohl Aussagen von Ordered Set und Field benutzt, um Aussagen über Ordered Field zuzeigen.
Aber zuerst müssen Axiom A1 und Axiom A2 einführt werden.
In diesem Kapitel muss man zusätzlich bestimmte Aussagen für SAD betonen da der Kontext schon ziemlich groß ist.

\begin{forthel}
	\begin{axiom} A1. If y$<$z then x+y$<$x+z.

\end{axiom}
	\begin{axiom} A2. If x$>$0 and y$>$0 then x*y$>$0.

\end{axiom}
	
	\begin{theorem}
 P118a. If x$>$0 then -x$<$0. 
\end{theorem}	\begin{proof}

	Assume x$>$0. Then 0= -x + x $>$ -x + 0.
	Then -x$<$0.
	\end{proof}

	\begin{theorem}
 P118a2. If x$<$0 then -x$>$0. 
\end{theorem}	\begin{proof}

	Assume x$<$0. Then 0= -x + x $<$ -x + 0.
	Then -x$>$0.
	\end{proof}

	\begin{theorem}
 P118b. If x$>$0 and y$<$z then x*y$<$x*z.
\end{theorem}	\begin{proof}

	Assume x$>$0 and y$<$z. 
	Let us show that z-y $>$ y-y = 0.
	y$<$z.
	then (-y)+y$<$(-y)+z (by A1).
	
	end.
	Then x*(z-y)$>$0.
	Then x*z = (x*(z-y))+(x*y).
	
	Take a = x*(z-y) and b = x*y.
	Then a$>$0.
	Then x*z = a + b $>$ 0+b = (x*y).
	
	Then x*z $>$ x*y.
	\end{proof}

	\begin{theorem}
 P118c. If x$<$0 and y$<$z then x*y$>$x*z.
\end{theorem}	\begin{proof}

	Assume x$<$0 and y$<$z.
	Then -x$>$0 (by P118a2).
	Then z-y$>$0. 
	Let us show that z-y$>$0.
	y$<$z.
	then (-y)+y$<$(-y)+z (by A1).
	then z-y$>$y-y=0.
	end.
	Then -(x*(z-y)) = (-x)*(z-y) (by P116c).
	Then (-x)*(z-y) $>$ 0.
	Then x*(z-y) $<$ 0.
	Then (x*z) - (x*y) $<$ 0.
	
	Take a = (x*z) and b = - (x*y).
	Then a+b$<$0.
	Then a $<$ -b.
	
	Then x*z $<$ x*y.
	\end{proof}

	
	\begin{theorem}
 P118d. If not x = 0 then x*x $>$ 0.
\end{theorem}	\begin{proof}

	Case x$>$0. Then x*x $>$ 0.
	end.
	Case x$<$0. Then -x$>$0. Then x*x = (-x)*(-x)$>$0.
	end.
	
	\end{proof}

	\begin{lemma} 1$>$0. 

\end{lemma}
	\begin{proof}

	Then 1 = 1*1 $>$ 0 (by P118d, M42).
	\end{proof}

	
	\begin{theorem}
 P118e1. If 0$<$y then 0 $<$ inv(y).
\end{theorem}	\begin{proof}

	(1) Assume the contrary.
	Then y*inv(y) $\leq$ 0.
	Then y*inv(y) = 1 $>$ 0.
	Contradiction.
	
	\end{proof}

	\begin{theorem}
 P118e2. If 0$<$x$<$y then 0 $<$ inv(y)$<$inv(x).
\end{theorem}	\begin{proof}
 
	Assume 0$<$x$<$y.Then 0$<$inv(y) and 0$<$inv(x).
	Then inv(x)*inv(y)$>$0.
	Take a = inv(y)*inv(x).	
	Then x$<$y. Then x*a$<$y*a.
	Then x*(inv(x)*inv(y))$<$ y*(inv(x)*inv(y)).
	Then (x*inv(x))*inv(y)$<$ (y*inv(y))*inv(x).
	Then inv(y) $<$ inv(x).
	
	\end{proof}

	

\end{forthel}








\subsection{Positive Integers}
In diesem Kapitel werden Aussagen gezeigt die für die Kapitel Reals-A und Reals-B wichtig sind.
Es war nicht möglich eine weitere Datei an Master.ftl anzuhägen da der Kontext zu groß war.
Da SAD schwierige Existenz Aussagen zeigen muss die Zeitkritisch sind, besonders schwierig war der Beweis P118b. Diesem ist die Datei Reals-B gewidmet.
 
\begin{forthel}
	\begin{signature} A natural number is a real number.

\end{signature}
	
	\begin{definition} NatSet.

\end{definition}
	$\mathbb{N}$ is the set of natural numbers.
	
	\begin{axiom} 0 is natural number.

\end{axiom}
	Let n, m denote a natural number.
	\begin{axiom} n+1 is a natural number.

\end{axiom}
	\begin{axiom} n $\geq$0.

\end{axiom}
	
	\begin{signature} a natural number is a real number.

\end{signature}
	
	

	
	\begin{theorem}
 Lis2. Let n be a natural number and not n=0. If n*x$<$m then x$<$m*inv(n).
\end{theorem}	\begin{proof}

	Assume n*x$<$m.
	n $>$ 0.
	inv(n) $>$ 0 (by P118e1).
	inv(n)*(n*x)$<$inv(n)*m (by P118b).
	(x*n)*inv(n)$<$m*inv(n).
	x*(n*inv(n))$<$m*inv(n).
	x*1$<$m*inv(n).
	x$<$m*inv(n).
	\end{proof}

	\begin{theorem}
 Lis3. Let n be a natural number and not n=0. if m$<$n*y then m*inv(n)$<$y.
\end{theorem}	\begin{proof}

	Assume m$<$n*y.
	n $>$ 0.
	inv(n) $>$ 0 (by P118e1).
	inv(n)*m$<$inv(n)*(n*y) (by P118b).
	m*inv(n)$<$(n*inv(n))*y.
	m*inv(n)$<$1*y.
	m*inv(n)$<$y.
	\end{proof}

\end{forthel}

\subsection{Archimedian properties}
Finally we have formalized what Rudin assumes for the proof of his 1.20. We can now contrast the formalization with Rudin's original.

\newpage
The next two theorems correspond to Theorems 1.20(a) and (b) of Rudin. We start with the original introductory text of Rudin and then display the two texts in parallel to illustrate the closeness of the formalization to the original.

\vspace{1cm}

The next theorem could be extracted from this construction [of the real numbers] with very little extra effort. However, we prefer to derive it from Theorem 1.19 since this provides a good illustration of what one can do with the least-upper-bound property.


\newpage


\section{Remarks}
\subsection{Structures}

Rudin introduces ordered sets and fields axiomatically and then postulates the ordered field of reals. Indeed the reals are constructed from the rationals numbers in an Appendix to chapter 1.

Since ForTheL does not provide general structures and mechanisms to express that a structure satisfies some abstract axioms, I have instead postulated the structure $\mathbb{R}$ of the real numbers right away. The axioms of ordered sets and fields are then only stated for this particular structure.

Instead of building up number systems "from below" we define the sets of rational, integer and positive integer numbers as subsets of $\mathbb{R}$. This has the advantage that we can use the real addition and multiplication also for those number systems and they become substructures without substructure mechanisms.

Future versions of SAD3 should allow for elegant handling of structures. This also involves a liberal use of operator symbols like $+$ in several additive structures, without mentioning the structure, e.g., as a subscript. This "overloading" would require some (Prolog-like?) derivations of notions for variables and terms.

\subsection{Text Comparisons}

We cover most of the material on pages 3 - 9 of Rudin. By the different organisation, the two presentations only align on page 9 for some archimedean properties. We illustrate this alignment in a two-column synoptic view of the formalization and the original. One could go through the text statement by statement and discuss similarities and differences. One could also put in further work to minimize differences. To some degree this could take part on the level of the formalization, but eventually the ForTheL language and its interpretation in SAD3 would have to be extended.
I shall only mention a few points.






\begin{thebibliography}{1}

\bibitem{Rudin}
  Walter Rudin,
  \textit{Principles of mathematical analysis},
  McGraw-Hill,
  1976.

\end{thebibliography}
  


\end{document}
