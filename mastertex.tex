\documentclass{article}
\usepackage[english]{babel}
\usepackage{enumerate, latexsym, amssymb, amsmath}
\usepackage{framed, multicol}
\newenvironment{forthel}{\begin{leftbar}}{\end{leftbar}}

%%%%%%%%%% Start TeXmacs macros
\newcommand{\tmaffiliation}[1]{\\ #1}
\newcommand{\tmem}[1]{{\em #1\/}}
\newenvironment{enumeratenumeric}{\begin{enumerate}[1.] }{\end{enumerate}}
\newenvironment{proof}{\noindent\textbf{Proof\ }}{\hspace*{\fill}$\Box$\medskip}
\newenvironment{quoteenv}{\begin{quote} }{\end{quote}}
\newtheorem{axiom}{Axiom}
\newtheorem{lemma}{Lemma}
\newtheorem{theorem}{Theorem}
\newtheorem{definition}{Definition}
\newtheorem{signature}{Signature}
\newtheorem{proposition}{Proposition}
%%%%%%%%%% End TeXmacs macros

\newcommand{\event}{UITP 2018}
\newcommand{\dom}{Dom}
\newcommand{\fun}{aFunction}
\newcommand{\sym}{sym}
\newcommand{\halfline}{{\vspace{3pt}}}
\newcommand{\tab}{{\hspace{1cm}}}
\newcommand{\ball}[2]{B_{#1}(#2)}
\newcommand{\llbracket}{[}
\newcommand{\rrbracket}{]}
\newcommand{\less}[1]{<_{#1}}
\newcommand{\greater}[1]{>_{#1}}
\newcommand{\leeq}[1]{{\leq}_{#1}}
\newcommand{\supr}[1]{\mathrm{sup}_{#1}}
\newcommand{\RR}{\mathbb{R}}
\newcommand{\QQ}{\mathbb{Q}}
\newcommand{\ZZ}{\mathbb{Z}}
\newcommand{\NN}{\mathbb{N}}

\begin{document}

\title{An SAD3 Formalization of the Basic Number Systems for Walter Rudin's
\it{Principles of Mathematical Analysis}}

\author{Marcin, Roman Pleskovsky}

\date{9 September 2018}

\maketitle


\section{Einleitung}
SAD3 kann formalieserte Texte verifizieren, die f\"ur Menschen angenehm lesbar sind.
Dies wird in diesem Text dargestellt. Es folgen die Kapitel von Ordered Set, Field, Ordered Field von {\it Principles of Mathematical Analysis} by Walter Rudin \cite{Rudin}.\\
Die Kapitel Reel Field and Complex Field sind in seperaten Texten.\\

Die Datein ord-set.ftl, field.ftl, ord-field.ftl und complex.ftl sind nah am Rudin geschrieben. F\"ur die Abgabe und zusammenf\"uhrung mit den anderen Arbeitgruppen wurden die Dateien in den spezialen Fall von reellen Zahlen umgeschrieben. Praktisch gesehen war das Umschreiben nicht viel, aber das Wiederherstellen der Funktionalit\"at war m\"uhsam, da es durch den erh\"ohten Kontext (4 Kapitel) in manchen Beweisen zu Problemen kommt.\\
L\"osung daf\"ur ist, die Beweise mit Attributen zu erg\"anzen z.B. "(by Satz-name)".
Ich m\"ochte mit der Anmerkung "wenig Umschreiben" unterstreichen, wie gut die Modularit\"at von SAD3 funktioniert.\\

Die ForThel-Dateien existieren f\"ur fast jedes Kapitel in zwei Versionen.\\
Die zwei Versionen unterscheiden sich insofern, dass z.B. in OrdSet.ftl die Relationen mit R benannt sind und die Relation deshalb beliebig ist. Das hei\ss t, dass man beweisen kann, dass, falls man ein 2-stelliges Funktionssymbol hat, dieses eine Relation ist, falls es eine ist.\\
In der Version in Master.ftl ist OrdSet in der Schreibweise mit < geschrieben, sodass sp\"atere Kapitel dies benutzen k\"onnen.\\

Am dramatischsten ist der Unterschied in Field.ftl zu sehen, da mul(AM)[( - , - )] zu einem einfachen * geworden ist.\\


\subsection{Megentheorie}

Formalisierung von Rudins Ordered Set-Kapitel und grundlegende Mengentheorie, Begriffe.\\

\begin{forthel}
[set/-s] [element/-s] [belong/-s] [subset/-s] [relation/-s] [number/-s]

\begin{signature} A real number is a notion.

\end{signature}
\begin{definition} $\mathbb{R}$ is the set of real numbers. 

\end{definition}
\begin{axiom} every real numbers is element of $\mathbb{R}$.

\end{axiom}


Let S,T,A,B,C denote sets.

\begin{definition} DefSubset.   A subset of S is a set T

\end{definition}
such that every element of T is a element of S.

\begin{definition} DefEmpty.    S is empty iff S has no elements.

\end{definition}

\begin{signature} A relation is a notion.

\end{signature}

\begin{signature} An order is a relation.

\end{signature}

\begin{signature}Let $<$ be a relation.

\end{signature}

\begin{signature} Let x,y be elements of $\mathbb{R}$. x$<$y is an atom.

\end{signature}

Let x,y,z denote real numbers.

\begin{axiom} x$<$y or y$<$x or x = y.

\end{axiom}

\begin{axiom} Then not ((x$<$y and y$<$x) or (x$<$y and y=x) or (x=y and y$<$x)).

\end{axiom}
\begin{axiom} If x$<$y and y$<$z then x$<$z.

\end{axiom}



\begin{definition} upperBound.

\end{definition}
Assume A is a subset of $\mathbb{R}$.
An upper bound of A is an element b of $\mathbb{R}$ such that ( y$<$b or y = b ) for every element y of A.

\begin{definition} lowerBound.

\end{definition}
Assume A is a subset of $\mathbb{R}$.
A lower bound of A is an element b of $\mathbb{R}$ such that ( b$<$y or y = b ) for every element y of A.

\begin{definition} Supremum.

\end{definition}
Assume A is a subset of $\mathbb{R}$.
Let s be an element of $\mathbb{R}$.
s is supremum of A  iff s is an upper bound of A 
and for every element x of $\mathbb{R}$ if x$<$s then x is not an upper bound of A .

\begin{definition} Infimum.

\end{definition}
Assume A is a subset of $\mathbb{R}$.
Let s be an element of $\mathbb{R}$.
s is infimum of A  iff s is an lower bound of A 
and for every element x of $\mathbb{R}$ if s$<$x then x is not an lower bound of A .

\begin{definition} BoundedBelow.

\end{definition}
Assume A is a subset of $\mathbb{R}$.
A is bounded below  iff 
there exists an element b of $\mathbb{R}$ such that b is a lower bound of A .

\begin{definition} BoundedAbove.

\end{definition}
Assume A is a subset of $\mathbb{R}$.
A is bounded above  iff 
there exists an element b of $\mathbb{R}$ such that b is an upper bound of A .

\begin{definition} leastupperboundproperty.

\end{definition}
Assume R is a subset of $\mathbb{R}$.
R is lub iff for every subset A of R
if (A is bounded above and not empty) then (there exists an element s of R such that s is supremum of A ).

\begin{theorem}
.
\end{theorem}
Assume $\mathbb{R}$ is lub.
Assume B is a subset of $\mathbb{R}$ and not empty and bounded below.
Let L = {a in $\mathbb{R}$ | a is lower bound of B}.
Then there exists an element a of $\mathbb{R}$ such that a is supremum of L and a is infimum of B .

\begin{proof}
L is not empty.
Every element x of B is an upper bound of L.
Then L is bounded above.
Take an element a of $\mathbb{R}$ such that a is supremum of L.
a is an element of L.
\end{proof}





\end{forthel}

\subsection{The Field}

Im Kapitel Field und ord-field werden aus den Axiomen viele einfache Aussagen der K\"orpertheorie gezeigt und Umformung-, Rechen- und K\"urzungsregen gezeigt. Diese sind essenziell f\"ur das Weiterrechnen in anderen Beweisen.
Manche Aussagen sind so einfach und der Kontext ist noch so klein, dass SAD diese Aussagen selbstst\"andig beweisen kann.

\begin{forthel}

\begin{signature} A field is a notion.

\end{signature}

Let x,y,z denote real numbers.

\begin{signature} Add. x + y is real number.

\end{signature}
\begin{axiom} A1. Let x, y be real numbers. Then x + y is a real number.

\end{axiom}
\begin{axiom} A2. x + y = y + x.

\end{axiom}
\begin{axiom} A3. (x + y) + z = x + (y + z). 

\end{axiom}
\begin{signature} A4. 0 is element of $\mathbb{R}$ such that for every real number x x + 0 = x.

\end{signature}
\begin{signature} A5. -x is element of $\mathbb{R}$ such that x+(-x) = 0.

\end{signature}

\begin{signature} M1. x * y is a real number.

\end{signature}
\begin{axiom} M2. x * y = y * x.

\end{axiom}
\begin{axiom} M3. (x * y) * z = x * (y * z).

\end{axiom}
\begin{signature} M4. 1 is element of $\mathbb{R}$ such that for every real number x x * 1 = x.

\end{signature}
\begin{axiom} M42. 0 is not equal to 1.

\end{axiom}
\begin{signature} M5. Assume not x = 0. inv(x) is a real number such that inv(x)*x = 1.

\end{signature}

\begin{axiom} D. x*(y + z) = (x*y) + (x*z).

\end{axiom}



\begin{theorem}
 P114a.
If x + y = x + z then y = z.\end{theorem}
\begin{proof}


Assume x + y = x + z.
y  = 0 + y
= ((-x)+x) + y
= -x + (x+y)
= -x + (x+z)
= ((-x)+x) + z
= 0 + z
= z.
\end{proof}


\begin{theorem}
 P114b.
If x + y = x then y = 0.
\end{theorem}
\begin{theorem}
 P114c. 
If x + y = 0 then y = -x.\end{theorem}
\begin{proof}

Assume x + y = 0.
Take z = -x. Then x + z = 0.
Then x + z = x + y. Then y = -x (by P114a).
\end{proof}

\begin{theorem}
 P114d. 
Let x be elements of $\mathbb{R}$. -(-x) = x.
\end{theorem}
\begin{proof}

Then (-x)+x = 0.
Then -(-x) = x (by P114c).
\end{proof}



\begin{theorem}
 P115a.
Let x,y,z be elements of $\mathbb{R}$. If x is not equal to 0 and x*y = x*z then y = z.\end{theorem}
\begin{theorem}
Let x,y be elements of $\mathbb{R}$. If x is not equal to 0 and x*y = x then y = 1.\end{theorem}
\begin{theorem}

Let x,y be elements of $\mathbb{R}$. If x is not equal to 0 and x*y = 1 then y = inv(x).\end{theorem}
\begin{theorem}

Let x be elements of $\mathbb{R}$. If x is not equal to 0 then inv(inv(x)) = x.\end{theorem}



\begin{theorem}
 P116a.
Let x be elements of $\mathbb{R}$. 0*x = 0.\end{theorem}
\begin{proof}

Then (0*x) + (0*x)
= (0+0)*x
= 0*x.
Then 0*x = 0 (by P114b).
\end{proof}

\begin{theorem}
 P116b. 
If x is not equal to 0 and y is not equal to 0 then x*y is not equal to 0.\end{theorem}
\begin{proof}


Let x is not equal to 0 and y is not equal to 0.
Assume the contrary.

Then 1 = (inv(x)*inv(y))*(x*y)
= (inv(x)*inv(y))*0
= 0 .
Contradiction.
\end{proof}


\begin{lemma} L1.


(-x)*y = -(x*y).\end{lemma}
\begin{proof}


((-x)*y) + (x*y) .= (y*(-x)) + (y*x) (by M2)
.= y*(-x+x) (by D) 
.= (-x+x)*y (by M2)
.= 0*y (by A2, A5)
.= 0 (by P116a, M2).

\end{proof}
 
\begin{theorem}
 P116c.
(-x)*y = -(x*y) = x*(-y).\end{theorem}
\begin{proof}

Then(-x)*y = -(x*y) (by L1).

Then -(x*y) = -(y*x) = (-y)*x = x*(-y).

\end{proof}
 

\begin{theorem}
 P116d.
(-x)*(-y)= x*y.\end{theorem}


Let x - y stand for x + (-y).


Let x $>$ y stand for y $<$ x.
Let x $\leq$ y stand for x$<$y or x=y.
Let x $\geq$ y stand for y $\leq$ x.



\end{forthel}

\subsection{The Ordered Real Field}
Im Kapitel Orderen Field werden sowohl Aussagen von Ordered Set als auch von Field benutzt, um Aussagen \"uber Ordered Field zu zeigen.
Aber zuerst m\"ussen Axiom A1 und Axiom A2 einf\"uhrt werden.
In diesem Kapitel muss man zus\"atzlich bestimmte Aussagen f\"ur SAD betonen, da der Kontext schon ziemlich gro\ss \enspace ist.

\begin{forthel}
	\begin{axiom} A1. If y$<$z then x+y$<$x+z.

\end{axiom}
	\begin{axiom} A2. If x$>$0 and y$>$0 then x*y$>$0.

\end{axiom}
	
	\begin{theorem}
 P118a. If x$>$0 then -x$<$0. 
\end{theorem}	\begin{proof}

	Assume x$>$0. Then 0= -x + x $>$ -x + 0.
	Then -x$<$0.
	\end{proof}

	\begin{theorem}
 P118a2. If x$<$0 then -x$>$0. 
\end{theorem}	\begin{proof}

	Assume x$<$0. Then 0= -x + x $<$ -x + 0.
	Then -x$>$0.
	\end{proof}

	\begin{theorem}
 P118b. If x$>$0 and y$<$z then x*y$<$x*z.
\end{theorem}	\begin{proof}

	Assume x$>$0 and y$<$z. 
	Let us show that z-y $>$ y-y = 0.
	y$<$z.
	then (-y)+y$<$(-y)+z (by A1).
	
	end.
	Then x*(z-y)$>$0.
	Then x*z = (x*(z-y))+(x*y).
	
	Take a = x*(z-y) and b = x*y.
	Then a$>$0.
	Then x*z = a + b $>$ 0+b = (x*y).
	
	Then x*z $>$ x*y.
	\end{proof}

	\begin{theorem}
 P118c. If x$<$0 and y$<$z then x*y$>$x*z.
\end{theorem}	\begin{proof}

	Assume x$<$0 and y$<$z.
	Then -x$>$0 (by P118a2).
	Then z-y$>$0. 
	Let us show that z-y$>$0.
	y$<$z.
	then (-y)+y$<$(-y)+z (by A1).
	then z-y$>$y-y=0.
	end.
	Then -(x*(z-y)) = (-x)*(z-y) (by P116c).
	Then (-x)*(z-y) $>$ 0.
	Then x*(z-y) $<$ 0.
	Then (x*z) - (x*y) $<$ 0.
	
	Take a = (x*z) and b = - (x*y).
	Then a+b$<$0.
	Then a $<$ -b.
	
	Then x*z $<$ x*y.
	\end{proof}

	
	\begin{theorem}
 P118d. If not x = 0 then x*x $>$ 0.
\end{theorem}	\begin{proof}

	Case x$>$0. Then x*x $>$ 0.
	end.
	Case x$<$0. Then -x$>$0. Then x*x = (-x)*(-x)$>$0.
	end.
	
	\end{proof}

	\begin{lemma} 1$>$0. 

\end{lemma}
	\begin{proof}

	Then 1 = 1*1 $>$ 0 (by P118d, M42).
	\end{proof}

	
	\begin{theorem}
 P118e1. If 0$<$y then 0 $<$ inv(y).
\end{theorem}	\begin{proof}

	(1) Assume the contrary.
	Then y*inv(y) $\leq$ 0.
	Then y*inv(y) = 1 $>$ 0.
	Contradiction.
	
	\end{proof}

	\begin{theorem}
 P118e2. If 0$<$x$<$y then 0 $<$ inv(y)$<$inv(x).
\end{theorem}	\begin{proof}
 
	Assume 0$<$x$<$y.Then 0$<$inv(y) and 0$<$inv(x).
	Then inv(x)*inv(y)$>$0.
	Take a = inv(y)*inv(x).	
	Then x$<$y. Then x*a$<$y*a.
	Then x*(inv(x)*inv(y))$<$ y*(inv(x)*inv(y)).
	Then (x*inv(x))*inv(y)$<$ (y*inv(y))*inv(x).
	Then inv(y) $<$ inv(x).
	
	\end{proof}

	

\end{forthel}








\subsection{Zus\"atzliche Aussagen als Vorbereitung}
In diesem Kapitel werden Aussagen gezeigt, die f\"ur die Kapitel Reals-A und Reals-B wichtig sind.
Es war nicht m\"oglich eine weitere Datei an Master.ftl anzuh\"agen, da der Kontext zu gro\ss \enspace war
und da SAD schwierige Existenzaussagen zeigen muss, die zeitaufwendig sind. Besonders schwierig war der Beweis P118b. Diesem ist die Datei Reals-B gewidmet.
 
\begin{forthel}
	\begin{signature} A natural number is a real number.

\end{signature}
	
	\begin{definition} NatSet.

\end{definition}
	$\mathbb{N}$ is the set of natural numbers.
	
	\begin{axiom} 0 is natural number.

\end{axiom}
	Let n, m denote a natural number.
	\begin{axiom} n+1 is a natural number.

\end{axiom}
	\begin{axiom} n $\geq$0.

\end{axiom}
	
	\begin{signature} a natural number is a real number.

\end{signature}
	
	

	
	\begin{theorem}
 Lis2. Let n be a natural number and not n=0. If n*x$<$m then x$<$m*inv(n).
\end{theorem}	\begin{proof}

	Assume n*x$<$m.
	n $>$ 0.
	inv(n) $>$ 0 (by P118e1).
	inv(n)*(n*x)$<$inv(n)*m (by P118b).
	(x*n)*inv(n)$<$m*inv(n).
	x*(n*inv(n))$<$m*inv(n).
	x*1$<$m*inv(n).
	x$<$m*inv(n).
	\end{proof}

	\begin{theorem}
 Lis3. Let n be a natural number and not n=0. if m$<$n*y then m*inv(n)$<$y.
\end{theorem}	\begin{proof}

	Assume m$<$n*y.
	n $>$ 0.
	inv(n) $>$ 0 (by P118e1).
	inv(n)*m$<$inv(n)*(n*y) (by P118b).
	m*inv(n)$<$(n*inv(n))*y.
	m*inv(n)$<$1*y.
	m*inv(n)$<$y.
	\end{proof}

\end{forthel}




\section{Bemerkungen}

Ich h\"atte mir ein besseres Feedback vom ausf\"uhrendem Prozess gew\"unscht (alice).\\
Zum Beispiel,dass nach jedem Beweis eine Zahl ausgegeben wird, die sagt wie schwer der Bewies war.\\
Damit k\"onnte man fr\"uh erkennen, ob man irgendwelche Anfängerfehler macht, und sehen wo evtl. man ein Lemma erg\"anzen muss oder  wo es sp\"ater mit der Lauff\"ahigkeit eng wird.\\
Und man k\"onnte besser beurteilen, ob \"Anderungen an dem Kontext zu Verbesserungen in Beweisen f\"uhren oder nicht.
Irgendwie war das benutzen des DEC komisch. Weil manche Umformungen die lexikalischen Ordnung erh\"ohen (z.B. Anwendung der  Kommutativit\"at), konnte man sie nicht benutzen.\\
Wir hatten ein ideales Beispiel, wo ein Beweis so kleinlich aufgeschrieben war, dass der DEC es ohne Problem verifizieren konnte.\\
Dazu musste man nur "= zu .=" \"andern. Nachteil war das alles kleinlich aufzuschreiben war.\\



\begin{thebibliography}{1}

\bibitem{Rudin}
  Walter Rudin,
  \textit{Principles of mathematical analysis},
  McGraw-Hill,
  1976.

\end{thebibliography}
  


\end{document}
