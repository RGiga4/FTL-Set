\documentclass{article}
\usepackage[english]{babel}
\usepackage{enumerate, latexsym, amssymb, amsmath}
\usepackage{framed, multicol}
\newenvironment{forthel}{\begin{leftbar}}{\end{leftbar}}

%%%%%%%%%% Start TeXmacs macros
\newcommand{\tmaffiliation}[1]{\\ #1}
\newcommand{\tmem}[1]{{\em #1\/}}
\newenvironment{enumeratenumeric}{\begin{enumerate}[1.] }{\end{enumerate}}
\newenvironment{proof}{\noindent\textbf{Proof\ }}{\hspace*{\fill}$\Box$\medskip}
\newenvironment{quoteenv}{\begin{quote} }{\end{quote}}
\newtheorem{axiom}{Axiom}
\newtheorem{lemma}{Lemma}
\newtheorem{theorem}{Theorem}
\newtheorem{definition}{Definition}
\newtheorem{signature}{Signature}
\newtheorem{proposition}{Proposition}
%%%%%%%%%% End TeXmacs macros

\newcommand{\event}{UITP 2018}
\newcommand{\dom}{Dom}
\newcommand{\fun}{aFunction}
\newcommand{\sym}{sym}
\newcommand{\halfline}{{\vspace{3pt}}}
\newcommand{\tab}{{\hspace{1cm}}}
\newcommand{\ball}[2]{B_{#1}(#2)}
\newcommand{\llbracket}{[}
\newcommand{\rrbracket}{]}
\newcommand{\less}[1]{<_{#1}}
\newcommand{\greater}[1]{>_{#1}}
\newcommand{\leeq}[1]{{\leq}_{#1}}
\newcommand{\supr}[1]{\mathrm{sup}_{#1}}
\newcommand{\RR}{\mathbb{R}}
\newcommand{\QQ}{\mathbb{Q}}
\newcommand{\ZZ}{\mathbb{Z}}
\newcommand{\NN}{\mathbb{N}}
\begin{document}

\title{The Complex Field}

\maketitle

\begin{forthel}
[set/-s] [element/-s] [number/-s]
\begin{signature}. A real number is a notion.

\end{signature}


Let x,y,z,a,b,c,d,e,f,g,h,j,k,r,s denote real numbers.



\begin{signature} Add. x + y is a real number.

\end{signature}
\begin{axiom} RComm. x+y = y+x.

\end{axiom}
\begin{axiom}. (x+y)+z = x+(y+z).

\end{axiom}
\begin{signature} Zero. 0 is a real number such that for every real number x x + 0 = x.

\end{signature}
\begin{signature}. -x is a real number such that -x + x = 0.

\end{signature}

Let x - y stand for x + (-y).

\begin{signature} Mult. x * y is real number.

\end{signature}
\begin{axiom}. x*y = y*x.

\end{axiom}
\begin{axiom}. (x*y)*z = x*(y*z).

\end{axiom}
\begin{signature} One. 1 is a real number such that for every real number x x*1 = x.

\end{signature}
\begin{signature}. Assume not x = 0. inv(x) is a real number such that inv(x)*x = 1 .

\end{signature}
\begin{axiom}. 1 is not equal to 0.

\end{axiom}

\begin{axiom} Dis1. x*(y+z)=(x*y)+(x*z).

\end{axiom}
\begin{axiom} Dis2. (y+z)*x=(y*x)+(z*x).

\end{axiom}


\begin{axiom} P114c. If x+y = 0 then y = -x.

\end{axiom}
\begin{axiom} P114d. -(-x) = x.

\end{axiom}
\begin{axiom} P115c. If (not x=0) and x*y = 1 then y = inv(x).  

\end{axiom}
\begin{axiom} P115d. If not x=0 then inv(inv(x)) = x.

\end{axiom}
\begin{axiom} P116c. (-x)*y = -(x*y) = x*(-y).

\end{axiom}

\begin{axiom}. -(x+y) = -x + -y. 

\end{axiom}
\begin{axiom}. x*0 = 0. 

\end{axiom}

\begin{theorem}
. (c + d) + (e  + f) = (c + e) + (f   + d).
\end{theorem}\begin{proof}
 (c + d) + (e  + f)	= ((c +  d) +  e)  + f 
= ( c + (d  +  e)) + f 
= ( c + (e  +  d)) + f 
= ((c +  e) +  d)  + f 
=  (c +  e) + (f   + d).
\end{proof}


\begin{theorem}
. (x*x) - (y*y) = (x+y)*(x-y).
\end{theorem}\begin{proof}
 (x+y)*(x-y) 	= (x+y)*(x+ -y) = (((x+y)*x)+ ((x+y)*(-y))) = ( ((x*x) + (y*x)) + ((x*(-y)) + (y*(-y))) )
= ( ((x*x) + (y*x)) + ((x*(-y)) + -(y*y)) )
= ( (x*x) + ((y*x) + ((x*(-y)) + -(y*y))) )
= ( (x*x) + (((y*x) + (x*(-y))) + -(y*y) ) )
= ( (x*x) + (((y*x) + -(x*y)) + -(y*y) ) )
= ( (x*x) + (((y*x) + -(y*x)) + -(y*y) ) )
= ( (x*x) + (0 + -(y*y) ) )
= (x*x) - (y*y).
\end{proof}








\begin{signature}. A complex number is a notion.

\end{signature}
Let u,v,w, u2 denote complex numbers.
\begin{signature}. Let x,y be real numbers. (x,y) is a complex number.

\end{signature}
\begin{signature}. u ++ v is a complex number.

\end{signature}
\begin{signature}. u ** v is a complex number.

\end{signature}
\begin{axiom}. If u is a complex number then there exist real numbers x,y such that u = (x,y).

\end{axiom}
\begin{axiom} CAdd. (x, y) ++ (a, b) = (x + a, y + b) .

\end{axiom}
\begin{axiom} CMult. (x, y) ** (a, b) = ((x*a) - (y*b), (x*b) + (y*a)).

\end{axiom}




\begin{theorem}
 CmpFld2.  u ++ v = v ++ u.
\end{theorem}\begin{proof}
 Let x,y,a,b be real numbers such that (u = (x,y) and v = (a,b)).
Then u ++ v = (x,y) ++ (a,b).
Then (x,y) ++ (a,b) = (x + a, y + b) (by CAdd).	
(x + a, y + b) = (a + x, b + y). 
(a + x, b + y) = (a,b) ++ (x,y) (by CAdd).
(a,b) ++ (x,y) = v ++ u. 
Then u ++ v = v ++ u.
\end{proof}
 



\begin{theorem}
 CmpFld3. (u ++ v) ++ w = u ++ (v ++ w).
\end{theorem}\begin{proof}
 Let x,y,a,b,r,s be real numbers such that (u = (x,y) and v = (a,b) and w = (r,s)).
Then (u ++ v) ++ w = ((x+a)+r, (y+b)+s) (by CAdd). 
((x+a)+r, (y+b)+s)= (x+(a+r), y+(b+s)).
(x+(a+r), y+(b+s)) = u ++ (v ++ w) (by CAdd).
Then (u++v)++w=u++(v++w). 
\end{proof}


\begin{theorem}
 CmpFld4. u ++ (0,0) = u.
\end{theorem}\begin{proof}
 
Let x,y be real numbers such that u = (x,y). Then u ++ (0,0) = (x+0,y+0) (by CAdd). (x+0,y+0) = (x,y) = u. 
\end{proof}




\begin{theorem}
 CmpFld5. For every complex number u there exists a complex number v such that u ++ v = (0,0).
\end{theorem}\begin{proof}
 Let u be a complex number.
Take real numbers x,y such that u = (x,y). Take v = (-x, -y). Then  u ++ v = (x-x, y-y)(by CAdd) . (x-x, y-y) = (0,0). 
Then v is a complex number such that u++v = (0,0).

\end{proof}


\begin{theorem}
 CmpFld6. u**v = v**u.
\end{theorem}\begin{proof}
 Let x,y,a,b be real numbers such that (u = (x,y) and v = (a,b)).
Then u**v =  ((x*a) - (y*b), (x*b) + (y*a)) =  ((a*x) - (b*y), (a*y) + (b*x)) = v**u. Then u**v = v**u. \end{proof}


\begin{theorem}
 CmpFld7. (u**v)**w = u**(v**w).
\end{theorem}\begin{proof}
 Let x,y,a,b,r,s be real numbers such that (u = (x,y) and v = (a,b) and w = (r,s)). 
(u**v)**w = (((x*a) - (y*b), (x*b) + (y*a))) ** w 
= (  ( ((x*a)  - (y*b))*r ) - ( ((x*b) + (y*a))*s )  ,  ( ((x*a)  - (y*b))*s ) + ( ((x*b) + (y*a))*r ) )
= (  ( ((x*a) + -(y*b))*r ) - ( ((x*b) + (y*a))*s )  ,  ( ((x*a) + -(y*b))*s ) + ( ((x*b) + (y*a))*r ) )
= (  ( ((x*a) + -(y*b))*r ) - ( ((x*b)*s) + ((y*a)*s) )  ,  ( ((x*a) + -(y*b))*s ) + ( ((x*b)*r) + ((y*a)*r) )  )
= (  ( ((x*a)*r) + ((-(y*b))*r) ) - ( ((x*b)*s) + ((y*a)*s) )  ,  ( ((x*a)*s) + ((-(y*b))*s) ) + ( ((x*b)*r) + ((y*a)*r) )  ) 
= (  ( ((x*a)*r) + -((y*b)*r) ) - ( ((x*b)*s) + ((y*a)*s) )  ,  ( ((x*a)*s) - ((y*b)*s) ) + ( ((x*b)*r) + ((y*a)*r) )  )
= (  ( ((x*a)*r) + -((y*b)*r) ) + ( -((x*b)*s) + -((y*a)*s) )  ,  ( ((x*a)*s) + -((y*b)*s) ) + ( ((x*b)*r) + ((y*a)*r) )  ).
Let c = (x*a)*r and d = -((y*b)*r) and e = -((x*b)*s) and f = -((y*a)*s) and g = ((x*a)*s) and  h = -((y*b)*s) and  j = ((x*b)*r) and k = ((y*a)*r).
Then (u**v)**w =  ( (c + d) + (e  + f) , (g + h) +  (j + k)  )
=  ( (c + e) + (d + f) , (g +  j) + (h  + k) )
= ( (c + e) + (f + d) , (g + j) + (k + h) ).
Then v ** w = ((a*r) - (b*s),(a*s) + (b*r)) (by CMult).
u**(v**w) = u ** ((a,b) ** (r,s)) 
= u ** (  ((a*r) - (b*s) , (a*s) + (b*r))  )
= (  ( x * ((a*r) -(b*s)) ) - (y*((a*s) + (b*r))) , (x*((a*s) + (b*r))) + (y*((a*r) - (b*s))))
= ((x*((a*r) + -(b*s))) + -(y*((a*s) + (b*r))) , (x*((a*s) 	+ (b*r))) + (y*((a*r) + -(b*s))))
= (  ( (x*(a*r)) + (x*(-(b*s))) ) + -( (y*(a*s)) + (y*(b*r)) )  , ( (x*(a*s)) + (x*(b*r)) ) + ( (y*(a*r)) + (y*(-(b*s))) ) )
= (  ( (x*(a*r)) + -(x*(b*s)) ) + ( -(y*(a*s)) + -(y*(b*r)) )  ,  ( (x*(a*s)) + (x*(b*r)) ) + ( (y*(a*r)) + -(y*(b*s)) ) )
= (  ( ((x*a)*r) + -((x*b)*s) ) + ( -((y*a)*s) + -((y*b)*r) )  ,  ( ((x*a)*s) + ((x*b)*r) ) + ( ((y*a)*r) + -((y*b)*s) ) )
= ( (c + e) + (f + d) , (g + j) + (k + h) ) = (u**v)**w.
Then (u**v)**w = u**(v**w). \end{proof}



\begin{theorem}
 CmpFld8. u**(1,0) = u.
\end{theorem}\begin{proof}
 Let x,y be real numbers such that u = (x,y). Then u ** (1,0) = ((x,y)) ** (1,0) = ((x*1)-(y*0) , (x*0)+(y*1)). 
Let a = x*1 and b = y*0 and c = x*0 and d = y*1.

((x*1)-(y*0) , (x*0)+(y*1)) = (a-b, c+d).
Then a = x and b = 0 and c = 0 and d = y.
(a-b, c+d) = (x-0 , 0+y) = (x,y) = u .
Then u**(1,0) = u.
\end{proof}







\begin{signature}. x $<$ y is an atom.

\end{signature}
Let x $>$ y stand for y $<$ x.
Let x $\leq$ y stand for x$<$y or x=y.
Let x $\geq$ y stand for y $\leq$ x.

\begin{axiom} InEqCompl. Then x$<$y or y$<$x or x = y.

\end{axiom}
\begin{axiom}. If x$<$y then not x=y.

\end{axiom}

\begin{axiom} Trans. If x$<$y and y$<$z then x$<$z.

\end{axiom}


\begin{axiom} InEqAdd. If y$<$z then x+y$<$x+z.

\end{axiom}
\begin{axiom} InEqMult. If x$>$0 and y$>$0 then x*y$>$0.

\end{axiom}


\begin{axiom} P118d. If not x = 0 then x*x $>$ 0.

\end{axiom}
\begin{axiom} P118e1. If 0$<$y then 0 $<$ inv(y).

\end{axiom}

\begin{axiom} SqrtEind. If x*x = y*y and x$\geq$0 and y $\geq$ 0 then x=y.

\end{axiom}

\begin{theorem}
 SqrtMon. If x*x $>$ y*y and x $>$ 0  and y $\geq$ 0 then x$>$y.
\end{theorem}\begin{proof}
	Assume x*x $>$ y*y and x$>$0 and y $\geq$ 0.
Thus  -(y*y) + (x*x) $>$ -(y*y) + (y*y) (by InEqAdd). 
Hence (x*x) - (y*y) $>$ 0.
Then  (x + y) * (x - y) $>$ 0.
x + y $>$ 0.
proof. 	x $>$ 0. Then x + y $>$ y.
case y $>$ 0. Then x + y $>$ 0 (by Trans). end.
case y = 0. Then x + y  $>$ 0 . end.
end.
Thus inv(x + y) $>$ 0. Hence inv(x + y) * ((x + y) * (x - y)) $>$ 0.
We have inv(x + y) * ((x + y) * (x - y)) 
= (inv(x + y) * (x + y)) * (x - y)
= 1 * (x - y) = x - y.
Then x - y $>$ 0.
Thus (x + -y) + y $>$ 0 + y.
Hence x + (-y + y) $>$ 0 + y.
\end{proof}


\begin{lemma} LeqTrans. If x$\leq$y and y$\leq$z then x$\leq$z.

\end{lemma}
\begin{proof}
 	Obvious.\end{proof}



\begin{lemma} LeqAdd. 	If y $\leq$ z then x + y $\leq$ x+z.

\end{lemma}
\begin{proof}
			Case y $<$ z. Then x + y $\leq$ x + z (by InEqAdd). end.
Case y = z. Then x+y = x+z. end.
\end{proof}


\begin{lemma} LeqAdd2.  If a $\leq$ b and c $\leq$ d then a+c $\leq$ b+d.

\end{lemma}
\begin{proof}
			Assume a $\leq$ b and c $\leq$ d. 
Case a=b. Then a+c = b+c. b+c $\leq$ b+d (by LeqAdd). Thus a+c $\leq$ b+d. end.
Case a $<$ b. Then c+a $<$ c+b. Thus c+a $\leq$ c+b. Hence a+c $\leq$ b+c. b+c $\leq$ b+d (by LeqAdd).
Then a+c $\leq$ b+d (by LeqTrans). end.
\end{proof}



\begin{lemma} 21O. If (not x = 0) or (not y = 0) then (x*x) + (y*y)$>$0.

\end{lemma}
\begin{proof}
 
Assume (not x=0) or (not y=0).
Let a = x*x and b = y*y.
Then a$>$ 0 or b$>$0.
Then a$\geq$ 0 and b$\geq$0.
Let us show that a + b $>$ 0.
Case a$>$0. Then a+b$\geq$a+0$\geq$a$>$0. end.
Case b$>$0. Then a+b$\geq$0+b$\geq$b$>$0. end.
end.
\end{proof}


\begin{theorem}
 CmpFld9. If not u = (0,0) then there exists a complex number v such that u**v = (1,0).
\end{theorem}\begin{proof}
 Assume not u = (0,0).
Let x,y be real numbers such that u = (x,y).

Then (not x=0) or (not y=0).
Then (x*x) + (y*y)$>$0 (by 21O).
Then not (x*x) + (y*y) = 0 .

Let d = inv((x*x) + (y*y)) and 
v = (x*d , -y*d). 
Then u**v = ( (x*(x*d)) - (y*(-y*d)) , (x*(-y*d)) + (y*(x*d)))
= ( (x*(x*d)) + -(-y*(y*d)) , -(x*(y*d)) + (y*(x*d)))
= ( (x*(x*d)) + (y*(y*d)) , -(x*(y*d)) + (y*(x*d)))
= ( ((x*x)*d) + ((y*y)*d) , -((x*y)*d) + ((y*x)*d))
= ( ((x*x) + (y*y))*d , -((x*y)*d) + ((x*y)*d))
= ( ((x*x) + (y*y))*inv((x*x) + (y*y)) , 0)
= (1,0).
Then u**v = (1,0).
Then v is a complex number such that u**v = (1,0).
If not u = (0,0) then there exists a complex number w such that u**w = (1,0).
\end{proof}





\begin{theorem}
 CmpFld10. u**(v++w) = (u**v) ++ (u**w).
\end{theorem}\begin{proof}
	Let x,y,a,b,r,s be real numbers such that (u = (x,y) and v = (a,b) and w = (r,s)).
Let c = x*a and d = x*r and e = -(y*b) and f = -(y*s) and g = x*b and  h = x*s and  j = y*a and k = y*r.
v++w = (a+r,b+s) (by CAdd).
u**v = (x,y) ** (a,b) = ((x*a) - (y*b), (x*b) + (y*a)) (by CMult).  
u**w = (x,y) ** (r,s) = ((x*r) - (y*s) , (x*s) + (y*r)) (by CMult). 
Then u**(v++w) 	= (x,y) ** (a+r,b+s) = ( (x*(a+r)) - (y*(b+s)) , (x*(b+s)) + (y*(a+r)) ).
( (x*(a+r)) - (y*(b+s)) , (x*(b+s)) + (y*(a+r)) )
.= ( ((x*a) + (x*r)) - (y*(b+s)) , (x*(b+s)) + (y*(a+r)) )
.= ( ((x*a) + (x*r)) - ((y*b) + (y*s)) , (x*(b+s)) + (y*(a+r)) ).

( ((x*a) + (x*r)) - ((y*b) + (y*s)) , (x*(b+s)) + (y*(a+r)) )
= ( ((x*a) + (x*r)) - ((y*b) + (y*s)) , ((x*b) + (x*s)) + (y*(a+r)) ) (by Dis1).

( ((x*a) + (x*r)) - ((y*b) + (y*s)) , ((x*b) + (x*s)) + (y*(a+r)) )
= ( ((x*a) + (x*r)) - ((y*b) + (y*s)) , ((x*b) + (x*s)) + ((y*a) + (y*r)) ) (by Dis1).

( ((x*a) + (x*r)) - ((y*b) + (y*s)) , ((x*b) + (x*s)) + ((y*a) + (y*r)) )
= (((x*a) + (x*r)) + -((y*b) + (y*s)) , ((x*b) + (x*s)) + ((y*a) + (y*r)) )
= (((x*a) + (x*r)) + (-(y*b) + -(y*s)) , ((x*b) + (x*s)) + ((y*a) + (y*r)) )
= ((c + d) + (e + f) , (g + h) + (j + k))
= ((c + e) + (d + f) , (g + j) + (h + k))
= (c+e,g+j) ++ (d+f,h+k)
= ((x*a) + -(y*b), (x*b) + (y*a)) ++ ((x*r) + -(y*s) , (x*s) + (y*r))
= ((x*a) - (y*b), (x*b) + (y*a)) ++ ((x*r) - (y*s) , (x*s) + (y*r))
= ((x,y) ** (a,b)) ++ ((x,y) ** (r,s)) 
= (u**v) ++ (u**w).
Then u**(v++w) = (u**v) ++ (u**w).
\end{proof}


\begin{theorem}
 ZeroMult. u**(0,0) = (0,0).
\end{theorem}\begin{proof}
 Let x,y be real numbers such that u = (x,y). Then u**(0,0) = ((x*0) - (y*0), (x*0) + (y*0)) (by CMult). 
Let a=x*0 and b=y*0.
((x*0) - (y*0), (x*0) + (y*0)) = (a-b,a+b) = (0 - 0, 0 + 0) = (0,0). 
Then u**(0,0) = (0,0). 
\end{proof}


\begin{theorem}
. (a,0) ++ (b,0) = (a+b,0) and (a,0)**(b,0) = (a*b,0).
\end{theorem}\begin{proof}
 	(a,0) ++ (b,0) = (a+b,0+0) = (a+b,0) (by CAdd).
(a,0) ** (b,0) = ((a*b) - (0*0), (a*0) + (0*b)) (by CMult).
Let c = a*0 and d = b*0 and e = 0*0.
Then c = 0 and d = 0 and e = 0.
Then ((a*b) - e, c + d) = ((a*b) -0, 0 + 0) = ((a*b )+ -0,0) = (a*b,0) . \end{proof}


\begin{signature}. ! is a complex number such that ! = (0,1).

\end{signature}

\begin{theorem}
. !**! = (-1,0).
\end{theorem}\begin{proof}
 	!**! .= (0,1) ** (0,1) 
.= ((0*0) - (1*1),( 0*1) + (1*0)) (by CMult)
.=((0*0) - (1*1),( 0*1) + (1*0)) 
.= (0 - 1, 0+0) 
.= (0+(-1),0).
Then (0+(-1),0) = ((-1),0). 
\end{proof}


Let x+y! stand for (x,0) ++ ((y,0)**!).


\begin{theorem}
. a+b! = (a,b).
\end{theorem}\begin{proof}
 	(b,0)**(0,1) = ((b*0)-(0*1) , (b*1)+(0*0)) (by CMult).
a+b! = (a,0) ++ ((b,0)**(0,1)). 

(a,0) ++ ((b,0)**(0,1))
= (a,0) ++ ((b*0)-(0*1) , (b*1)+(0*0)) (by CMult).

(a,0) ++ ((b*0)-(0*1) , (b*1)+(0*0))
= (a,0) ++ (0-0,b+0) 
= (a,0) ++ (0,b).

(a,0) ++ (0,b)
= (a+0,0+b) 
= (a,b).
\end{proof}



\begin{signature}. Conj(u) is a complex number.

\end{signature}
\begin{signature}. Re(u) is a real number.

\end{signature}
\begin{signature}. Im(u) is a real number.

\end{signature}
\begin{axiom}. Conj((x,y)) = (x,-y).

\end{axiom}
\begin{axiom}. Re((x,y)) = x.

\end{axiom}
\begin{axiom}. Im((x,y)) = y.

\end{axiom}

\begin{theorem}
 Conj1. Conj(u++v) = Conj(u) ++ Conj(v).
\end{theorem}\begin{proof}
 	Let x,y,a,b be real numbers such that (u = (x,y) and v = (a,b)).
Then u++v = (x+a,y+b) (by CAdd).
Then Conj(u ++ v) = Conj((x+a,y+b)) = (x+a,-(y+b)) = (x+a, -y + -b) = (x,-y) ++ (a,-b) = Conj(u) ++ Conj(v).
Then Conj(u++v) = Conj(u) ++ Conj(v).
\end{proof}
		
\begin{theorem}
 Conj2. Conj(u**v) = Conj(u) ** Conj(v).
\end{theorem}\begin{proof}
 	Let x,y,a,b be real numbers such that (u = (x,y) and v = (a,b)).
Then u ** v = ((x*a)-(y*b),(x*b)+(y*a)) (by CMult).
Then Conj(u ** v) 	= Conj(((x*a)-(y*b),(x*b)+(y*a))) 
= ((x*a) - (y*b)       ,-((x*b)+(y*a))) 
= ((x*a) - (y*b)       , -(x*b)+ -(y*a))
= ((x*a) - (-(-(y*b))) , -(x*b)+ -(y*a))
= ((x*a) - ( -((-y)*b) ) , (x*(-b))+ ((-y)*a))
= ((x*a) - ((-y)*(-b)) , (x*(-b))+ ((-y)*a))
= (x,-y) ** (a,-b)
= Conj(u) ** Conj(v). 
Then Conj(u**v) = Conj(u) ** Conj(v).
\end{proof}


\begin{theorem}
 Conj3. u ++ Conj(u) = (Re(u)+Re(u),0) and u ++ ((-1,0)**Conj(u)) = (0,Im(u)+Im(u)).
\end{theorem}\begin{proof}
 Let x,y be real numbers such that u=(x,y).
Then u ++ Conj(u) = (x,y) ++ Conj((x,y)) = (x,y) ++ (x,-y) = (x+x,y-y) = (x+x,0) = (Re(u)+Re(u),0).
(-1,0)**(x,-y) = ( ((-1)*x) - (0*(-y)) , ((-1)*(-y)) + (0*x)) (by CMult).
u ++ ((-1,0)**Conj(u))	= (x,y) ++ ((-1,0)**Conj((x,y))) = (x,y) ++ ((-1,0)**(x,-y)) 
= (x,y) ++ ( ((-1)*x) - (0*(-y)) , ((-1)*(-y)) + (0*x))
= (x,y) ++ (-x - 0, -(-y) + 0)
= (x,y) ++ (-x , y) = (x-x,y+y) = (0,Im(u)+Im(u)).
Then u ++ Conj(u) = (Re(u)+Re(u),0) and u ++ ((-1,0)**Conj(u)) = (0,Im(u)+Im(u)).
\end{proof}
	   

\begin{theorem}
 Conj4. If not u = (0,0) then there exists a real number z such that (z$>$0 and u ** Conj(u) = (z,0)).
\end{theorem}\begin{proof}
	Let not u = (0,0).
Take real numbers x,y such that u = (x,y).

Then (x*x)$>$ 0 or (y*y)$>$0. Take z = (x*x) + (y*y). 
Then z $>$ 0 (by 21O).
u**Conj(u) = (x,y)**Conj((x,y)) = (x,y)**(x,-y) = ((x*x) - (y*(-y)) , (x*(-y)) + (y*x)) = ((x*x) + -(-(y*y)) , -(x*y) + (x*y)) = ((x*x) + (y*y), 0) = (z,0).
z is a real number.
Then z is a real number and (z$>$0 and u**Conj(u) = (z,0)).
\end{proof}


\begin{theorem}
 Conj5. Conj(Conj(u)) = u.
\end{theorem}\begin{proof}
 	Let x,y be real numbers such that u = (x,y).
Then Conj(Conj(u)) = Conj(Conj((x,y))) = Conj((x,-y)) = (x,--y) = (x,y) = u. \end{proof}



\begin{signature}. |u| is a real number such that (|u|*|u|,0) = u**Conj(u) and |u| $\geq$ 0. 

\end{signature}
\begin{axiom}  EindAbs. If (x*x,0) = u**Conj(u) and x $\geq$ 0 then x = |u|.

\end{axiom}



\begin{theorem}
 Abs1. If not u = (0,0) then |u| $>$ 0.
\end{theorem}\begin{proof}
  Let z be a real number such that (z $>$ 0 and u**Conj(u) = (z,0)). Then (|u|*|u|,0) = (z,0).
Assume |u| = 0. Then z = 0*0 = 0. Contradiction. 
Then |u| $>$ 0.
\end{proof}


\begin{theorem}
 Abs2. |(0,0)| = 0.
\end{theorem}\begin{proof}
 	(0,0) ** Conj((0,0)) = Conj((0,0)) ** (0,0) = (0,0). 
Let us show that |(0,0)|*|(0,0)| = 0.
Assume |(0,0)| $>$ 0. Then not |(0,0)| = 0 .Then |(0,0)|*|(0,0)| $>$ 0. Then not |(0,0)|*|(0,0)| = 0. Contradiction.end. 
Hence |(0,0)| = 0.
\end{proof}


\begin{theorem}
 Abs3. |u| = |Conj(u)|.
\end{theorem}\begin{proof}
 	|u| is a real number such that ((|u|*|u|,0) = u**Conj(u) and |u| $\geq$ 0). 
(|u|*|u|,0) = Conj(Conj(u))**Conj(u) = Conj(u)**Conj(Conj(u)).
Let v = Conj(u). Then |u| is real number such that ((|u|*|u|,0) = v**Conj(v) and |u| $\geq$ 0). 
Hence |u| = |v|.
\end{proof}


\begin{theorem}
 Abs4. |u**v| = |u| * |v|.
\end{theorem}\begin{proof}
 	Case u = (0,0) or v = (0,0). Then v**u = (0,0) or u**v = (0,0) (by ZeroMult). Hence |u**v| = 0.
|u| = 0 or |v| = 0 (by Abs2). Hence |u| * |v| = 0.
Then  |u**v| = |u|*|v|. end.
Case (not u=(0,0)) and (not v = (0,0)). Then |u| $>$ 0 and |v| $>$ 0 (by Abs1). Let z = |u| * |v|. Then z $>$ 0.
z*z = (|u|*|v|) * (|u|*|v|) = ((|u|*|v|)*|u|)*|v| = (|u|*(|v|*|u|))*|v|
= (|u|*(|u|*|v|))*|v| = ((|u|*|u|)*|v|)*|v| = (|u|*|u|)*(|v|*|v|).
Hence (z*z,0)	= ((|u|*|u|)*(|v|*|v|), 0) = (|u|*|u|,0) ** (|v|*|v|,0)
= (u  **  Conj(u)) ** (v   ** Conj(v)) 
= ((u **  Conj(u)) **  v)  ** Conj(v) 
= ( u ** (Conj(u)  **  v)) ** Conj(v) 
= ( u ** (v  **  Conj(u))) ** Conj(v) 
= ((u **  v) **  Conj(u))  ** Conj(v) 
= ( u **  v) **  (Conj(u)   ** Conj(v)) = (u**v)**Conj(u**v).
Then z = |u**v|. Then |u| * |v| = |u**v|. end.
\end{proof}


\begin{theorem}
 Abs5. |(Re(u),0)| $\leq$ |u|.
\end{theorem}\begin{proof}
 	Let x,y be real numbers such that u = (x,y). 
Case y = 0. Then |u| = |(x,y)| = |(Re(u),0)|. end.
Case not y = 0. Then not u = (0,0). Thus |u| $>$ 0.
(|u|*|u|,0) = u**Conj(u) = (x,y) ** (x,-y) = ((x*x) - (y*(-y)) , (x*(-y)) + (y*x)) = ((x*x) + -(-(y*y)) , -(x*y) + (x*y)) 
= ((x*x) + (y*y), 0). Then |u|*|u| = (x*x)+(y*y).
(|(Re(u),0)|*|(Re(u),0)|,0) = (Re(u),0)**Conj((Re(u),0)) = (Re(u),0)**(Re(u),-0) = (Re(u),0)**(Re(u),0) = (x,0)**(x,0) = (x*x,0).
Then |(Re(u),0)|*|(Re(u),0)| = x*x.
Hence (|u|*|u|) -  (|(Re(u),0)|*|(Re(u),0)|) = ((x*x) + (y*y)) - (x*x) = ((y*y) + (x*x)) + -(x*x) = (y*y) + ((x*x) + -(x*x)) = (y*y) + 0 = y*y and y*y $>$ 0.
Thus (|u|*|u|) -  (|(Re(u),0)|*|(Re(u),0)|) $>$ 0.
Hence |u|*|u| $>$ |(Re(u),0)|*|(Re(u),0)| and |u| $>$ 0 and |(Re(u),0)| $\geq$ 0.
Thus  |u| $>$ |(Re(u),0)| (by SqrtMon).
end.
\end{proof}


\begin{lemma} AbsMon. Re(u) $\leq$ |(Re(u),0)|.

\end{lemma}
\begin{proof}

Let x,y be real numbers such that u = (x,y).
Let v = (x, 0).
Then Re(u) = x = Re(v).
Case x$>$0.

Then (x*x,0) = v**Conj(v).
Then x = |v| (by EindAbs).
Then Re(u)=Re(v)=|v|=|(Re(v),0)| = |(Re(u),0)|.
Then Re(u) $\leq$ |(Re(u),0)|.
end.
Case x $\leq$ 0.
Then |(Re(u),0)| $\geq$0.
x $\leq$ 0 $\leq$ |(Re(u),0)|.
end.
\end{proof}


\begin{theorem}
 Abs6. |u ++ v| $\leq$ |u| + |v|.
\end{theorem}\begin{proof}
 (|u ++ v|*|u ++ v|,0)	.= (u ++ v) ** Conj(u ++ v)
.=  ((u ++ v) ** Conj(u)) ++ ((u ++ v) ** Conj(v)) 
.=  (Conj(u) ** (u ++ v)) ++ (Conj(v) ** (u ++ v)) 
.=  ((Conj(u) ** u) ++ (Conj(u) ** v)) ++ ((Conj(v) ** u) ++ (Conj(v) ** v)) (by CmpFld10)
.=  ((u ** Conj(u)) ++ (v ** Conj(u))) ++ ((u ** Conj(v)) ++ (v ** Conj(v)))
.=  (u ** Conj(u)) ++ ((v ** Conj(u)) ++ ((u ** Conj(v)) ++ (v ** Conj(v))))
.=  (u ** Conj(u)) ++ (((v ** Conj(u)) ++ (u ** Conj(v))) ++ (v ** Conj(v)))
.=  (u ** Conj(u)) ++ ((v ** Conj(v)) ++ ((v ** Conj(u)) ++ (u ** Conj(v)))) 
.=  (u ** Conj(u)) ++ ((v ** Conj(v)) ++ ((v ** Conj(u)) ++ (Conj(Conj(u)) ** Conj(v))))
.=  (u ** Conj(u)) ++ ((v ** Conj(v)) ++ ((v ** Conj(u)) ++ Conj(Conj(u) ** v) ))
.=  (u ** Conj(u)) ++ ((v ** Conj(v)) ++ ((v ** Conj(u)) ++ Conj(v ** Conj(u)) ))
.=  (u ** Conj(u)) ++ ((v ** Conj(v)) ++ (Re(v ** Conj(u))+Re(v ** Conj(u)),0) )
.=  (|u|*|u|,0) ++ ((|v|*|v|,0) ++ (Re(v ** Conj(u))+Re(v ** Conj(u)),0) )
.=  ((|u|*|u|,0) ++ (|v|*|v|,0)) ++ (Re(v ** Conj(u))+Re(v ** Conj(u)),0)
.= ((|u|*|u|) + (|v|*|v|),0+0) ++ (Re(v ** Conj(u))+Re(v ** Conj(u)),0)
.=(((|u|*|u|) + (|v|*|v|)) + (Re(v ** Conj(u))+Re(v ** Conj(u))), 0 + 0)
.=(((|u|*|u|) + (|v|*|v|)) + (Re(v ** Conj(u))+Re(v ** Conj(u))), 0 ).
Thus |u ++ v|*|u ++ v| = (((|u|*|u|) + (|v|*|v|)) + Re(v ** Conj(u))) + Re(v ** Conj(u)).


Then Re(v ** Conj(u)) $\leq$ |(Re(v ** Conj(u)),0)| (by AbsMon). |(Re(v ** Conj(u)),0)| $\leq$ |v**Conj(u)| (by Abs5).
Hence Re(v ** Conj(u)) $\leq$ |v**Conj(u)| (by LeqTrans).
Thus ((|u|*|u|) + (|v|*|v|)) + Re(v ** Conj(u)) $\leq$ ((|u|*|u|) + (|v|*|v|)) + |v**Conj(u)| (by LeqAdd).
Thus (((|u|*|u|) + (|v|*|v|)) + Re(v ** Conj(u))) + Re(v ** Conj(u)) $\leq$ (((|u|*|u|) + (|v|*|v|)) + |v**Conj(u)|)+|v**Conj(u)| (by LeqAdd2).
(((|u|*|u|) + (|v|*|v|)) + |v**Conj(u)|)+|v**Conj(u)|
= ((|u|*|u|) + (|v|*|v|)) + (|v**Conj(u)|+|v**Conj(u)|) 
= ((|u|*|u|) + (|v|*|v|)) + ((|v|*|Conj(u)|) + (|v|*|Conj(u)|))
= ((|u|*|u|) + (|v|*|v|)) + ((|v|*|u|) + (|v|*|u|))
= ((|u|*|u|) + (|v|*|u|)) + ((|v|*|v|) + (|v|*|u|))
= ((|u| + |v|) * |u|) + (|v|*(|v|+|u|))
= (|u| * (|u| + |v|)) + (|v| * (|u| + |v|))
= (|u| + |v|)*(|u| + |v|).

Hence |u ++ v|*|u ++ v| $\leq$ (|u| + |v|)*(|u| + |v|).
0 $\leq$ |u| and 0 $\leq$ |v|. Thus 0 + 0 $\leq$ |u| + |v| (by LeqAdd2).
Case |u| + |v| $>$ 0. 
Case |u ++ v|*|u ++ v| $<$ (|u| + |v|)*(|u| + |v|). 
Then (|u| + |v|)*(|u| + |v|) $>$ |u ++ v|*|u ++ v| and |u ++ v| $\geq$ 0 and |u| + |v| $>$ 0 .
Thus |u ++ v| $<$ |u| + |v| (by SqrtMon). end.
Case |u ++ v|*|u ++ v| = (|u| + |v|)*(|u| + |v|). Then |u ++ v| = |u| + |v| (by SqrtEind).end.
end.
Case |u| + |v| = 0. Thus (|u| + |v|)*(|u| + |v|) = 0. Hence |u ++ v|*|u ++ v| $\leq$ 0. 
Let us show that |u ++ v| $\leq$ 0. 
proof. 	Assume |u ++ v| $>$ 0. Then |u ++ v|*|u ++ v| $>$ 0 (by InEqMult). Contradiction.
Thus not (|u ++ v| $>$ 0). Hence |u ++ v| $<$ 0 or |u ++ v| = 0 (by InEqCompl).
end.
Thus |u ++ v| $\leq$ |u| + |v|.
end.
\end{proof}

\end{forthel}
\end{document}