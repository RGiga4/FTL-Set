\documentclass{article}
\usepackage[english]{babel}
\usepackage{enumerate, latexsym, amssymb, amsmath}
\usepackage{framed, multicol}
\newenvironment{forthel}{\begin{leftbar}}{\end{leftbar}}

%%%%%%%%%% Start TeXmacs macros
\newcommand{\tmaffiliation}[1]{\\ #1}
\newcommand{\tmem}[1]{{\em #1\/}}
\newenvironment{enumeratenumeric}{\begin{enumerate}[1.] }{\end{enumerate}}
\newenvironment{proof}{\noindent\textbf{Proof\ }}{\hspace\rmul {\fill}$\Box$\medskip}
\newenvironment{quoteenv}{\begin{quote} }{\end{quote}}
\newtheorem{axiom}{\begin{axiom} }
\newtheorem{lemma}{Lemma}
\newtheorem{theorem}{Theorem}
\newtheorem{definition}{Definition}
\newtheorem{signature}{Signature}
\newtheorem{proposition}{Proposition}
%%%%%%%%%% End TeXmacs macros

\newcommand{\event}{UITP 2018}
\newcommand{\dom}{Dom}
\newcommand{\cmul}{\cdot}
\newcommand{\rmul}{\cdot}
\newcommand{\cadd}{+}
\newcommand{\radd}{+}
\newcommand{\fun}{aFunction}
\newcommand{\sym}{sym}
\newcommand{\halfline}{{\vspace{3pt}}}
\newcommand{\tab}{{\hspace{1cm}}}
\newcommand{\ball}[2]{B_{#1}(#2)}
\newcommand{\llbracket}{[}
\newcommand{\rrbracket}{]}
\newcommand{\less}[1]{<_{#1}}
\newcommand{\greater}[1]{>_{#1}}
\newcommand{\leeq}[1]{{\leq}_{#1}}
\newcommand{\supr}[1]{\mathrm{sup}_{#1}}
\newcommand{\RR}{\mathbb{R}}
\newcommand{\QQ}{\mathbb{Q}}
\newcommand{\ZZ}{\mathbb{Z}}
\newcommand{\NN}{\mathbb{N}}

\begin{document}

\title{The Complex Field}

\maketitle

\begin{forthel}
[set/-s] [element/-s] [number/-s]
\begin{signature} A real number is a notion.
\end{signature}
#Let F denote an ordered field.
Let x,y,z,a,b,c,d,e,f,g,h,j,k,r,s denote real numbers.

###Axiome fur field 

\begin{signature} Add. x \radd  y is a real number.
\end{signature}
\begin{axiom}  RComm. x\radd y = y\radd x.
\end{axiom}
\begin{axiom}  (x\radd y)\radd z = x\radd (y\radd z).
\end{axiom}
\begin{signature} Zero. 0 is a real number such that for every real number x x \radd  0 = x.
\end{signature}
\begin{signature} -x is a real number such that -x \radd  x = 0.
\end{signature}

Let x - y stand for x \radd  (-y).

\begin{signature} Mult. x \rmul  y is real number.
\end{signature}
\begin{axiom} x\rmul y = y\rmul x.
\end{axiom}
\begin{axiom} (x\rmul y)\rmul z = x\rmul (y\rmul z).
\end{axiom}
\begin{signature} One. 1 is a real number such that for every real number x x\rmul 1 = x.
\end{signature}
\begin{signature} Assume not x = 0. inv(x) is a real number such that inv(x)\rmul x = 1 .
\end{signature}
\begin{axiom} 1 is not equal to 0.
\end{axiom}

\begin{axiom}[Dis1] x\rmul (y\radd z)=(x\rmul y)\radd (x\rmul z).
\end{axiom}
\begin{axiom}[Dis2] (y\radd z)\rmul x=(y\rmul x)\radd (z\rmul x).
\end{axiom}

###weitere bewissene Aussagen
\begin{axiom}  [P114c] If x\radd y = 0 then y = -x.
\end{axiom}
\begin{axiom}  [P114d] -(-x) = x.# Drastische beschleunigung.
\end{axiom}
\begin{axiom}  [P115c] If (not x=0) and x\rmul y = 1 then y = inv(x).  
\end{axiom}
\begin{axiom}  [P115d] If not x=0 then inv(inv(x)) = x.
\end{axiom}
\begin{axiom} [P116c] (-x)\rmul y = -(x\rmul y) = x\rmul (-y).
\end{axiom}

\begin{axiom} -(x\radd y) = -x \radd  -y. 
\end{axiom}
\begin{axiom} x\rmul 0 = 0. 
\end{axiom}

Proposition. (c \radd  d) \radd  (e  \radd  f) = (c \radd  e) \radd  (f   \radd  d).
Proof. (c \radd  d) \radd  (e  \radd  f)	= ((c \radd   d) \radd   e)  \radd  f 
       	 	   				= ( c \radd  (d  \radd   e)) \radd  f 
       	    				= ( c \radd  (e  \radd   d)) \radd  f 
      						= ((c \radd   e) \radd   d)  \radd  f 
      						=  (c \radd   e) \radd  (f   \radd  d).
qed.

Proposition. (x\rmul x) - (y\rmul y) = (x\radd y)\rmul (x-y).
Proof. (x\radd y)\rmul (x-y) 	= (x\radd y)\rmul (x\radd  -y) = (((x\radd y)\rmul x)\radd  ((x\radd y)\rmul (-y))) = ( ((x\rmul x) \radd  (y\rmul x)) \radd  ((x\rmul (-y)) \radd  (y\rmul (-y))) )
					= ( ((x\rmul x) \radd  (y\rmul x)) \radd  ((x\rmul (-y)) \radd  -(y\rmul y)) )
					= ( (x\rmul x) \radd  ((y\rmul x) \radd  ((x\rmul (-y)) \radd  -(y\rmul y))) )
					= ( (x\rmul x) \radd  (((y\rmul x) \radd  (x\rmul (-y))) \radd  -(y\rmul y) ) )
					= ( (x\rmul x) \radd  (((y\rmul x) \radd  -(x\rmul y)) \radd  -(y\rmul y) ) )
					= ( (x\rmul x) \radd  (((y\rmul x) \radd  -(y\rmul x)) \radd  -(y\rmul y) ) )
					= ( (x\rmul x) \radd  (0 \radd  -(y\rmul y) ) )
					= (x\rmul x) - (y\rmul y).
qed.


### mehr axiome




\begin{signature} A complex number is a notion.
\end{signature}
Let u,v,w, u2 denote complex numbers.
\begin{signature} Let x,y be real numbers. (x,y) is a complex number.
\end{signature}
\begin{signature} u \cadd  v is a complex number.
\end{signature}
\begin{signature} u \cmul  v is a complex number.
\end{signature}
\begin{axiom} If u is a complex number then there exist real numbers x,y such that u = (x,y).
\end{axiom}
\begin{axiom}  [CAdd] (x, y) \cadd  (a, b) = (x \radd  a, y \radd  b) .
\end{axiom}
\begin{axiom}  [CMult] (x, y) \cmul  (a, b) = ((x\rmul a) - (y\rmul b), (x\rmul b) \radd  (y\rmul a)).
\end{axiom}




Proposition CmpFld2.  u \cadd  v = v \cadd  u.
Proof. Let x,y,a,b be real numbers such that (u = (x,y) and v = (a,b)).
Then u \cadd  v = (x,y) \cadd  (a,b).
Then (x,y) \cadd  (a,b) = (x \radd  a, y \radd  b) (by CAdd).	#Hier ein Problem was?
	    (x \radd  a, y \radd  b) = (a \radd  x, b \radd  y). 
	    (a \radd  x, b \radd  y) = (a,b) \cadd  (x,y) (by CAdd).
	    (a,b) \cadd  (x,y) = v \cadd  u. 
Then u \cadd  v = v \cadd  u.
qed. 



Proposition CmpFld3. (u \cadd  v) \cadd  w = u \cadd  (v \cadd  w).
Proof. Let x,y,a,b,r,s be real numbers such that (u = (x,y) and v = (a,b) and w = (r,s)).
Then (u \cadd  v) \cadd  w = ((x\radd a)\radd r, (y\radd b)\radd s) (by CAdd). 
((x\radd a)\radd r, (y\radd b)\radd s)= (x\radd (a\radd r), y\radd (b\radd s)).
(x\radd (a\radd r), y\radd (b\radd s)) = u \cadd  (v \cadd  w) (by CAdd).
Then (u\cadd v)\cadd w=u\cadd (v\cadd w). 
qed.

Proposition CmpFld4. u \cadd  (0,0) = u.
Proof. 
	Let x,y be real numbers such that u = (x,y). Then u \cadd  (0,0) = (x\radd 0,y\radd 0) (by CAdd). (x\radd 0,y\radd 0) = (x,y) = u. 
qed.



Proposition CmpFld5. For every complex number u there exists a complex number v such that u \cadd  v = (0,0).
Proof. Let u be a complex number.
	Take real numbers x,y such that u = (x,y). Take v = (-x, -y). Then  u \cadd  v = (x-x, y-y)(by CAdd) . (x-x, y-y) = (0,0). 
	Then v is a complex number such that u\cadd v = (0,0).
	
qed.

Proposition CmpFld6. u\cmul v = v\cmul u.
Proof. Let x,y,a,b be real numbers such that (u = (x,y) and v = (a,b)).
Then u\cmul v =  ((x\rmul a) - (y\rmul b), (x\rmul b) \radd  (y\rmul a)) =  ((a\rmul x) - (b\rmul y), (a\rmul y) \radd  (b\rmul x)) = v\cmul u. Then u\cmul v = v\cmul u. qed.

Proposition CmpFld7. (u\cmul v)\cmul w = u\cmul (v\cmul w).
Proof. Let x,y,a,b,r,s be real numbers such that (u = (x,y) and v = (a,b) and w = (r,s)). 
(u\cmul v)\cmul w = (((x\rmul a) - (y\rmul b), (x\rmul b) \radd  (y\rmul a))) \cmul  w 
		  = (  ( ((x\rmul a)  - (y\rmul b))\rmul r ) - ( ((x\rmul b) \radd  (y\rmul a))\rmul s )  ,  ( ((x\rmul a)  - (y\rmul b))\rmul s ) \radd  ( ((x\rmul b) \radd  (y\rmul a))\rmul r ) )
	  	  = (  ( ((x\rmul a) \radd  -(y\rmul b))\rmul r ) - ( ((x\rmul b) \radd  (y\rmul a))\rmul s )  ,  ( ((x\rmul a) \radd  -(y\rmul b))\rmul s ) \radd  ( ((x\rmul b) \radd  (y\rmul a))\rmul r ) )
		  = (  ( ((x\rmul a) \radd  -(y\rmul b))\rmul r ) - ( ((x\rmul b)\rmul s) \radd  ((y\rmul a)\rmul s) )  ,  ( ((x\rmul a) \radd  -(y\rmul b))\rmul s ) \radd  ( ((x\rmul b)\rmul r) \radd  ((y\rmul a)\rmul r) )  )
		  = (  ( ((x\rmul a)\rmul r) \radd  ((-(y\rmul b))\rmul r) ) - ( ((x\rmul b)\rmul s) \radd  ((y\rmul a)\rmul s) )  ,  ( ((x\rmul a)\rmul s) \radd  ((-(y\rmul b))\rmul s) ) \radd  ( ((x\rmul b)\rmul r) \radd  ((y\rmul a)\rmul r) )  ) 
		  = (  ( ((x\rmul a)\rmul r) \radd  -((y\rmul b)\rmul r) ) - ( ((x\rmul b)\rmul s) \radd  ((y\rmul a)\rmul s) )  ,  ( ((x\rmul a)\rmul s) - ((y\rmul b)\rmul s) ) \radd  ( ((x\rmul b)\rmul r) \radd  ((y\rmul a)\rmul r) )  )
		  = (  ( ((x\rmul a)\rmul r) \radd  -((y\rmul b)\rmul r) ) \radd  ( -((x\rmul b)\rmul s) \radd  -((y\rmul a)\rmul s) )  ,  ( ((x\rmul a)\rmul s) \radd  -((y\rmul b)\rmul s) ) \radd  ( ((x\rmul b)\rmul r) \radd  ((y\rmul a)\rmul r) )  ).
Let c = (x\rmul a)\rmul r and d = -((y\rmul b)\rmul r) and e = -((x\rmul b)\rmul s) and f = -((y\rmul a)\rmul s) and g = ((x\rmul a)\rmul s) and  h = -((y\rmul b)\rmul s) and  j = ((x\rmul b)\rmul r) and k = ((y\rmul a)\rmul r).
Then (u\cmul v)\cmul w =  ( (c \radd  d) \radd  (e  \radd  f) , (g \radd  h) \radd   (j \radd  k)  )
      		   =  ( (c \radd  e) \radd  (d \radd  f) , (g \radd   j) \radd  (h  \radd  k) )
      		   = ( (c \radd  e) \radd  (f \radd  d) , (g \radd  j) \radd  (k \radd  h) ).
Then v \cmul  w = ((a\rmul r) - (b\rmul s),(a\rmul s) \radd  (b\rmul r)) (by CMult).
u\cmul (v\cmul w) = u \cmul  ((a,b) \cmul  (r,s)) 
		  = u \cmul  (  ((a\rmul r) - (b\rmul s) , (a\rmul s) \radd  (b\rmul r))  )
		  = (  ( x \rmul  ((a\rmul r) -(b\rmul s)) ) - (y\rmul ((a\rmul s) \radd  (b\rmul r))) , (x\rmul ((a\rmul s) \radd  (b\rmul r))) \radd  (y\rmul ((a\rmul r) - (b\rmul s))))
		  = ((x\rmul ((a\rmul r) \radd  -(b\rmul s))) \radd  -(y\rmul ((a\rmul s) \radd  (b\rmul r))) , (x\rmul ((a\rmul s) 	\radd  (b\rmul r))) \radd  (y\rmul ((a\rmul r) \radd  -(b\rmul s))))
		  = (  ( (x\rmul (a\rmul r)) \radd  (x\rmul (-(b\rmul s))) ) \radd  -( (y\rmul (a\rmul s)) \radd  (y\rmul (b\rmul r)) )  , ( (x\rmul (a\rmul s)) \radd  (x\rmul (b\rmul r)) ) \radd  ( (y\rmul (a\rmul r)) \radd  (y\rmul (-(b\rmul s))) ) )
		  = (  ( (x\rmul (a\rmul r)) \radd  -(x\rmul (b\rmul s)) ) \radd  ( -(y\rmul (a\rmul s)) \radd  -(y\rmul (b\rmul r)) )  ,  ( (x\rmul (a\rmul s)) \radd  (x\rmul (b\rmul r)) ) \radd  ( (y\rmul (a\rmul r)) \radd  -(y\rmul (b\rmul s)) ) )
		  = (  ( ((x\rmul a)\rmul r) \radd  -((x\rmul b)\rmul s) ) \radd  ( -((y\rmul a)\rmul s) \radd  -((y\rmul b)\rmul r) )  ,  ( ((x\rmul a)\rmul s) \radd  ((x\rmul b)\rmul r) ) \radd  ( ((y\rmul a)\rmul r) \radd  -((y\rmul b)\rmul s) ) )
		  = ( (c \radd  e) \radd  (f \radd  d) , (g \radd  j) \radd  (k \radd  h) ) = (u\cmul v)\cmul w.
Then (u\cmul v)\cmul w = u\cmul (v\cmul w). qed.


Proposition CmpFld8. u\cmul (1,0) = u.
Proof. Let x,y be real numbers such that u = (x,y). Then u \cmul  (1,0) = ((x,y)) \cmul  (1,0) = ((x\rmul 1)-(y\rmul 0) , (x\rmul 0)\radd (y\rmul 1)). 
	Let a = x\rmul 1 and b = y\rmul 0 and c = x\rmul 0 and d = y\rmul 1.
		
	((x\rmul 1)-(y\rmul 0) , (x\rmul 0)\radd (y\rmul 1)) = (a-b, c\radd d).
	Then a = x and b = 0 and c = 0 and d = y.
 	(a-b, c\radd d) = (x-0 , 0\radd y) = (x,y) = u .
Then u\cmul (1,0) = u.
qed.




###Axiome fur ordSet

\begin{signature} x < y is an atom.
\end{signature}
Let x > y stand for y < x.
Let x <= y stand for x<y or x=y.
Let x >= y stand for y <= x.

\begin{axiom}[InEqCompl] Then x<y or y<x or x = y.
\end{axiom}
\begin{axiom} . If x<y then not x=y. #and if x<y then not y<x.
\end{axiom}
\begin{axiom}  Trans. If x<y and y<z then x<z.
\end{axiom}

#################################################
###Axiome für ordField
\begin{axiom}[InEqAdd] If y<z then x\radd y<x\radd z.
\end{axiom}
\begin{axiom}[InEqMult] If x>0 and y>0 then x\rmul y>0.
\end{axiom}

#Bewiesene Aussagen
\begin{axiom}[P118d] If not x = 0 then x\rmul x > 0.
\end{axiom}
\begin{axiom}[P118e1] If 0<y then 0 < inv(y).
\end{axiom}

\begin{axiom}[SqrtEind] If x\rmul x = y\rmul y and x>=0 and y >= 0 then x=y.
\end{axiom}

Proposition SqrtMon. If x\rmul x > y\rmul y and x > 0  and y >= 0 then x>y.
Proof.	Assume x\rmul x > y\rmul y and x>0 and y >= 0.
		Thus  -(y\rmul y) \radd  (x\rmul x) > -(y\rmul y) \radd  (y\rmul y) (by InEqAdd). 
		Hence (x\rmul x) - (y\rmul y) > 0.
		Then  (x \radd  y) \rmul  (x - y) > 0.
		x \radd  y > 0.
		proof. 	x > 0. Then x \radd  y > y.
				case y > 0. Then x \radd  y > 0 (by Trans). end.
				case y = 0. Then x \radd  y  > 0 . end.
		end.
		Thus inv(x \radd  y) > 0. Hence inv(x \radd  y) \rmul  ((x \radd  y) \rmul  (x - y)) > 0.
		We have inv(x \radd  y) \rmul  ((x \radd  y) \rmul  (x - y)) 
			 = (inv(x \radd  y) \rmul  (x \radd  y)) \rmul  (x - y)
			 = 1 \rmul  (x - y) = x - y.
		Then x - y > 0.
		Thus (x \radd  -y) \radd  y > 0 \radd  y.
		Hence x \radd  (-y \radd  y) > 0 \radd  y.
qed.

Lemma LeqTrans. If x<=y and y<=z then x<=z.
Proof. 	Obvious.


Lemma LeqAdd. 	If y <= z then x \radd  y <= x\radd z.
Proof.			Case y < z. Then x \radd  y <= x \radd  z (by InEqAdd). end.
				Case y = z. Then x\radd y = x\radd z. end.
qed.

Lemma LeqAdd2.  If a <= b and c <= d then a\radd c <= b\radd d.
Proof.			Assume a <= b and c <= d. 
				Case a=b. Then a\radd c = b\radd c. b\radd c <= b\radd d (by LeqAdd). Thus a\radd c <= b\radd d. end.
				Case a < b. Then c\radd a < c\radd b. Thus c\radd a <= c\radd b. Hence a\radd c <= b\radd c. b\radd c <= b\radd d (by LeqAdd).
							Then a\radd c <= b\radd d (by LeqTrans). end.
qed.
############################################

Lemma 21O. If (not x = 0) or (not y = 0) then (x\rmul x) \radd  (y\rmul y)>0.
Proof. 
	Assume (not x=0) or (not y=0).
	Let a = x\rmul x and b = y\rmul y.
	Then a> 0 or b>0.
	Then a>= 0 and b>=0.
	Let us show that a \radd  b > 0.
		Case a>0. Then a\radd b>=a\radd 0>=a>0. end.
		Case b>0. Then a\radd b>=0\radd b>=b>0. end.
	end.
qed.

Proposition CmpFld9. If not u = (0,0) then there exists a complex number v such that u\cmul v = (1,0).
Proof. Assume not u = (0,0).
	   Let x,y be real numbers such that u = (x,y).
	   #Assume x = 0 and y=0. Then u = (0,0). Contradiction. 
		Then (not x=0) or (not y=0).
		Then (x\rmul x) \radd  (y\rmul y)>0 (by 21O).
	   Then not (x\rmul x) \radd  (y\rmul y) = 0 .

	   Let d = inv((x\rmul x) \radd  (y\rmul y)) and 
	   	   v = (x\rmul d , -y\rmul d). 
	   Then u\cmul v = ( (x\rmul (x\rmul d)) - (y\rmul (-y\rmul d)) , (x\rmul (-y\rmul d)) \radd  (y\rmul (x\rmul d)))
	   			= ( (x\rmul (x\rmul d)) \radd  -(-y\rmul (y\rmul d)) , -(x\rmul (y\rmul d)) \radd  (y\rmul (x\rmul d)))
	   			= ( (x\rmul (x\rmul d)) \radd  (y\rmul (y\rmul d)) , -(x\rmul (y\rmul d)) \radd  (y\rmul (x\rmul d)))
	   			= ( ((x\rmul x)\rmul d) \radd  ((y\rmul y)\rmul d) , -((x\rmul y)\rmul d) \radd  ((y\rmul x)\rmul d))
	   			= ( ((x\rmul x) \radd  (y\rmul y))\rmul d , -((x\rmul y)\rmul d) \radd  ((x\rmul y)\rmul d))
	   			= ( ((x\rmul x) \radd  (y\rmul y))\rmul inv((x\rmul x) \radd  (y\rmul y)) , 0)
	   			= (1,0).
		Then u\cmul v = (1,0).
		Then v is a complex number such that u\cmul v = (1,0).
		If not u = (0,0) then there exists a complex number w such that u\cmul w = (1,0).
	   qed.




Proposition CmpFld10. u\cmul (v\cadd w) = (u\cmul v) \cadd  (u\cmul w).
Proof.	Let x,y,a,b,r,s be real numbers such that (u = (x,y) and v = (a,b) and w = (r,s)).
		Let c = x\rmul a and d = x\rmul r and e = -(y\rmul b) and f = -(y\rmul s) and g = x\rmul b and  h = x\rmul s and  j = y\rmul a and k = y\rmul r.
		v\cadd w = (a\radd r,b\radd s) (by CAdd).
		u\cmul v = (x,y) \cmul  (a,b) = ((x\rmul a) - (y\rmul b), (x\rmul b) \radd  (y\rmul a)) (by CMult).  #1
		u\cmul w = (x,y) \cmul  (r,s) = ((x\rmul r) - (y\rmul s) , (x\rmul s) \radd  (y\rmul r)) (by CMult). #2
		Then u\cmul (v\cadd w) 	= (x,y) \cmul  (a\radd r,b\radd s) = ( (x\rmul (a\radd r)) - (y\rmul (b\radd s)) , (x\rmul (b\radd s)) \radd  (y\rmul (a\radd r)) ).
						( (x\rmul (a\radd r)) - (y\rmul (b\radd s)) , (x\rmul (b\radd s)) \radd  (y\rmul (a\radd r)) )
						.= ( ((x\rmul a) \radd  (x\rmul r)) - (y\rmul (b\radd s)) , (x\rmul (b\radd s)) \radd  (y\rmul (a\radd r)) )
						.= ( ((x\rmul a) \radd  (x\rmul r)) - ((y\rmul b) \radd  (y\rmul s)) , (x\rmul (b\radd s)) \radd  (y\rmul (a\radd r)) ).

						( ((x\rmul a) \radd  (x\rmul r)) - ((y\rmul b) \radd  (y\rmul s)) , (x\rmul (b\radd s)) \radd  (y\rmul (a\radd r)) )
						= ( ((x\rmul a) \radd  (x\rmul r)) - ((y\rmul b) \radd  (y\rmul s)) , ((x\rmul b) \radd  (x\rmul s)) \radd  (y\rmul (a\radd r)) ) (by Dis1).
						
						( ((x\rmul a) \radd  (x\rmul r)) - ((y\rmul b) \radd  (y\rmul s)) , ((x\rmul b) \radd  (x\rmul s)) \radd  (y\rmul (a\radd r)) )
						= ( ((x\rmul a) \radd  (x\rmul r)) - ((y\rmul b) \radd  (y\rmul s)) , ((x\rmul b) \radd  (x\rmul s)) \radd  ((y\rmul a) \radd  (y\rmul r)) ) (by Dis1).

						( ((x\rmul a) \radd  (x\rmul r)) - ((y\rmul b) \radd  (y\rmul s)) , ((x\rmul b) \radd  (x\rmul s)) \radd  ((y\rmul a) \radd  (y\rmul r)) )
						= (((x\rmul a) \radd  (x\rmul r)) \radd  -((y\rmul b) \radd  (y\rmul s)) , ((x\rmul b) \radd  (x\rmul s)) \radd  ((y\rmul a) \radd  (y\rmul r)) )
						= (((x\rmul a) \radd  (x\rmul r)) \radd  (-(y\rmul b) \radd  -(y\rmul s)) , ((x\rmul b) \radd  (x\rmul s)) \radd  ((y\rmul a) \radd  (y\rmul r)) )
						= ((c \radd  d) \radd  (e \radd  f) , (g \radd  h) \radd  (j \radd  k))
						= ((c \radd  e) \radd  (d \radd  f) , (g \radd  j) \radd  (h \radd  k))
						= (c\radd e,g\radd j) \cadd  (d\radd f,h\radd k)
						= ((x\rmul a) \radd  -(y\rmul b), (x\rmul b) \radd  (y\rmul a)) \cadd  ((x\rmul r) \radd  -(y\rmul s) , (x\rmul s) \radd  (y\rmul r))
						= ((x\rmul a) - (y\rmul b), (x\rmul b) \radd  (y\rmul a)) \cadd  ((x\rmul r) - (y\rmul s) , (x\rmul s) \radd  (y\rmul r))
						= ((x,y) \cmul  (a,b)) \cadd  ((x,y) \cmul  (r,s)) ##wegen #1 und #2
						= (u\cmul v) \cadd  (u\cmul w).
		Then u\cmul (v\cadd w) = (u\cmul v) \cadd  (u\cmul w).
qed.

Proposition ZeroMult. u\cmul (0,0) = (0,0).
Proof. Let x,y be real numbers such that u = (x,y). Then u\cmul (0,0) = ((x\rmul 0) - (y\rmul 0), (x\rmul 0) \radd  (y\rmul 0)) (by CMult). 
Let a=x\rmul 0 and b=y\rmul 0.
((x\rmul 0) - (y\rmul 0), (x\rmul 0) \radd  (y\rmul 0)) = (a-b,a\radd b) = (0 - 0, 0 \radd  0) = (0,0). 
Then u\cmul (0,0) = (0,0). 
qed.

Proposition. (a,0) \cadd  (b,0) = (a\radd b,0) and (a,0)\cmul (b,0) = (a\rmul b,0).
Proof. 	(a,0) \cadd  (b,0) = (a\radd b,0\radd 0) = (a\radd b,0) (by CAdd).
		(a,0) \cmul  (b,0) = ((a\rmul b) - (0\rmul 0), (a\rmul 0) \radd  (0\rmul b)) (by CMult).
		Let c = a\rmul 0 and d = b\rmul 0 and e = 0\rmul 0.
		Then c = 0 and d = 0 and e = 0.
		Then ((a\rmul b) - e, c \radd  d) = ((a\rmul b) -0, 0 \radd  0) = ((a\rmul b )\radd  -0,0) = (a\rmul b,0) . qed.

\begin{signature} ! is a complex number such that ! = (0,1).
\end{signature}

Proposition. !\cmul ! = (-1,0).
Proof. 	!\cmul ! .= (0,1) \cmul  (0,1) 
		.= ((0\rmul 0) - (1\rmul 1),( 0\rmul 1) \radd  (1\rmul 0)) (by CMult)
		.=((0\rmul 0) - (1\rmul 1),( 0\rmul 1) \radd  (1\rmul 0)) 
		.= (0 - 1, 0\radd 0) 
		.= (0\radd (-1),0).
		Then (0\radd (-1),0) = ((-1),0). 
qed.

Let x\radd y! stand for (x,0) \cadd  ((y,0)\cmul !).


Proposition. a\radd b! = (a,b).
Proof. 	(b,0)\cmul (0,1) = ((b\rmul 0)-(0\rmul 1) , (b\rmul 1)\radd (0\rmul 0)) (by CMult).
		a\radd b! = (a,0) \cadd  ((b,0)\cmul (0,1)). 
		
		(a,0) \cadd  ((b,0)\cmul (0,1))
		= (a,0) \cadd  ((b\rmul 0)-(0\rmul 1) , (b\rmul 1)\radd (0\rmul 0)) (by CMult).

		(a,0) \cadd  ((b\rmul 0)-(0\rmul 1) , (b\rmul 1)\radd (0\rmul 0))
		= (a,0) \cadd  (0-0,b\radd 0) 
		= (a,0) \cadd  (0,b).
		
		(a,0) \cadd  (0,b)
		= (a\radd 0,0\radd b) 
		= (a,b).
qed.


\begin{signature} Conj(u) is a complex number.
\end{signature}
\begin{signature} Re(u) is a real number.
\end{signature}
\begin{signature} Im(u) is a real number.
\end{signature}
\begin{axiom} Conj((x,y)) = (x,-y).
\end{axiom}
\begin{axiom} Re((x,y)) = x.
\end{axiom}
\begin{axiom} Im((x,y)) = y.
\end{axiom}

Proposition Conj1. Conj(u\cadd v) = Conj(u) \cadd  Conj(v).
Proof. 	Let x,y,a,b be real numbers such that (u = (x,y) and v = (a,b)).
		Then u\cadd v = (x\radd a,y\radd b) (by CAdd).
		Then Conj(u \cadd  v) = Conj((x\radd a,y\radd b)) = (x\radd a,-(y\radd b)) = (x\radd a, -y \radd  -b) = (x,-y) \cadd  (a,-b) = Conj(u) \cadd  Conj(v).
		Then Conj(u\cadd v) = Conj(u) \cadd  Conj(v).
qed.		
Proposition Conj2. Conj(u\cmul v) = Conj(u) \cmul  Conj(v).
Proof. 	Let x,y,a,b be real numbers such that (u = (x,y) and v = (a,b)).
		Then u \cmul  v = ((x\rmul a)-(y\rmul b),(x\rmul b)\radd (y\rmul a)) (by CMult).
		Then Conj(u \cmul  v) 	= Conj(((x\rmul a)-(y\rmul b),(x\rmul b)\radd (y\rmul a))) 
							= ((x\rmul a) - (y\rmul b)       ,-((x\rmul b)\radd (y\rmul a))) 
							= ((x\rmul a) - (y\rmul b)       , -(x\rmul b)\radd  -(y\rmul a))
							= ((x\rmul a) - (-(-(y\rmul b))) , -(x\rmul b)\radd  -(y\rmul a))
							= ((x\rmul a) - ( -((-y)\rmul b) ) , (x\rmul (-b))\radd  ((-y)\rmul a))
							= ((x\rmul a) - ((-y)\rmul (-b)) , (x\rmul (-b))\radd  ((-y)\rmul a))
							= (x,-y) \cmul  (a,-b)
							= Conj(u) \cmul  Conj(v). 
		Then Conj(u\cmul v) = Conj(u) \cmul  Conj(v).
qed.

Proposition Conj3. u \cadd  Conj(u) = (Re(u)\radd Re(u),0) and u \cadd  ((-1,0)\cmul Conj(u)) = (0,Im(u)\radd Im(u)).
Proof. Let x,y be real numbers such that u=(x,y).
	   Then u \cadd  Conj(u) = (x,y) \cadd  Conj((x,y)) = (x,y) \cadd  (x,-y) = (x\radd x,y-y) = (x\radd x,0) = (Re(u)\radd Re(u),0).
	   (-1,0)\cmul (x,-y) = ( ((-1)\rmul x) - (0\rmul (-y)) , ((-1)\rmul (-y)) \radd  (0\rmul x)) (by CMult).
	   u \cadd  ((-1,0)\cmul Conj(u))	= (x,y) \cadd  ((-1,0)\cmul Conj((x,y))) = (x,y) \cadd  ((-1,0)\cmul (x,-y)) 
	   							= (x,y) \cadd  ( ((-1)\rmul x) - (0\rmul (-y)) , ((-1)\rmul (-y)) \radd  (0\rmul x))
	   							= (x,y) \cadd  (-x - 0, -(-y) \radd  0)
	   							= (x,y) \cadd  (-x , y) = (x-x,y\radd y) = (0,Im(u)\radd Im(u)).
	   	Then u \cadd  Conj(u) = (Re(u)\radd Re(u),0) and u \cadd  ((-1,0)\cmul Conj(u)) = (0,Im(u)\radd Im(u)).
qed.	   

Proposition Conj4. If not u = (0,0) then there exists a real number z such that (z>0 and u \cmul  Conj(u) = (z,0)).
Proof.	Let not u = (0,0).
	   	Take real numbers x,y such that u = (x,y).
	   	#Assume x = 0 and y=0. Then u = (0,0). Contradiction. Hence not x=0 or not y=0.
		Then (x\rmul x)> 0 or (y\rmul y)>0. Take z = (x\rmul x) \radd  (y\rmul y). 
		Then z > 0 (by 21O).
		u\cmul Conj(u) = (x,y)\cmul Conj((x,y)) = (x,y)\cmul (x,-y) = ((x\rmul x) - (y\rmul (-y)) , (x\rmul (-y)) \radd  (y\rmul x)) = ((x\rmul x) \radd  -(-(y\rmul y)) , -(x\rmul y) \radd  (x\rmul y)) = ((x\rmul x) \radd  (y\rmul y), 0) = (z,0).
		z is a real number.
		Then z is a real number and (z>0 and u\cmul Conj(u) = (z,0)).
qed.

Proposition Conj5. Conj(Conj(u)) = u.
Proof. 	Let x,y be real numbers such that u = (x,y).
		Then Conj(Conj(u)) = Conj(Conj((x,y))) = Conj((x,-y)) = (x,--y) = (x,y) = u. qed.


\begin{signature} |u| is a real number such that (|u|\rmul |u|,0) = u\cmul Conj(u) and |u| >= 0. 
\end{signature}
\begin{axiom}[EindAbs] If (x\rmul x,0) = u\cmul Conj(u) and x >= 0 then x = |u|.
\end{axiom}



Proposition Abs1. If not u = (0,0) then |u| > 0.
Proof.  Let z be a real number such that (z > 0 and u\cmul Conj(u) = (z,0)). Then (|u|\rmul |u|,0) = (z,0).
		Assume |u| = 0. Then z = 0\rmul 0 = 0. Contradiction. 
		Then |u| > 0.
qed.

Proposition Abs2. |(0,0)| = 0.
Proof. 	(0,0) \cmul  Conj((0,0)) = Conj((0,0)) \cmul  (0,0) = (0,0). 
		Let us show that |(0,0)|\rmul |(0,0)| = 0.
		Assume |(0,0)| > 0. Then not |(0,0)| = 0 .Then |(0,0)|\rmul |(0,0)| > 0. Then not |(0,0)|\rmul |(0,0)| = 0. Contradiction.end. 
		Hence |(0,0)| = 0.
qed.

Proposition Abs3. |u| = |Conj(u)|.
Proof. 	|u| is a real number such that ((|u|\rmul |u|,0) = u\cmul Conj(u) and |u| >= 0). 
		(|u|\rmul |u|,0) = Conj(Conj(u))\cmul Conj(u) = Conj(u)\cmul Conj(Conj(u)).
		Let v = Conj(u). Then |u| is real number such that ((|u|\rmul |u|,0) = v\cmul Conj(v) and |u| >= 0). 
		Hence |u| = |v|.
qed.

Proposition Abs4. |u\cmul v| = |u| \rmul  |v|.
Proof. 	Case u = (0,0) or v = (0,0). Then v\cmul u = (0,0) or u\cmul v = (0,0) (by ZeroMult). Hence |u\cmul v| = 0.
									|u| = 0 or |v| = 0 (by Abs2). Hence |u| \rmul  |v| = 0.
									Then  |u\cmul v| = |u|\rmul |v|. end.
		Case (not u=(0,0)) and (not v = (0,0)). Then |u| > 0 and |v| > 0 (by Abs1). Let z = |u| \rmul  |v|. Then z > 0.
												z\rmul z = (|u|\rmul |v|) \rmul  (|u|\rmul |v|) = ((|u|\rmul |v|)\rmul |u|)\rmul |v| = (|u|\rmul (|v|\rmul |u|))\rmul |v|
													= (|u|\rmul (|u|\rmul |v|))\rmul |v| = ((|u|\rmul |u|)\rmul |v|)\rmul |v| = (|u|\rmul |u|)\rmul (|v|\rmul |v|).
												Hence (z\rmul z,0)	= ((|u|\rmul |u|)\rmul (|v|\rmul |v|), 0) = (|u|\rmul |u|,0) \cmul  (|v|\rmul |v|,0)
																= (u  \cmul   Conj(u)) \cmul  (v   \cmul  Conj(v)) 
																= ((u \cmul   Conj(u)) \cmul   v)  \cmul  Conj(v) 
									       	 	   				= ( u \cmul  (Conj(u)  \cmul   v)) \cmul  Conj(v) 
									       	    				= ( u \cmul  (v  \cmul   Conj(u))) \cmul  Conj(v) 
									      						= ((u \cmul   v) \cmul   Conj(u))  \cmul  Conj(v) 
									      						= ( u \cmul   v) \cmul   (Conj(u)   \cmul  Conj(v)) = (u\cmul v)\cmul Conj(u\cmul v).
									      		Then z = |u\cmul v|. Then |u| \rmul  |v| = |u\cmul v|. end.
qed.

Proposition Abs5. |(Re(u),0)| <= |u|.
Proof. 	Let x,y be real numbers such that u = (x,y). 
		Case y = 0. Then |u| = |(x,y)| = |(Re(u),0)|. end.
		Case not y = 0. Then not u = (0,0). Thus |u| > 0.
 						(|u|\rmul |u|,0) = u\cmul Conj(u) = (x,y) \cmul  (x,-y) = ((x\rmul x) - (y\rmul (-y)) , (x\rmul (-y)) \radd  (y\rmul x)) = ((x\rmul x) \radd  -(-(y\rmul y)) , -(x\rmul y) \radd  (x\rmul y)) 
									= ((x\rmul x) \radd  (y\rmul y), 0). Then |u|\rmul |u| = (x\rmul x)\radd (y\rmul y).
						(|(Re(u),0)|\rmul |(Re(u),0)|,0) = (Re(u),0)\cmul Conj((Re(u),0)) = (Re(u),0)\cmul (Re(u),-0) = (Re(u),0)\cmul (Re(u),0) = (x,0)\cmul (x,0) = (x\rmul x,0).
						Then |(Re(u),0)|\rmul |(Re(u),0)| = x\rmul x.
						Hence (|u|\rmul |u|) -  (|(Re(u),0)|\rmul |(Re(u),0)|) = ((x\rmul x) \radd  (y\rmul y)) - (x\rmul x) = ((y\rmul y) \radd  (x\rmul x)) \radd  -(x\rmul x) = (y\rmul y) \radd  ((x\rmul x) \radd  -(x\rmul x)) = (y\rmul y) \radd  0 = y\rmul y and y\rmul y > 0.
						Thus (|u|\rmul |u|) -  (|(Re(u),0)|\rmul |(Re(u),0)|) > 0.
						Hence |u|\rmul |u| > |(Re(u),0)|\rmul |(Re(u),0)| and |u| > 0 and |(Re(u),0)| >= 0.
						Thus  |u| > |(Re(u),0)| (by SqrtMon).
		end.
qed.

Lemma AbsMon. Re(u) <= |(Re(u),0)|.
Proof.
	Let x,y be real numbers such that u = (x,y).
	Let v = (x, 0).
	Then Re(u) = x = Re(v).
	Case x>0.
		
		Then (x\rmul x,0) = v\cmul Conj(v).
		Then x = |v| (by EindAbs).
		Then Re(u)=Re(v)=|v|=|(Re(v),0)| = |(Re(u),0)|.
		Then Re(u) <= |(Re(u),0)|.
	end.
	Case x <= 0.
		Then |(Re(u),0)| >=0.
		x <= 0 <= |(Re(u),0)|.
	end.
qed.

Proposition Abs6. |u \cadd  v| <= |u| \radd  |v|.
Proof. (|u \cadd  v|\rmul |u \cadd  v|,0)	.= (u \cadd  v) \cmul  Conj(u \cadd  v)
								.=  ((u \cadd  v) \cmul  Conj(u)) \cadd  ((u \cadd  v) \cmul  Conj(v)) 
								.=  (Conj(u) \cmul  (u \cadd  v)) \cadd  (Conj(v) \cmul  (u \cadd  v)) 
								.=  ((Conj(u) \cmul  u) \cadd  (Conj(u) \cmul  v)) \cadd  ((Conj(v) \cmul  u) \cadd  (Conj(v) \cmul  v)) (by CmpFld10)
								.=  ((u \cmul  Conj(u)) \cadd  (v \cmul  Conj(u))) \cadd  ((u \cmul  Conj(v)) \cadd  (v \cmul  Conj(v)))
								.=  (u \cmul  Conj(u)) \cadd  ((v \cmul  Conj(u)) \cadd  ((u \cmul  Conj(v)) \cadd  (v \cmul  Conj(v))))
								.=  (u \cmul  Conj(u)) \cadd  (((v \cmul  Conj(u)) \cadd  (u \cmul  Conj(v))) \cadd  (v \cmul  Conj(v)))
								.=  (u \cmul  Conj(u)) \cadd  ((v \cmul  Conj(v)) \cadd  ((v \cmul  Conj(u)) \cadd  (u \cmul  Conj(v)))) 
								.=  (u \cmul  Conj(u)) \cadd  ((v \cmul  Conj(v)) \cadd  ((v \cmul  Conj(u)) \cadd  (Conj(Conj(u)) \cmul  Conj(v))))
								.=  (u \cmul  Conj(u)) \cadd  ((v \cmul  Conj(v)) \cadd  ((v \cmul  Conj(u)) \cadd  Conj(Conj(u) \cmul  v) ))
								.=  (u \cmul  Conj(u)) \cadd  ((v \cmul  Conj(v)) \cadd  ((v \cmul  Conj(u)) \cadd  Conj(v \cmul  Conj(u)) ))
								.=  (u \cmul  Conj(u)) \cadd  ((v \cmul  Conj(v)) \cadd  (Re(v \cmul  Conj(u))\radd Re(v \cmul  Conj(u)),0) )
								.=  (|u|\rmul |u|,0) \cadd  ((|v|\rmul |v|,0) \cadd  (Re(v \cmul  Conj(u))\radd Re(v \cmul  Conj(u)),0) )
								.=  ((|u|\rmul |u|,0) \cadd  (|v|\rmul |v|,0)) \cadd  (Re(v \cmul  Conj(u))\radd Re(v \cmul  Conj(u)),0)
								.= ((|u|\rmul |u|) \radd  (|v|\rmul |v|),0\radd 0) \cadd  (Re(v \cmul  Conj(u))\radd Re(v \cmul  Conj(u)),0)
								.=(((|u|\rmul |u|) \radd  (|v|\rmul |v|)) \radd  (Re(v \cmul  Conj(u))\radd Re(v \cmul  Conj(u))), 0 \radd  0)
								.=(((|u|\rmul |u|) \radd  (|v|\rmul |v|)) \radd  (Re(v \cmul  Conj(u))\radd Re(v \cmul  Conj(u))), 0 ).
								Thus |u \cadd  v|\rmul |u \cadd  v| = (((|u|\rmul |u|) \radd  (|v|\rmul |v|)) \radd  Re(v \cmul  Conj(u))) \radd  Re(v \cmul  Conj(u)).

								Then Re(v \cmul  Conj(u)) <= |(Re(v \cmul  Conj(u)),0)| (by AbsMon). |(Re(v \cmul  Conj(u)),0)| <= |v\cmul Conj(u)| (by Abs5).
								Hence Re(v \cmul  Conj(u)) <= |v\cmul Conj(u)| (by LeqTrans).
								Thus ((|u|\rmul |u|) \radd  (|v|\rmul |v|)) \radd  Re(v \cmul  Conj(u)) <= ((|u|\rmul |u|) \radd  (|v|\rmul |v|)) \radd  |v\cmul Conj(u)| (by LeqAdd).
								Thus (((|u|\rmul |u|) \radd  (|v|\rmul |v|)) \radd  Re(v \cmul  Conj(u))) \radd  Re(v \cmul  Conj(u)) <= (((|u|\rmul |u|) \radd  (|v|\rmul |v|)) \radd  |v\cmul Conj(u)|)\radd |v\cmul Conj(u)| (by LeqAdd2).
								(((|u|\rmul |u|) \radd  (|v|\rmul |v|)) \radd  |v\cmul Conj(u)|)\radd |v\cmul Conj(u)|
								= ((|u|\rmul |u|) \radd  (|v|\rmul |v|)) \radd  (|v\cmul Conj(u)|\radd |v\cmul Conj(u)|) 
								= ((|u|\rmul |u|) \radd  (|v|\rmul |v|)) \radd  ((|v|\rmul |Conj(u)|) \radd  (|v|\rmul |Conj(u)|))
								= ((|u|\rmul |u|) \radd  (|v|\rmul |v|)) \radd  ((|v|\rmul |u|) \radd  (|v|\rmul |u|))
								= ((|u|\rmul |u|) \radd  (|v|\rmul |u|)) \radd  ((|v|\rmul |v|) \radd  (|v|\rmul |u|))
								= ((|u| \radd  |v|) \rmul  |u|) \radd  (|v|\rmul (|v|\radd |u|))
								= (|u| \rmul  (|u| \radd  |v|)) \radd  (|v| \rmul  (|u| \radd  |v|))
								= (|u| \radd  |v|)\rmul (|u| \radd  |v|).

								Hence |u \cadd  v|\rmul |u \cadd  v| <= (|u| \radd  |v|)\rmul (|u| \radd  |v|).
								0 <= |u| and 0 <= |v|. Thus 0 \radd  0 <= |u| \radd  |v| (by LeqAdd2).
								Case |u| \radd  |v| > 0. 
									Case |u \cadd  v|\rmul |u \cadd  v| < (|u| \radd  |v|)\rmul (|u| \radd  |v|). 
										Then (|u| \radd  |v|)\rmul (|u| \radd  |v|) > |u \cadd  v|\rmul |u \cadd  v| and |u \cadd  v| >= 0 and |u| \radd  |v| > 0 .
										Thus |u \cadd  v| < |u| \radd  |v| (by SqrtMon). end.
									Case |u \cadd  v|\rmul |u \cadd  v| = (|u| \radd  |v|)\rmul (|u| \radd  |v|). Then |u \cadd  v| = |u| \radd  |v| (by SqrtEind).end.
								end.
								Case |u| \radd  |v| = 0. Thus (|u| \radd  |v|)\rmul (|u| \radd  |v|) = 0. Hence |u \cadd  v|\rmul |u \cadd  v| <= 0. 
													Let us show that |u \cadd  v| <= 0. 
													proof. 	Assume |u \cadd  v| > 0. Then |u \cadd  v|\rmul |u \cadd  v| > 0 (by InEqMult). Contradiction.
															Thus not (|u \cadd  v| > 0). Hence |u \cadd  v| < 0 or |u \cadd  v| = 0 (by InEqCompl).
													end.
													Thus |u \cadd  v| <= |u| \radd  |v|.
								end.
qed.



\end{document}