\documentclass{article}
\usepackage[english]{babel}
\usepackage{enumerate, latexsym, amssymb, amsmath}
\usepackage{framed, multicol}
\newenvironment{forthel}{\begin{leftbar}}{\end{leftbar}}

\newcommand{\tmaffiliation}[1]{\\ #1}
\newcommand{\tmem}[1]{{\em #1\/}}
\newenvironment{enumeratenumeric}{\begin{enumerate}[1.] }{\end{enumerate}}
\newenvironment{proof}{\noindent\textbf{Proof\ }}{\hspace*{\fill}$\Box$\medskip}
\newenvironment{quoteenv}{\begin{quote} }{\end{quote}}
\newtheorem{axiom}{Axiom}
\newtheorem{lemma}{Lemma}
\newtheorem{theorem}{Theorem}
\newtheorem{definition}{Definition}
\newtheorem{signature}{Signature}
\newtheorem{proposition}{Proposition}

\newcommand{\event}{UITP 2018}
\newcommand{\dom}{Dom}
\newcommand{\fun}{aFunction}
\newcommand{\sym}{sym}
\newcommand{\halfline}{{\vspace{3pt}}}
\newcommand{\tab}{{\hspace{1cm}}}
\newcommand{\ball}[2]{B_{#1}(#2)}
\newcommand{\llbracket}{[}
\newcommand{\rrbracket}{]}
\newcommand{\less}[1]{<_{#1}}
\newcommand{\greater}[1]{>_{#1}}
\newcommand{\leeq}[1]{{\leq}_{#1}}
\newcommand{\supr}[1]{\mathrm{sup}_{#1}}
\newcommand{\RR}{\mathbb{R}}
\newcommand{\QQ}{\mathbb{Q}}
\newcommand{\ZZ}{\mathbb{Z}}
\newcommand{\NN}{\mathbb{N}}
\newcommand{\CC}{\mathbb{C}}
\newcommand{\cmul}{\cdot}
\newcommand{\cadd}{+}
\newcommand{\di }{i}

\begin{document}

\title{The Complex Field}

\maketitle

Zunächst definieren wir die reellen Zahlen und übertragen einige Propositionen aus den vorherigen Kapiteln, die für die weiteren Beweise wichtig sein werden.

\begin{forthel}

[set/-s] [element/-s] [number/-s]
\begin{signature}. A real number is a notion.

\end{signature}


Let $x,y,z,a,b,c,d,e,f,g,h,j,k,r,s$ denote real numbers.



\begin{signature}[Add] $x + y$ is a real number.

\end{signature}
\begin{axiom}[RComm]$x+y = y+x$.

\end{axiom}
\begin{axiom} $(x+y)+z = x+(y+z)$.

\end{axiom}
\begin{signature}[Zero] $0$ is a real number such that for every real number $x$ $x + 0 = x$.

\end{signature}
\begin{signature} $-x$ is a real number such that $-x + x = 0$.

\end{signature}

Let $x - y$ stand for $x + (-y)$.

\begin{signature}[Mult]$x \cdot y$ is real number.

\end{signature}
\begin{axiom} $x\cdot y = y\cdot x$.

\end{axiom}
\begin{axiom} $(x\cdot y)\cdot z = x\cdot (y\cdot z)$.

\end{axiom}
\begin{signature}[One] $1$ is a real number such that for every real number $x$ $x\cdot 1 = x$.

\end{signature}
\begin{signature} Assume not $x = 0$. $inv(x)$ is a real number such that $inv(x)\cdot x = 1$ .

\end{signature}
\begin{axiom} $1$ is not equal to $0$.

\end{axiom}

\begin{axiom}[Dis1] $x\cdot(y+z)=(x\cdot y)+(x\cdot z)$.

\end{axiom}
\begin{axiom}[Dis2] $(y+z) \cdot x=(y \cdot x)+(z \cdot x)$.

\end{axiom}


\begin{axiom}[P114c] If $x+y = 0$ then $y = -x$.

\end{axiom}
\begin{axiom}[P114d] $-(-x) = x$.

\end{axiom}
\begin{axiom}[P115c] If (not $x=0$) and $x\cdot y = 1$ then $y = inv(x)$.  

\end{axiom}
\begin{axiom}[P115d] If not $x=0$ then $inv(inv(x)) = x$.

\end{axiom}
\begin{axiom}[P116c] $(-x)\cdot y = -(x\cdot y) = x \cdot (-y)$.

\end{axiom}

\begin{axiom} $-(x+y) = -x + -y$. 

\end{axiom}
\begin{axiom} $x \cdot 0 = 0$. 

\end{axiom}

\end{forthel}
Wir zeigen zwei nützliche Lemmas. Das erste spart an einigen Stellen Schreibarbeit. Rudin braucht kein derartiges Lemma, da aus der Assoziativität und der Kommutativität folgt, dass man in einem Term, der nur Addition enthält, die Klammern beliebig setzen und die atomaren Terme beliebig permutieren kann. Wir haben es jedoch nicht geschafft, eine analoge Aussage in FoTtheL zu formulieren geschweige denn zu beweisen. Also mussten wir das gesamte Kapitel hindurch alle Klammern und Variablen "per Hand" verschieben, was zu einem starken "Aufblasen" der Rechnungen geführt hat.\newline

Das zweite Lemma entspricht der 3.ten binomischen Formel, die an einigen Stellen hilfreich ist.
\begin{forthel}

\begin{lemma} $(c + d) + (e  + f) = (c + e) + (f   + d)$.
\end{lemma}
\begin{proof}\begin{align*}
&(c + d) + (e  + f)	= ((c +  d) +  e)  + f \\
= &( c + (d  +  e)) + f \\
= &( c + (e  +  d)) + f \\
= &((c +  e) +  d)  + f \\
=  &(c +  e) + (f   + d).
\end{align*}
\end{proof}


\begin{lemma}
 $(x \cdot x) - (y \cdot y) = (x+y) \cdot (x-y)$.
\end{lemma}
\begin{proof}\begin{align*}
 &(x+y) \cdot (x-y) 	= (x+y) \cdot (x+ -y) \\
= &(((x+y) \cdot x)+ ((x+y) \cdot (-y)))\\
= &( ((x \cdot x) + (y \cdot x)) + ((x \cdot (-y)) + (y \cdot (-y))) )\\
= &( ((x \cdot x) + (y \cdot x)) + ((x \cdot (-y)) + -(y \cdot y)) )\\
= &( (x \cdot x) + ((y \cdot x) + ((x \cdot (-y)) + -(y \cdot y))) )\\
= &( (x \cdot x) + (((y \cdot x) + (x \cdot (-y))) + -(y \cdot y) ) )\\
= &( (x \cdot x) + (((y \cdot x) + -(x \cdot y)) + -(y \cdot y) ) )\\
= &( (x \cdot x) + (((y \cdot x) + -(y \cdot x)) + -(y \cdot y) ) )\\
= &( (x \cdot x) + (0 + -(y \cdot y) ) )\\
= &(x \cdot x) - (y \cdot y).
\end{align*}
\end{proof}






\end{forthel}
Nun sind wir bereit Komplexe Zahlen einzuführen. Anders als Rudin beweisen wir die Abgeschlossenheit von $\CC$ bzgl. Addition und Multplikation nicht, da diese vorrausgesetzt werden muss, um die Operatoren für Add. und Mult. in ForTheL zu definieren. Die Theoreme CmpFld2 - CmpFld5 entsprechen den Körperaxiomen für Addition und CmpFld6 - CmpFld9 entsprechen denen für Multiplikation. 
\begin{forthel}


\begin{signature} A complex number is a notion.
\end{signature}

Let $u,v,w$ denote complex numbers.
\begin{signature} Let $x,y$ be real numbers. $(x,y)$ is a complex number.
\end{signature}

\begin{axiom} If $u$ is a complex number then there exist real numbers $x,y$ such that $u = (x,y)$.
\end{axiom}

\begin{signature} $u  \cadd  v$ is a complex number.
\end{signature}

\begin{signature} $u  \cmul  v$ is a complex number.
\end{signature}

\begin{axiom}[CAdd] $(x, y)  \cadd  (a, b) = (x + a, y + b)$ .
\end{axiom}

\begin{axiom}[CMult] $(x, y)  \cmul  (a, b) = ((x \cdot a) - (y \cdot b), (x \cdot b) + (y \cdot a))$.
\end{axiom}




\begin{theorem}[CmpFld2]  $u  \cadd  v = v  \cadd  u$.
\end{theorem}
\begin{proof} 
Let $x,y,a,b$ be real numbers such that ($u = (x,y)$ and $v = (a,b)$).\newline
Then $u  \cadd  v = (x,y)  \cadd  (a,b)$. \newline
Then $(x,y)  \cadd  (a,b) = (x + a, y + b)$ (by CAdd). \newline	
$(x + a, y + b) = (a + x, b + y)$. \newline
$(a + x, b + y) = (a,b)  \cadd  (x,y)$ (by CAdd).\newline
$(a,b)  \cadd  (x,y) = v  \cadd  u$. \newline
Then $u  \cadd  v = v  \cadd  u$.
\end{proof}
 



\begin{theorem}[CmpFld3] $(u  \cadd  v)  \cadd  w = u  \cadd  (v  \cadd  w)$.
\end{theorem}
\begin{proof} 
Let $x,y,a,b,r,s$ be real numbers such that ($u = (x,y)$ and $v = (a,b)$ and $w = (r,s)$).\newline
Then $(u  \cadd  v)  \cadd  w = ((x+a)+r, (y+b)+s)$ (by CAdd). \newline
$((x+a)+r, (y+b)+s)= (x+(a+r), y+(b+s))$.\newline
$(x+(a+r), y+(b+s)) = u  \cadd  (v  \cadd  w)$ (by CAdd).\newline
Then $(u \cadd v) \cadd w=u \cadd (v \cadd w)$. 
\end{proof}


\begin{theorem}[CmpFld4] $u  \cadd  (0,0) = u$.
\end{theorem}\begin{proof} 
Let $x,y$ be real numbers such that $u = (x,y)$. Then $u  \cadd  (0,0) = (x+0,y+0)$ (by CAdd).\newline
 $(x+0,y+0) = (x,y) = u$. 
\end{proof}




\begin{theorem}[CmpFld5] For every complex number $u$ there exists a complex number $v$ such that $u  \cadd  v = (0,0)$.
\end{theorem}\begin{proof}
 Let $u$ be a complex number.\newline
Take real numbers $x,y$ such that $u = (x,y)$. Take $v = (-x, -y)$.\newline
 Then  $u  \cadd  v = (x-x, y-y)(by CAdd) . (x-x, y-y) = (0,0)$. \newline
Then $v$ is a complex number such that $u \cadd v = (0,0)$.

\end{proof}


\begin{theorem}[CmpFld6] $u \cmul v = v \cmul u$.
\end{theorem}
\begin{proof}
 Let $x,y,a,b$ be real numbers such that ($u = (x,y)$ and $v = (a,b)$).\newline
Then $u \cmul v =  ((x \cdot a) - (y \cdot b), (x \cdot b) + (y \cdot a)) =  ((a \cdot x) - (b \cdot y), (a \cdot y) + (b \cdot x)) = v \cmul u$.\newline
Then $u \cmul v = v \cmul u$. 
 \end{proof}


\begin{theorem}[CmpFld7] $(u \cmul v) \cmul w = u \cmul (v \cmul w)$.
\end{theorem}\begin{proof}
 Let $x,y,a,b,r,s$ be real numbers such that ($u = (x,y)$ and $v = (a,b)$ and $w = (r,s)$).\newline 
\begin{align*}
&(u \cmul v) \cmul w \\
= &(((x \cdot a) - (y \cdot b), (x \cdot b) + (y \cdot a)))  \cmul  w \\
= &(  ( ((x \cdot a)  - (y \cdot b)) \cdot r ) - ( ((x \cdot b) + (y \cdot a)) \cdot s )  ,\\
  &( ((x \cdot a)  - (y \cdot b)) \cdot s ) + ( ((x \cdot b) + (y \cdot a)) \cdot r ) )\\
= &(  ( ((x \cdot a) + -(y \cdot b)) \cdot r ) - ( ((x \cdot b) + (y \cdot a)) \cdot s )  ,  \\
  &( ((x \cdot a) + -(y \cdot b)) \cdot s ) + ( ((x \cdot b) + (y \cdot a)) \cdot r ) )\\
= &(  ( ((x \cdot a) + -(y \cdot b)) \cdot r ) - ( ((x \cdot b) \cdot s) + ((y \cdot a) \cdot s) )  , \\
  &( ((x \cdot a) + -(y \cdot b)) \cdot s ) + ( ((x \cdot b) \cdot r) + ((y \cdot a) \cdot r) )  )\\
= &(  ( ((x \cdot a) \cdot r) + ((-(y \cdot b)) \cdot r) ) - ( ((x \cdot b) \cdot s) + ((y \cdot a) \cdot s) )  ,  \\
  &( ((x \cdot a) \cdot s) + ((-(y \cdot b)) \cdot s) ) + ( ((x \cdot b) \cdot r) + ((y \cdot a) \cdot r) )  )\\ 
= &(  ( ((x \cdot a) \cdot r) + -((y \cdot b) \cdot r) ) - ( ((x \cdot b) \cdot s) + ((y \cdot a) \cdot s) )  ,  \\
  &( ((x \cdot a) \cdot s) - ((y \cdot b) \cdot s) ) + ( ((x \cdot b) \cdot r) + ((y \cdot a) \cdot r) )  )\\
= &(  ( ((x \cdot a) \cdot r) + -((y \cdot b) \cdot r) ) + ( -((x \cdot b) \cdot s) + -((y \cdot a) \cdot s) )  ,  \\
  &( ((x \cdot a) \cdot s) + -((y \cdot b) \cdot s) ) + ( ((x \cdot b) \cdot r) + ((y \cdot a) \cdot r) )  ).\\
\end{align*}
Let $c = (x \cdot a) \cdot r$ and $d = -((y \cdot b) \cdot r)$ and $e = -((x \cdot b) \cdot s)$ and $f = -((y \cdot a) \cdot s)$ and $g = ((x \cdot a) \cdot s)$ and  $h = -((y \cdot b) \cdot s)$ and  $j = ((x \cdot b) \cdot r)$ and $k = ((y \cdot a) \cdot r)$.\newline
Then $(u \cmul v) \cmul w =  ( (c + d) + (e  + f) , (g + h) +  (j + k)  )
=  ( (c + e) + (d + f) , (g +  j) + (h  + k) )
= ( (c + e) + (f + d) , (g + j) + (k + h) )$.\newline
Then $v  \cmul  w = ((a \cdot r) - (b \cdot s),(a \cdot s) + (b \cdot r))$ (by CMult).\newline
\begin{align*}
&u \cmul (v \cmul w) \\
= &u  \cmul  ((a,b)  \cmul  (r,s)) \\
= &u  \cmul  (  ((a \cdot r) - (b \cdot s) , (a \cdot s) + (b \cdot r))  )\\
= &(  ( x  \cdot  ((a \cdot r) -(b \cdot s)) ) - (y \cdot ((a \cdot s) + (b \cdot r))) , \\
  &(x \cdot ((a \cdot s) + (b \cdot r))) + (y \cdot ((a \cdot r) - (b \cdot s))))\\
= &((x \cdot ((a \cdot r) + -(b \cdot s))) + -(y \cdot ((a \cdot s) + (b \cdot r))) , \\
  &(x \cdot ((a \cdot s) 	+ (b \cdot r))) + (y \cdot ((a \cdot r) + -(b \cdot s))))\\
= &(  ( (x \cdot (a \cdot r)) + (x \cdot (-(b \cdot s))) ) + -( (y \cdot (a \cdot s)) + (y \cdot (b \cdot r)) )  , \\
  &( (x \cdot (a \cdot s)) + (x \cdot (b \cdot r)) ) + ( (y \cdot (a \cdot r)) + (y \cdot (-(b \cdot s))) ) )\\
= &(  ( (x \cdot (a \cdot r)) + -(x \cdot (b \cdot s)) ) + ( -(y \cdot (a \cdot s)) + -(y \cdot (b \cdot r)) )  ,  \\
  &( (x \cdot (a \cdot s)) + (x \cdot (b \cdot r)) ) + ( (y \cdot (a \cdot r)) + -(y \cdot (b \cdot s)) ) )\\
= &(  ( ((x \cdot a) \cdot r) + -((x \cdot b) \cdot s) ) + ( -((y \cdot a) \cdot s) + -((y \cdot b) \cdot r) )  ,  \\
  &( ((x \cdot a) \cdot s) + ((x \cdot b) \cdot r) ) + ( ((y \cdot a) \cdot r) + -((y \cdot b) \cdot s) ) )\\
= &( (c + e) + (f + d) , (g + j) + (k + h) ) = (u \cmul v) \cmul w.
\end{align*}
Then $(u \cmul v) \cmul w = u \cmul (v \cmul w)$. \end{proof}



\begin{theorem}[CmpFld8] $u \cmul (1,0) = u$.
\end{theorem}\begin{proof}
 Let $x,y$ be real numbers such that $u = (x,y)$.\newline
 Then $u  \cmul  (1,0) = ((x,y))  \cmul  (1,0) = ((x \cdot 1)-(y \cdot 0) , (x \cdot 0)+(y \cdot 1))$. \newline
Let $a = x \cdot 1$ and $b = y \cdot 0$ and $c = x \cdot 0$ and $d = y \cdot 1$.\newline
$((x \cdot 1)-(y \cdot 0) , (x \cdot 0)+(y \cdot 1)) = (a-b, c+d)$.\newline
Then $a = x$ and $b = 0$ and $c = 0$ and $d = y$.\newline
$(a-b, c+d) = (x-0 , 0+y) = (x,y) = u$ .\newline
Then $u \cmul (1,0) = u$.
\end{proof}



\end{forthel}
Im Folgenden muss die Relation $<$ eingeführt werden, da man sie nutzten kann, um die Existenz von multiplikativen Inversen in $\CC \verb|\| \{ 0 \}$ zu zeigen. Wir übertragen wieder Aussagen über $>$ aus vorherigen Kaptieln und leiten das eine oder andere Lemma für spätere Verwendung her. Diese Lemmas verwendet Rudin an ein paar Stellen implizit, ohne sie formal zu zeigen. Das Axiom SqrtEind ist ein Spezialfall des im Kapitel The Real Field bewiesenen Theorems über Eindeutigkeit und Existenz von Wurzeln.
\begin{forthel}


\begin{signature}. $x < y$ is an atom.

\end{signature}
Let $x  >  y$ stand for $y  <  x$.\newline
Let $x  \leq  y$ stand for $x < y$ or $x=y$.\newline
Let $x  \geq  y$ stand for $y  \leq  x$.

\begin{axiom}[InEqCompl] $x < y$ or $y < x$ or $x = y$.

\end{axiom}
\begin{axiom} If $x < y$ then not $x=y$.

\end{axiom}

\begin{axiom}[Trans] If $x < y$ and $y < z$ then $x < z$.

\end{axiom}


\begin{axiom}[InEqAdd] If $y < z$ then $x+y < x+z$.

\end{axiom}
\begin{axiom}[InEqMult] If $x > 0$ and $y > 0$ then $x \cdot y > 0$.

\end{axiom}


\begin{axiom}[P118d] If not $x = 0$ then $x \cdot x  >  0$.

\end{axiom}
\begin{axiom}[P118e1] If $0 < y$ then $0  <  inv(y)$.

\end{axiom}

\begin{axiom}[SqrtEind] If $x \cdot x = y \cdot y and x \geq 0 and y  \geq  0$ then $x=y$.

\end{axiom}

\begin{lemma}[SqrtMon] If $x \cdot x  >  y \cdot y $ and $x  >  0$  and $y  \geq  0 $ then $x > y$.
\end{lemma}\begin{proof}
	Assume $x \cdot x  >  y \cdot y$ and $x > 0$ and $y  \geq  0$. \newline
Thus  $-(y \cdot y) + (x \cdot x)  >  -(y \cdot y) + (y \cdot y)$ (by InEqAdd).\newline 
Hence $(x \cdot x) - (y \cdot y)  >  0$.\newline
Then  $(x + y)  \cdot  (x - y)  >  0$.\newline
$x + y  >  0$.\newline
proof. 	$x  >  0$. Then $x + y  >  y$.\newline
case $y  >  0$. Then $x + y  >  0$ (by Trans). end.\newline
case $y = 0$. Then $x + y   >  0$ . end.\newline
end.\newline
Thus $inv(x + y)  >  0$. Hence $inv(x + y)  \cdot  ((x + y)  \cdot  (x - y))  >  0$.\newline
We have $inv(x + y)  \cdot  ((x + y)  \cdot  (x - y)) 
= (inv(x + y)  \cdot  (x + y))  \cdot  (x - y)
= 1  \cdot  (x - y) = x - y$.\newline
Then $x - y  >  0$.\newline
Thus $(x + -y) + y  >  0 + y$.\newline
Hence $x + (-y + y)  >  0 + y$.\newline
\end{proof}


\begin{lemma}[LeqTrans] If $x \leq y$ and $y \leq z$ then $x \leq z$.

\end{lemma}
\begin{proof}
 	Obvious.\end{proof}



\begin{lemma}[LeqAdd] 	If $y  \leq  z$ then $x + y  \leq  x+z$.

\end{lemma}
\begin{proof}
Case $y  <  z$. Then $x + y  \leq  x + z$ (by InEqAdd). end.\newline
Case $y = z$. Then $x+y = x+z$. end.
\end{proof}


\begin{lemma}[LeqAdd2]  If $a  \leq  b$ and $c  \leq  d$ then $a+c  \leq  b+d$.

\end{lemma}
\begin{proof}
Assume $a  \leq  b$ and $c  \leq  d$. \newline
Case $a=b$. Then $a+c = b+c$. $b+c  \leq  b+d$ (by LeqAdd). Thus $a+c  \leq  b+d$. end.\newline
Case $a  <  b$. Then $c+a  <  c+b$. Thus $c+a  \leq  c+b$. Hence $a+c  \leq  b+c$. $b+c  \leq  b+d$ (by LeqAdd).\newline
Then $a+c  \leq  b+d$ (by LeqTrans). end.
\end{proof}



\begin{lemma}[21O] If (not $x = 0$) or (not $y = 0$) then $(x \cdot x) + (y \cdot y) > 0$.

\end{lemma}
\begin{proof}
Assume (not $x=0$) or (not $y=0$).\newline
Let $a = x \cdot x$ and $b = y \cdot y$.\newline
Then $a >  0$ or $b > 0$.\newline
Then $a \geq  0$ and $b \geq 0$.\newline
Let us show that $a + b  >  0$.\newline
Case $a > 0$. Then $a+b \geq a+0 \geq a > 0$. end.\newline
Case $b > 0$. Then $a+b \geq 0+b \geq b > 0$. end.\newline
end.
\end{proof}


\begin{theorem}[CmpFld9] If not $u = (0,0)$ then there exists a complex number $v$ such that $u \cmul v = (1,0)$.
\end{theorem}\begin{proof}
 Assume not $u = (0,0)$.\newline
Let $x,y$ be real numbers such that $u = (x,y)$.\newline
Then (not $x=0$) or (not $y=0$).\newline
Then $(x \cdot x) + (y \cdot y) > 0$ (by 21O).\newline
Then not $(x \cdot x) + (y \cdot y) = 0$ .\newline
Let $d = inv((x \cdot x) + (y \cdot y))$ and  $v = (x \cdot d , -y \cdot d)$. \newline
Then \begin{align*}
&u \cmul v \\
= &( (x \cdot (x \cdot d)) - (y \cdot (-y \cdot d)) , (x \cdot (-y \cdot d)) + (y \cdot (x \cdot d)))\\
= &( (x \cdot (x \cdot d)) + -(-y \cdot (y \cdot d)) , -(x \cdot (y \cdot d)) + (y \cdot (x \cdot d)))\\
= &( (x \cdot (x \cdot d)) + (y \cdot (y \cdot d)) , -(x \cdot (y \cdot d)) + (y \cdot (x \cdot d)))\\
= &( ((x \cdot x) \cdot d) + ((y \cdot y) \cdot d) , -((x \cdot y) \cdot d) + ((y \cdot x) \cdot d))\\
= &( ((x \cdot x) + (y \cdot y)) \cdot d , -((x \cdot y) \cdot d) + ((x \cdot y) \cdot d))\\
= &( ((x \cdot x) + (y \cdot y)) \cdot inv((x \cdot x) + (y \cdot y)) , 0)\\
= &(1,0).
\end{align*}
Then $u \cmul v = (1,0)$.\newline
Then $v$ is a complex number such that $u \cmul v = (1,0)$.\newline
If not $u = (0,0)$ then there exists a complex number w such that $u \cmul w = (1,0)$.
\end{proof}





\begin{theorem}[CmpFld10] $u \cmul (v \cadd w) = (u \cmul v)  \cadd  (u \cmul w)$.
\end{theorem}\begin{proof}
	Let $x,y,a,b,r,s$ be real numbers such that ($u = (x,y)$ and $v = (a,b)$ and $w = (r,s)$).\newline
Let $c = x \cdot a$ and $d = x \cdot r$ and $e = -(y \cdot b)$ and $f = -(y \cdot s)$ and $g = x \cdot b$ and $ h = x \cdot s$ and  $j = y \cdot a$ and $k = y \cdot r$.\newline
$v \cadd w = (a+r,b+s)$ (by CAdd).\newline
$u \cmul v = (x,y)  \cmul  (a,b) = ((x \cdot a) - (y \cdot b), (x \cdot b) + (y \cdot a))$ (by CMult).  \newline
$u \cmul w = (x,y)  \cmul  (r,s) = ((x \cdot r) - (y \cdot s) , (x \cdot s) + (y \cdot r))$ (by CMult). \newline
Then $u \cmul (v \cadd w) 	= (x,y)  \cmul  (a+r,b+s) = ( (x \cdot (a+r)) - (y \cdot (b+s)) , (x \cdot (b+s)) + (y \cdot (a+r)) )$.\newline

$( (x \cdot (a+r)) - (y \cdot (b+s)) , (x \cdot (b+s)) + (y \cdot (a+r)) )
= ( ((x \cdot a) + (x \cdot r)) - (y \cdot (b+s)) , (x \cdot (b+s)) + (y \cdot (a+r)) )
= ( ((x \cdot a) + (x \cdot r)) - ((y \cdot b) + (y \cdot s)) , (x \cdot (b+s)) + (y \cdot (a+r)) )$.\newline

$( ((x \cdot a) + (x \cdot r)) - ((y \cdot b) + (y \cdot s)) , (x \cdot (b+s)) + (y \cdot (a+r)) )
= ( ((x \cdot a) + (x \cdot r)) - ((y \cdot b) + (y \cdot s)) , ((x \cdot b) + (x \cdot s)) + (y \cdot (a+r)) )$ (by Dis1).\newline

$( ((x \cdot a) + (x \cdot r)) - ((y \cdot b) + (y \cdot s)) , ((x \cdot b) + (x \cdot s)) + (y \cdot (a+r)) )
= ( ((x \cdot a) + (x \cdot r)) - ((y \cdot b) + (y \cdot s)) , ((x \cdot b) + (x \cdot s)) + ((y \cdot a) + (y \cdot r)) )$ (by Dis1).\newline

\begin{align*}
&( ((x \cdot a) + (x \cdot r)) - ((y \cdot b) + (y \cdot s)) , ((x \cdot b) + (x \cdot s)) + ((y \cdot a) + (y \cdot r)) )\\
= &(((x \cdot a) + (x \cdot r)) + -((y \cdot b) + (y \cdot s)) , ((x \cdot b) + (x \cdot s)) + ((y \cdot a) + (y \cdot r)) )\\
= &(((x \cdot a) + (x \cdot r)) + (-(y \cdot b) + -(y \cdot s)) , ((x \cdot b) + (x \cdot s)) + ((y \cdot a) + (y \cdot r)) )\\
= &((c + d) + (e + f) , (g + h) + (j + k))\\
= &((c + e) + (d + f) , (g + j) + (h + k))\\
= &(c+e,g+j)  \cadd  (d+f,h+k)\\
= &((x \cdot a) + -(y \cdot b), (x \cdot b) + (y \cdot a))  \cadd  ((x \cdot r) + -(y \cdot s) , (x \cdot s) + (y \cdot r))\\
= &((x \cdot a) - (y \cdot b), (x \cdot b) + (y \cdot a))  \cadd  ((x \cdot r) - (y \cdot s) , (x \cdot s) + (y \cdot r))\\
= &((x,y)  \cmul  (a,b))  \cadd  ((x,y)  \cmul  (r,s)) \\
= &(u \cmul v)  \cadd  (u \cmul w).\\
\end{align*}
Then $u \cmul (v \cadd w) = (u \cmul v)  \cadd  (u \cmul w)$.
\end{proof}


\begin{lemma}[ZeroMult] $u \cmul (0,0) = (0,0)$.
\end{lemma}\begin{proof}
 Let $x,y$ be real numbers such that $u = (x,y)$. \newline
  Then $u \cmul (0,0) = ((x \cdot 0) - (y \cdot 0), (x \cdot 0) + (y \cdot 0))$ (by CMult). \newline
Let $a=x \cdot 0$ and $b=y \cdot 0$.\newline
$((x \cdot 0) - (y \cdot 0), (x \cdot 0) + (y \cdot 0)) = (a-b,a+b) = (0 - 0, 0 + 0) = (0,0)$. \newline
Then $u \cmul (0,0) = (0,0)$. 
\end{proof}


\begin{theorem}
 $(a,0)  \cadd  (b,0) = (a+b,0)$ and $(a,0) \cmul (b,0) = (a \cdot b,0)$.
\end{theorem}\begin{proof}
  $(a,0)  \cadd  (b,0) = (a+b,0+0) = (a+b,0)$ (by CAdd).\newline
$(a,0)  \cmul  (b,0) = ((a \cdot b) - (0 \cdot 0), (a \cdot 0) + (0 \cdot b))$ (by CMult).\newline
Let $c = a \cdot 0$ and $d = b \cdot 0$ and $e = 0 \cdot 0$.\newline
Then $c = 0$ and $d = 0$ and $e = 0$.\newline
Then $((a \cdot b) - e, c + d) = ((a \cdot b) -0, 0 + 0) = ((a \cdot b )+ -0,0) = (a \cdot b,0)$ . \end{proof}


\begin{signature} $\di $ is a complex number such that $\di  = (0,1)$.

\end{signature}

\begin{theorem} $\di  \cmul \di  = (-1,0)$.
\end{theorem}\begin{proof}
  $\di  \cmul \di  = (0,1)  \cmul  (0,1) 
= ((0 \cdot 0) - (1 \cdot 1),( 0 \cdot 1) + (1 \cdot 0))
=((0 \cdot 0) - (1 \cdot 1),( 0 \cdot 1) + (1 \cdot 0)) 
= (0 - 1, 0+0) 
= (0+(-1),0)$.\newline
Then $(0+(-1),0) = ((-1),0)$. 
\end{proof}


Let $x+y\di $ stand for $(x,0)  \cadd  ((y,0) \cmul \di )$.


\begin{theorem} $a+b\di  = (a,b)$.
\end{theorem}\begin{proof}
$(b,0) \cmul (0,1) = ((b \cdot 0)-(0 \cdot 1) , (b \cdot 1)+(0 \cdot 0))$ (by CMult).\newline
$a+b\di  = (a,0)  \cadd  ((b,0) \cmul (0,1))$. \newline

$(a,0)  \cadd  ((b,0) \cmul (0,1))
= (a,0)  \cadd  ((b \cdot 0)-(0 \cdot 1) , (b \cdot 1)+(0 \cdot 0))$ (by CMult).\newline

$(a,0)  \cadd  ((b \cdot 0)-(0 \cdot 1) , (b \cdot 1)+(0 \cdot 0))
= (a,0)  \cadd  (0-0,b+0) 
= (a,0)  \cadd  (0,b)$.\newline

$(a,0) \cadd (0,b)
= (a+0,0+b) 
= (a,b)$.
\end{proof}

\end{forthel}
Im Folgenden beweisen wir Theorem 1.31. Dies entspricht bei uns Conj1 - Conj4. Dabei zeigen wir auch die Aussage Conj5, um später Schreibarbeit zu sparen. Rudin zeigt diese Aussage nicht, da er die relevanten Beweise wegen Trivialität weglässt. \newline
Wir definieren dann den Absolutbetrag. Da SAD kein overloaden von Operatoren erlaubt, können wir dies nur für Komplexe Zahlen tun, was zu etwas umständlicher Notation an führt (z.B. müssen wir im Beweis von Abs5 ständig $|(Re(u),0)|$ statt $|Re(u)|$ schreiben).
sWir nehmen das Axiom EindAbs hinzu, welches die Eindeutigkeit des Betrags garantiert. Deren Gültigkeit folgt aus dem Theorem über Eindeutigkeit von Wurzeln.
\begin{forthel}


\begin{signature} $Conj(u)$ is a complex number.

\end{signature}
\begin{signature} $Re(u)$ is a real number.

\end{signature}
\begin{signature} $Im(u)$ is a real number.

\end{signature}
\begin{axiom} $Conj((x,y)) = (x,-y)$.

\end{axiom}
\begin{axiom} $Re((x,y)) = x$.

\end{axiom}
\begin{axiom} $Im((x,y)) = y$.

\end{axiom}

\begin{theorem}[Conj1] $Conj(u \cadd v) = Conj(u)  \cadd  Conj(v)$.
\end{theorem}\begin{proof}
 	Let $x,y,a,b$ be real numbers such that ($u = (x,y)$ and $v = (a,b)$).\newline
Then $u \cadd v = (x+a,y+b)$ (by CAdd).\newline
Then $Conj(u  \cadd  v) = Conj((x+a,y+b)) = (x+a,-(y+b)) = (x+a, -y + -b) = (x,-y)  \cadd  (a,-b) = Conj(u)  \cadd  Conj(v)$.\newline
Then $Conj(u \cadd v) = Conj(u)  \cadd  Conj(v)$.
\end{proof}
		
\begin{theorem}[Conj2] $Conj(u \cmul v) = Conj(u)  \cmul  Conj(v)$.
\end{theorem}\begin{proof}
 	Let $x,y,a,b$ be real numbers such that ($u = (x,y)$ and $v = (a,b)$).\newline
Then $u  \cmul  v = ((x \cdot a)-(y \cdot b),(x \cdot b)+(y \cdot a))$ (by CMult).\newline
Then
\begin{align*}
&Conj(u  \cmul  v) 	= Conj(((x \cdot a)-(y \cdot b),(x \cdot b)+(y \cdot a))) \\
= &((x \cdot a) - (y \cdot b)       ,-((x \cdot b)+(y \cdot a))) \\
= &((x \cdot a) - (y \cdot b)       , -(x \cdot b)+ -(y \cdot a))\\
= &((x \cdot a) - (-(-(y \cdot b))) , -(x \cdot b)+ -(y \cdot a))\\
= &((x \cdot a) - ( -((-y) \cdot b) ) , (x \cdot (-b))+ ((-y) \cdot a))\\
= &((x \cdot a) - ((-y) \cdot (-b)) , (x \cdot (-b))+ ((-y) \cdot a))\\
= &(x,-y)  \cmul  (a,-b)\\
= &Conj(u)  \cmul  Conj(v). 
\end{align*}
Then $Conj(u \cmul v) = Conj(u)  \cmul  Conj(v)$.
\end{proof}


\begin{theorem}[Conj3] $u  \cadd  Conj(u) = (Re(u)+Re(u),0)$ and $u  \cadd  ((-1,0) \cmul Conj(u)) = (0,Im(u)+Im(u))$.
\end{theorem}\begin{proof}
 Let $x,y$ be real numbers such that $u=(x,y)$.\newline
Then $u  \cadd  Conj(u) = (x,y)  \cadd  Conj((x,y)) = (x,y)  \cadd  (x,-y) = (x+x,y-y) = (x+x,0) = (Re(u)+Re(u),0)$.\newline
$(-1,0) \cmul (x,-y) = ( ((-1) \cdot x) - (0 \cdot (-y)) , ((-1) \cdot (-y)) + (0 \cdot x))$ (by CMult).\newline
$u  \cadd  ((-1,0) \cmul Conj(u))$\newline$	
= (x,y)  \cadd  ((-1,0) \cmul Conj((x,y))) $\newline$
= (x,y)  \cadd  ((-1,0) \cmul (x,-y)) $\newline$
= (x,y)  \cadd  ( ((-1) \cdot x) - (0 \cdot (-y)) , ((-1) \cdot (-y)) + (0 \cdot x))$\newline$
= (x,y)  \cadd  (-x - 0, -(-y) + 0)$\newline$
= (x,y)  \cadd  (-x , y) = (x-x,y+y) = (0,Im(u)+Im(u))$ \newline
Then $u  \cadd  Conj(u) = (Re(u)+Re(u),0)$ and $u  \cadd  ((-1,0) \cmul Conj(u)) = (0,Im(u)+Im(u))$.
\end{proof}

\begin{theorem}[Conj4] If not $u = (0,0)$ then there exists a real number $z$ such that ($z > 0$ and $u  \cmul  Conj(u) = (z,0)$).
\end{theorem}\begin{proof}
	Let not $u = (0,0)$.\newline
Take real numbers $x,y$ such that $u = (x,y)$.\newline
Then $(x \cdot x) >  0$ or $(y \cdot y) > 0$. Take $z = (x \cdot x) + (y \cdot y)$. \newline
Then $z  >  0$ (by 21O).\newline
$u \cmul Conj(u) = (x,y) \cmul Conj((x,y)) $\newline
$= (x,y) \cmul (x,-y)$ \newline
$= ((x \cdot x) - (y \cdot (-y)) , (x \cdot (-y)) + (y \cdot x)) $\newline
$= ((x \cdot x) + -(-(y \cdot y)) , -(x \cdot y) + (x \cdot y)) $\newline
$= ((x \cdot x) + (y \cdot y), 0) = (z,0)$.\newline
$z$ is a real number.\newline
Then $z$ is a real number and ($z > 0$ and $u \cmul Conj(u) = (z,0)$).\newline
\end{proof}

\begin{theorem}[Conj5] $Conj(Conj(u)) = u$.
\end{theorem}\begin{proof}
 	Let $x,y$ be real numbers such that $u = (x,y)$.\newline
Then $Conj(Conj(u)) = Conj(Conj((x,y))) = Conj((x,-y)) = (x,--y) = (x,y) = u$. \end{proof}

\begin{signature}. $|u|$ is a real number such that $(|u| \cdot |u|,0) = u \cmul Conj(u)$ and $|u|  \geq  0$. \end{signature}
\begin{axiom}[EindAbs] If $(x \cdot x,0) = u \cmul Conj(u)$ and $x  \geq  0$ then $x = |u|$.\end{axiom}

\begin{theorem}[Abs1] If not $u = (0,0)$ then $|u|  >  0$.
\end{theorem}\begin{proof}
  Let $z$ be a real number such that ($z  >  0$ and $u \cmul Conj(u) = (z,0)$). Then $(|u| \cdot |u|,0) = (z,0)$.\newline
Assume $|u| = 0$. Then $z = 0 \cdot 0 = 0$. Contradiction. \newline
Then $|u|  >  0$.
\end{proof}
\begin{theorem}[Abs2] $|(0,0)| = 0$.
\end{theorem}\begin{proof}
 	$(0,0)  \cmul  Conj((0,0)) = Conj((0,0))  \cmul  (0,0) = (0,0)$. \newline
Let us show that $|(0,0)| \cdot |(0,0)| = 0$.\newline
Assume $|(0,0)|  >  0$. Then not $|(0,0)| = 0$ .Then $|(0,0)| \cdot |(0,0)|  >  0$. Then not $|(0,0)| \cdot |(0,0)| = 0$. Contradiction.end. \newline
Hence $|(0,0)| = 0$.
\end{proof}
\begin{theorem}[Abs3] $|u| = |Conj(u)|$.
\end{theorem}\begin{proof}
 	$|u|$ is a real number such that $((|u| \cdot |u|,0) = u \cmul Conj(u) and |u|  \geq  0)$. \newline
$(|u| \cdot |u|,0) = Conj(Conj(u)) \cmul Conj(u) = Conj(u) \cmul Conj(Conj(u))$.\newline
Let $v = Conj(u)$. Then $|u|$ is real number such that ($(|u| \cdot |u|,0) = v \cmul Conj(v)$ and $|u|  \geq  0$). \newline
Hence $|u| = |v|$.
\end{proof}
\begin{theorem}[Abs4] $|u \cmul v| = |u|  \cdot  |v|$.
\end{theorem}\begin{proof}
 Case $u = (0,0)$ or $v = (0,0)$. Then $v \cmul u = (0,0)$ or $u \cmul v = (0,0)$ (by ZeroMult). \newline
Hence $|u \cmul v| = 0$. $|u| = 0$ or $|v| = 0$ (by Abs2).\newline
Hence $|u|  \cdot  |v| = 0$. Then  $|u \cmul v| = |u| \cdot |v|$. end.\newline
Case (not $u=(0,0)$) and (not $v = (0,0)$). Then $|u|  >  0$ and $|v|  >  0$ (by Abs1). \newline
Let $z = |u|  \cdot  |v|$. Then $z  >  0$.\newline
$z \cdot z = (|u| \cdot |v|)  \cdot  (|u| \cdot |v|) $\newline$
= ((|u| \cdot |v|) \cdot |u|) \cdot |v| 
= (|u| \cdot (|v| \cdot |u|)) \cdot |v| $\newline$
= (|u| \cdot (|u| \cdot |v|)) \cdot |v| 
= ((|u| \cdot |u|) \cdot |v|) \cdot |v| $\newline$
= (|u| \cdot |u|) \cdot (|v| \cdot |v|)$.\newline
Hence 
$(z \cdot z,0)	= ((|u| \cdot |u|) \cdot (|v| \cdot |v|), 0) = (|u| \cdot |u|,0)  \cmul  (|v| \cdot |v|,0) = (u   \cmul   Conj(u))  \cmul  (v    \cmul  Conj(v)) $\newline$
= ((u  \cmul   Conj(u))  \cmul   v)   \cmul  Conj(v) = ( u  \cmul  (Conj(u)   \cmul   v))  \cmul  Conj(v) $\newline$
= ( u  \cmul  (v   \cmul   Conj(u)))  \cmul  Conj(v) = ((u  \cmul   v)  \cmul   Conj(u))   \cmul  Conj(v) $\newline$
= ( u  \cmul   v)  \cmul   (Conj(u)    \cmul  Conj(v)) = (u \cmul v) \cmul Conj(u \cmul v)$.\newline
Then $z = |u \cmul v|$. Then $|u|  \cdot  |v| = |u \cmul v|$. end.
\end{proof}
\begin{theorem}[Abs5] $|(Re(u),0)|  \leq  |u|$.
\end{theorem}\begin{proof}
 	Let $x,y$ be real numbers such that $u = (x,y)$. \newline
Case $y = 0$. Then $|u| = |(x,y)| = |(Re(u),0)|$. end.\newline
Case not $y = 0$. Then not $u = (0,0)$. Thus $|u|  >  0$.\newline
$(|u| \cdot |u|,0) $\newline$
= u \cmul Conj(u) = (x,y)  \cmul  (x,-y)  = ((x \cdot x) - (y \cdot (-y)) , (x \cdot (-y)) + (y \cdot x)) $\newline$
= ((x \cdot x) + -(-(y \cdot y)) , -(x \cdot y) + (x \cdot y))  = ((x \cdot x) + (y \cdot y), 0)$.\newline
Then $|u| \cdot |u| = (x \cdot x)+(y \cdot y)$.\newline
$(|(Re(u),0)| \cdot |(Re(u),0)|,0) $\newline$
= (Re(u),0) \cmul Conj((Re(u),0)) $\newline$
= (Re(u),0) \cmul (Re(u),-0) $\newline$
= (Re(u),0) \cmul (Re(u),0) = (x,0) \cmul (x,0)$\newline$
 = (x \cdot x,0)$.\newline
Then $|(Re(u),0)| \cdot |(Re(u),0)| = x \cdot x$.\newline
Hence $(|u| \cdot |u|) -  (|(Re(u),0)| \cdot |(Re(u),0)|) $\newline$
= ((x \cdot x) + (y \cdot y)) - (x \cdot x) = ((y \cdot y) + (x \cdot x)) + -(x \cdot x) $\newline$
= (y \cdot y) + ((x \cdot x) + -(x \cdot x)) = (y \cdot y) + 0 = y \cdot y$ \newline
and $y \cdot y  >  0$.\newline
Thus $(|u| \cdot |u|) -  (|(Re(u),0)| \cdot |(Re(u),0)|)  >  0$.\newline
Hence $|u| \cdot |u|  >  |(Re(u),0)| \cdot |(Re(u),0)|$ and $|u|  >  0$ and $|(Re(u),0)|  \geq  0$.\newline
Thus  $|u|  >  |(Re(u),0)|$ (by SqrtMon).\newline
end.
\end{proof}
\begin{lemma}[AbsMon] $Re(u)  \leq  |(Re(u),0)|$.

\end{lemma}
\begin{proof}

Let $x,y$ be real numbers such that $u = (x,y)$.\newline
Let $v = (x, 0)$.\newline
Then $Re(u) = x = Re(v)$.\newline\newline
Case $x > 0$.\newline
Then $(x \cdot x,0) = v \cmul Conj(v)$.\newline
Then $x = |v|$ (by EindAbs).\newline
Then $Re(u)=Re(v)=|v|=|(Re(v),0)| = |(Re(u),0)|$.\newline
Then $Re(u)  \leq  |(Re(u),0)|$.\newline
end.\newline\newline
Case $x  \leq  0$.\newline
Then $|(Re(u),0)|  \geq 0$.\newline
$x  \leq  0  \leq  |(Re(u),0)|$.\newline
end.
\end{proof}
\begin{theorem}[Abs6] $|u  \cadd  v|  \leq  |u| + |v|$.
\end{theorem}\begin{proof}
 \begin{align*}
 &(|u  \cadd  v| \cdot |u  \cadd  v|,0)	= (u  \cadd  v)  \cmul  Conj(u  \cadd  v)\\
=  &((u  \cadd  v)  \cmul  Conj(u))  \cadd  ((u  \cadd  v)  \cmul  Conj(v)) \\
=  &(Conj(u)  \cmul  (u  \cadd  v))  \cadd  (Conj(v)  \cmul  (u  \cadd  v)) \\
=  &((Conj(u)  \cmul  u)  \cadd  (Conj(u)  \cmul  v))  \cadd  ((Conj(v)  \cmul  u)  \cadd  (Conj(v)  \cmul  v)) \\
=  &((u  \cmul  Conj(u))  \cadd  (v  \cmul  Conj(u)))  \cadd  ((u  \cmul  Conj(v))  \cadd  (v  \cmul  Conj(v)))\\
=  &(u  \cmul  Conj(u))  \cadd  ((v  \cmul  Conj(u))  \cadd  ((u  \cmul  Conj(v))  \cadd  (v  \cmul  Conj(v))))\\
=  &(u  \cmul  Conj(u))  \cadd  (((v  \cmul  Conj(u))  \cadd  (u  \cmul  Conj(v)))  \cadd  (v  \cmul  Conj(v)))\\
=  &(u  \cmul  Conj(u))  \cadd  ((v  \cmul  Conj(v))  \cadd  ((v  \cmul  Conj(u))  \cadd  (u  \cmul  Conj(v)))) \\
=  &(u  \cmul  Conj(u))  \cadd  ((v  \cmul  Conj(v))  \cadd  ((v  \cmul  Conj(u))  \cadd  (Conj(Conj(u))  \cmul  Conj(v))))\\
=  &(u  \cmul  Conj(u))  \cadd  ((v  \cmul  Conj(v))  \cadd  ((v  \cmul  Conj(u))  \cadd  Conj(Conj(u)  \cmul  v) ))\\
=  &(u  \cmul  Conj(u))  \cadd  ((v  \cmul  Conj(v))  \cadd  ((v  \cmul  Conj(u))  \cadd  Conj(v  \cmul  Conj(u)) ))\\
=  &(u  \cmul  Conj(u))  \cadd  ((v  \cmul  Conj(v))  \cadd  (Re(v  \cmul  Conj(u))+Re(v  \cmul  Conj(u)),0) )\\
=  &(|u| \cdot |u|,0)  \cadd  ((|v| \cdot |v|,0)  \cadd  (Re(v  \cmul  Conj(u))+Re(v  \cmul  Conj(u)),0) )\\
=  &((|u| \cdot |u|,0)  \cadd  (|v| \cdot |v|,0))  \cadd  (Re(v  \cmul  Conj(u))+Re(v  \cmul  Conj(u)),0)\\
= &((|u| \cdot |u|) + (|v| \cdot |v|),0+0)  \cadd  (Re(v  \cmul  Conj(u))+Re(v  \cmul  Conj(u)),0)\\
=&(((|u| \cdot |u|) + (|v| \cdot |v|)) + (Re(v  \cmul  Conj(u))+Re(v  \cmul  Conj(u))), 0 + 0)\\
=&(((|u| \cdot |u|) + (|v| \cdot |v|)) + (Re(v  \cmul  Conj(u))+Re(v  \cmul  Conj(u))), 0 )\\
\end{align*}
Thus $|u  \cadd  v| \cdot |u  \cadd  v| = (((|u| \cdot |u|) + (|v| \cdot |v|)) + Re(v  \cmul  Conj(u))) + Re(v  \cmul  Conj(u))$.\newline
Then $Re(v  \cmul  Conj(u))  \leq  |(Re(v  \cmul  Conj(u)),0)|$ (by AbsMon).\newline
$|(Re(v  \cmul  Conj(u)),0)|  \leq  |v \cmul Conj(u)|$ (by Abs5).\newline
Hence $Re(v  \cmul  Conj(u))  \leq  |v \cmul Conj(u)|$ (by LeqTrans).\newline
Thus $((|u| \cdot |u|) + (|v| \cdot |v|)) + Re(v  \cmul  Conj(u))  \leq  ((|u| \cdot |u|) + (|v| \cdot |v|)) + |v \cmul Conj(u)|$ (by LeqAdd).\newline
Thus $(((|u| \cdot |u|) + (|v| \cdot |v|)) + Re(v  \cmul  Conj(u))) + Re(v  \cmul  Conj(u))  \leq  (((|u| \cdot |u|) + (|v| \cdot |v|)) + |v \cmul Conj(u)|)+|v \cmul Conj(u)|$ (by LeqAdd2).
\begin{align*}
&(((|u| \cdot |u|) + (|v| \cdot |v|)) + |v \cmul Conj(u)|)+|v \cmul Conj(u)|\\
= &((|u| \cdot |u|) + (|v| \cdot |v|)) + (|v \cmul Conj(u)|+|v \cmul Conj(u)|)\\ 
= &((|u| \cdot |u|) + (|v| \cdot |v|)) + ((|v| \cdot |Conj(u)|) + (|v| \cdot |Conj(u)|))\\
= &((|u| \cdot |u|) + (|v| \cdot |v|)) + ((|v| \cdot |u|) + (|v| \cdot |u|))\\
= &((|u| \cdot |u|) + (|v| \cdot |u|)) + ((|v| \cdot |v|) + (|v| \cdot |u|))\\
= &((|u| + |v|)  \cdot  |u|) + (|v| \cdot (|v|+|u|))\\
= &(|u|  \cdot  (|u| + |v|)) + (|v|  \cdot  (|u| + |v|))\\
= &(|u| + |v|) \cdot (|u| + |v|).
\end{align*}
Hence $|u  \cadd  v| \cdot |u  \cadd  v|  \leq  (|u| + |v|) \cdot (|u| + |v|)$.\newline
$0  \leq  |u|$ and $0  \leq  |v|$. Thus $0 + 0  \leq  |u| + |v|$ (by LeqAdd2).\newline\newline
	Case $|u| + |v|  >  0$. \newline\newline
Case $|u  \cadd  v| \cdot |u  \cadd  v|  <  (|u| + |v|) \cdot (|u| + |v|)$. \newline
Then $(|u| + |v|) \cdot (|u| + |v|)  >  |u  \cadd  v| \cdot |u  \cadd  v| and |u  \cadd  v|  \geq  0 and |u| + |v|  >  0$ .\newline
Thus $|u  \cadd  v|  <  |u| + |v|$ (by SqrtMon). end.\newline
Case $|u  \cadd  v| \cdot |u  \cadd  v| = (|u| + |v|) \cdot (|u| + |v|)$.\newline
Then $|u  \cadd  v| = |u| + |v|$ (by SqrtEind).end.\newline
end.\newline\newline
	Case $|u| + |v| = 0$. \newline\newline
Thus $(|u| + |v|) \cdot (|u| + |v|) = 0$. Hence $|u  \cadd  v| \cdot |u  \cadd  v|  \leq  0$. \newline 
Let us show that $|u  \cadd  v|  \leq  0$. \newline
proof. 	Assume $|u  \cadd  v|  >  0$. 
Then $|u  \cadd  v| \cdot |u  \cadd  v|  >  0$ (by InEqMult). Contradiction.\newline
Thus not ($|u  \cadd  v|  >  0$). Hence $|u  \cadd  v|  <  0$ or $|u  \cadd  v| = 0$ (by InEqCompl). \newline
end.\newline
Thus $|u  \cadd  v|  \leq  |u| + |v|$.\newline
end.\newline
\end{proof}

\end{forthel}
\end{document}
