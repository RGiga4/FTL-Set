\documentclass{article}
\usepackage[english]{babel}
\usepackage{enumerate, latexsym, amssymb, amsmath}
\usepackage{framed, multicol}
\newenvironment{forthel}{\begin{leftbar}}{\end{leftbar}}

%%%%%%%%%% Start TeXmacs macros
\newcommand{\tmaffiliation}[1]{\\ #1}
\newcommand{\tmem}[1]{{\em #1\/}}
\newenvironment{enumeratenumeric}{\begin{enumerate}[1.] }{\end{enumerate}}
\newenvironment{proof}{\noindent\textbf{Proof\ }}{\hspace*{\fill}$\Box$\medskip}
\newenvironment{quoteenv}{\begin{quote} }{\end{quote}}
\newtheorem{axiom}{Axiom}
\newtheorem{lemma}{Lemma}
\newtheorem{theorem}{Theorem}
\newtheorem{definition}{Definition}
\newtheorem{signature}{Signature}
\newtheorem{proposition}{Proposition}
%%%%%%%%%% End TeXmacs macros

\newcommand{\event}{UITP 2018}
\newcommand{\dom}{Dom}
\newcommand{\fun}{aFunction}
\newcommand{\sym}{sym}
\newcommand{\halfline}{{\vspace{3pt}}}
\newcommand{\tab}{{\hspace{1cm}}}
\newcommand{\ball}[2]{B_{#1}(#2)}
\newcommand{\llbracket}{[}
\newcommand{\rrbracket}{]}
\newcommand{\less}[1]{<_{#1}}
\newcommand{\greater}[1]{>_{#1}}
\newcommand{\leeq}[1]{{\leq}_{#1}}
\newcommand{\supr}[1]{\mathrm{sup}_{#1}}
\newcommand{\RR}{\mathbb{R}}
\newcommand{\QQ}{\mathbb{Q}}
\newcommand{\ZZ}{\mathbb{Z}}
\newcommand{\NN}{\mathbb{N}}
\begin{document}

\title{The Reel Field part A}

\maketitle

Reels-A benutz einen geschikten Trick um P120a zu zeigen ohne schwierige $\epsilon$ Argumente zu benutzen.\\
In einer Version wurde versucht Reels-A mit Reels-B oder Master zu vereinigen, doch gab es große Schwierigkeiten.\\
Deshalb diese Version, wo in Reels-A am ende von dieser Aussage Lis1 , bzw. Lis2, Lis3 in Master gezeigt werden die für Reels-B wichtig sind.\\ 

\begin{forthel}
[set/-s] [element/-s] [number/-s]

\begin{signature} A real number is a notion.

\end{signature}

\begin{definition} $\mathbb{R}$Set.

\end{definition}
$\mathbb{R}$ is the set of real numbers.

Let x,y,z,v, a,b denote real numbers.



\begin{signature} Add. x + y is real number.

\end{signature}
\begin{axiom} x+y = y+x.

\end{axiom}
\begin{axiom} (x+y)+z = x+(y+z).

\end{axiom}
\begin{signature} 0 is a real number such that for every real number x x + 0 = x.

\end{signature}
\begin{signature} -x is a real number such that -x + x = 0.

\end{signature}

Let x - y stand for x + (-y).

\begin{signature} Add. x $\cdot$ y is real number.

\end{signature}
\begin{axiom} x$\cdot$y = y$\cdot$x.

\end{axiom}
\begin{axiom} (x$\cdot$y)$\cdot$z = x$\cdot$(y$\cdot$z).

\end{axiom}
\begin{signature} 1 is a real number such that for every real number x x$\cdot$1 = x.

\end{signature}


\begin{axiom} x$\cdot$(y+z)=(x$\cdot$y)+(x$\cdot$z).

\end{axiom}


\begin{axiom} P114c. If x+y = 0 then y = -x.

\end{axiom}
\begin{axiom} P114d. -(-x) = x.

\end{axiom}

\begin{axiom} P116c. (-x)$\cdot$y = -(x$\cdot$y) = x$\cdot$(-y).

\end{axiom}




\begin{signature} x $<$ y is an atom.

\end{signature}
Let x $>$ y stand for y $<$ x.


\begin{axiom} Then x$<$y or y$<$x or x = y.

\end{axiom}
\begin{axiom} Then not ((x$<$y and y$<$x) or (x$<$y and y=x) or (x=y and y$<$x)).

\end{axiom}
\begin{axiom} If x$<$y and y$<$z then x$<$z.

\end{axiom}







\begin{axiom} A1. If y$<$z then x+y$<$x+z.

\end{axiom}
\begin{axiom} A2. If x$>$0 and y$>$0 then x$\cdot$y$>$0.

\end{axiom}

\begin{axiom} P118a. If x$>$0 then -x$<$0.

\end{axiom}
\begin{axiom} P118a2. If x$<$0 then -x$>$0.

\end{axiom}

\begin{axiom} 1 $>$ 0.

\end{axiom}


\begin{signature} A natural number is a real number.

\end{signature}

\begin{definition} NatSet.

\end{definition}
$\mathbb{N}$ is the set of natural numbers.

\begin{axiom} 0 is natural number.

\end{axiom}
Let n denote a natural number.
\begin{axiom} n+1 is a natural number.

\end{axiom}
\begin{axiom} n $>$ 0 or n = 0.

\end{axiom}




Let R,S,T,A,B,C denote sets.

\begin{definition} DefSubset.   A subset of S is a set T

\end{definition}
such that every element of T is a element of S.

\begin{definition} DefEmpty.    S is empty iff S has no elements.

\end{definition}


\begin{definition} upperBound.

\end{definition}
Assume A is a subset of $\mathbb{R}$.
An upper bound of A is an element b of $\mathbb{R}$ such that ( y$<$b or y = b ) for every element y of A.

\begin{definition} lowerBound.

\end{definition}
Assume A is a subset of $\mathbb{R}$.
A lower bound of A is an element b of $\mathbb{R}$ such that ( b$<$y or y = b ) for every element y of A.

\begin{definition} Supremum.

\end{definition}
Assume A is a subset of $\mathbb{R}$ and not empty.
Let s be an element of $\mathbb{R}$.
s is supremum of A  iff s is an upper bound of A 
and for every element x of $\mathbb{R}$ if x$<$s then x is not an upper bound of A .

\begin{definition} Infimum.

\end{definition}
Assume A is a subset of $\mathbb{R}$ and not empty.
Let s be an element of $\mathbb{R}$.
s is infimum of A  iff s is an lower bound of A 
and for every element x of $\mathbb{R}$ if s$<$x then x is not an lower bound of A .

\begin{definition} BoundedBelow.

\end{definition}
Assume A is a subset of $\mathbb{R}$.
A is bounded below  iff 
there exists an element b of $\mathbb{R}$ such that b is a lower bound of A .

\begin{definition} BoundedAbove.

\end{definition}
Assume A is a subset of $\mathbb{R}$.
A is bounded above  iff 
there exists an element b of $\mathbb{R}$ such that b is an upper bound of A .

\begin{definition} least-upper-bound-property.

\end{definition}
Assume R is a subset of $\mathbb{R}$.
R is lub iff for every subset A of R
if (A is bounded above and not empty) then (there exists an element s of R such that s is supremum of A ).

\begin{definition} greatest-lower-bound-property.

\end{definition}
Assume R is a subset of $\mathbb{R}$.
R is glb iff for every subset A of R
if (A is bounded below and not empty) then (there exists an element s of R such that s is infimum of A ).

\begin{axiom} $\mathbb{R}$ is lub.

\end{axiom}
\begin{axiom} $\mathbb{R}$ is glb.

\end{axiom}

AXiom. If x-y = z then x =z+y. 


\begin{theorem}
 P120a. If x$>$0 and y is element of $\mathbb{R}$ then exists n  n$\cdot$x$>$y.
\end{theorem}\begin{proof}

Assume x$>$0 and y is element of $\mathbb{R}$.
(1)	Assume the contrary.	
Define J = {g$\cdot$x $|$ g is natural number}
Then J is bounded above.
Take an element alpha of $\mathbb{R}$ such that alpha is supremum of J.
Then alpha-x is not an upper bound of J.
Take element beta of J such that alpha-x$<$beta.
Take natural number m such that beta = m$\cdot$x.
Then alpha-x$<$m$\cdot$x. Then alpha $<$(m$\cdot$x)+x.	
Then alpha$<$(m+1)$\cdot$x.
Then (m+1)$\cdot$x is element of J.
A Contradiction.
\end{proof}




\begin{signature} A integer is a real number.

\end{signature}
\begin{signature} A natural number is an integer.

\end{signature}

\begin{definition} $\mathbb{Z}$et.

\end{definition}
$\mathbb{Z}$ is the set of integer.
\begin{axiom} 0,1 is an integer.

\end{axiom}

Let i,j denote integer.
\begin{axiom} i+j, -i are integer.

\end{axiom}

Let x $\leq$ y stand for x$<$y or x=y.
Let x $\geq$ y stand for y $\leq$ x.

\begin{theorem}
 Lis1. for every x exists j j-1$\leq$x$<$j.
\end{theorem}\begin{proof}\\
Let x be a real number.
Define J = \{g $|$ g is integer and x$<$g\}.\\
Then J is a subset of $\mathbb{Z}$ and bounded below.\\
Then J is a subset of $\mathbb{R}$.\\
1 $>$ 0 and x is element of $\mathbb{R}$.\\
Take a natural number n such that n$\cdot$1$>$ x.\\
n is a real number. Thus n is an element of J. Then J is not empty.\\
Take an element alpha of $\mathbb{R}$ such that alpha is infimum of J.\\
Take an element beta of $\mathbb{R}$ such that beta = alpha + 1 .\\
beta is not a lower bound of J.\\
Take an element i of J such that i $<$ beta. Then  alpha $\leq$ i and i-1 $<$ beta - 1 = alpha. Thus i-1 $\leq$ x $<$ i.\\

\end{proof}














\end{forthel}
\end{document}