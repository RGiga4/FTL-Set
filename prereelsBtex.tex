\documentclass{article}
\usepackage[english]{babel}
\usepackage{enumerate, latexsym, amssymb, amsmath}
\usepackage{framed, multicol}
\newenvironment{forthel}{\begin{leftbar}}{\end{leftbar}}

%%%%%%%%%% Start TeXmacs macros
\newcommand{\tmaffiliation}[1]{\\ #1}
\newcommand{\tmem}[1]{{\em #1\/}}
\newenvironment{enumeratenumeric}{\begin{enumerate}[1.] }{\end{enumerate}}
\newenvironment{proof}{\noindent\textbf{Proof\ }}{\hspace*{\fill}$\Box$\medskip}
\newenvironment{quoteenv}{\begin{quote} }{\end{quote}}
\newtheorem{axiom}{Axiom}
\newtheorem{lemma}{Lemma}
\newtheorem{theorem}{Theorem}
\newtheorem{definition}{Definition}
\newtheorem{signature}{Signature}
\newtheorem{proposition}{Proposition}
%%%%%%%%%% End TeXmacs macros

\newcommand{\event}{UITP 2018}
\newcommand{\dom}{Dom}
\newcommand{\fun}{aFunction}
\newcommand{\sym}{sym}
\newcommand{\halfline}{{\vspace{3pt}}}
\newcommand{\tab}{{\hspace{1cm}}}
\newcommand{\ball}[2]{B_{#1}(#2)}
\newcommand{\llbracket}{[}
\newcommand{\rrbracket}{]}
\newcommand{\less}[1]{<_{#1}}
\newcommand{\greater}[1]{>_{#1}}
\newcommand{\leeq}[1]{{\leq}_{#1}}
\newcommand{\supr}[1]{\mathrm{sup}_{#1}}
\newcommand{\RR}{\mathbb{R}}
\newcommand{\QQ}{\mathbb{Q}}
\newcommand{\ZZ}{\mathbb{Z}}
\newcommand{\NN}{\mathbb{N}}
\begin{document}

\title{The Complex Field}

\maketitle

\begin{forthel}
[set/-s] [element/-s] [number/-s] [integer/-s]

Signature. A real number is a notion.

Definition RELSet.
REL is the set of real numbers.

Let x,y,z,v, a,b denote real numbers.

###Axiome fur field

Signature Add. x + y is real number.
Axiom A2. x+y = y+x.
Axiom A3. (x+y)+z = x+(y+z).
Signature. 0 is a real number such that for every real number x x + 0 = x.
Signature. -x is a real number such that -x + x = 0.

Let x - y stand for x + (-y).

Signature Add. x * y is real number.
Axiom. x*y = y*x.
Axiom. (x*y)*z = x*(y*z).
Signature. 1 is a real number such that for every real number x x*1 = x.
Signature. inv(x) is a real number.
Axiom. If (not x = 0) then inv(x)*x = 1.
Axiom. 1 is not equal to 0.

Axiom. x*(y+z)=(x*y)+(x*z).

###weitere bewissene Aussagen
Axiom P114c. If x+y = 0 then y = -x.
Axiom P114d. -(-x) = x.

Axiom P116c. (-x)*y = -(x*y) = x*(-y).


###Axiome fur ordSet

Signature. x < y is an atom.
Let x > y stand for y < x.
Let x <= y stand for x<y or x=y.
Let x >= y stand for y <= x.

Axiom. Then x<y or y<x or x = y.
Axiom. Then not ((x<y and y<x) or (x<y and y=x) or (x=y and y<x)).
Axiom. If x<y and y<z then x<z.



###Axiome fur ordfield

Axiom A11. If y<z then x+y<x+z.
Axiom A21. If x>0 and y>0 then x*y>0.

Axiom P118a. If x>0 then -x<0.
Axiom P118a2. If x<0 then -x>0.



#### Axiome Naturliche Zahlen
Signature. A natural number is a real number.

Definition NatSet.
NAT is the set of natural numbers.

Axiom. 0 is natural number.
Let n denote a natural number.
Axiom. n+1 is a natural number.

Axiom. n >=0.
#### Axiome ganze Zahlen
Signature. A integer is a real number.

Definition ZSet.
ZS is the set of integer.
Let i,j denote integer.

#### Axiome ratio Zahlen
Signature. A rational number is a real number.

Definition QSet.
QS is the set of rational numbers.

Axiom. Let m be an element of ZS and n be an element of NAT. then m*inv(n) is a rational number.

Signature. A natural number is an integer.


Axiom. If x-y = z then x =z+y. 
Axiom. If x>-y then y>-x. 


Axiom P120a. If x>0 and y is element of REL then exists n  n*x>y.


Let q denote an element of QS.
Let m, m1, m2, mt denote integer.

### Axiome uber ganze Zahlen
Axiom Lis1. for every x exists m m-1<=x<m.
Axiom Lis2. if n*x<m then x<m*inv(n).
Axiom Lis3. if m<n*y then m*inv(n)<y.



Proposition P120b. if x < y then exists q   x<q<y.
Proof.
Assume x < y.
Take z = y-x.Then z >0.
Take n  such that n*z>1 (by P120a).
Then n is not 0.
Take v = n*x. then 1>0.

Take m such that m-1<=v<m.

Then n*x<m<=1+(n*x).

Then n*z = n*(y-x)=(n*y)-(n*x).
Then (n*y)-(n*x)>1.
Let us show that 1+(n*x)<n*y.

Take a = (n*y) and b = (n*x).
a-b>1.
b+(a+(-b))>b+1 (by A11).
(a+(-b))+b>1+b (by A2).
a+((-b)+b)>1+b (by A3).
a>1+b.
1+(n*x)<n*y.
end.
Then n*x<m<n*y.

Then x<m*inv(n)<y.

qed.











\end{forthel}
\end{document}