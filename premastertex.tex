\documentclass{article}
\usepackage[english]{babel}
\usepackage{enumerate, latexsym, amssymb, amsmath}
\usepackage{framed, multicol}
\newenvironment{forthel}{\begin{leftbar}}{\end{leftbar}}

%%%%%%%%%% Start TeXmacs macros
\newcommand{\tmaffiliation}[1]{\\ #1}
\newcommand{\tmem}[1]{{\em #1\/}}
\newenvironment{enumeratenumeric}{\begin{enumerate}[1.] }{\end{enumerate}}
\newenvironment{proof}{\noindent\textbf{Proof\ }}{\hspace*{\fill}$\Box$\medskip}
\newenvironment{quoteenv}{\begin{quote} }{\end{quote}}
\newtheorem{axiom}{Axiom}
\newtheorem{lemma}{Lemma}
\newtheorem{theorem}{Theorem}
\newtheorem{definition}{Definition}
\newtheorem{signature}{Signature}
\newtheorem{proposition}{Proposition}
%%%%%%%%%% End TeXmacs macros

\newcommand{\event}{UITP 2018}
\newcommand{\dom}{Dom}
\newcommand{\fun}{aFunction}
\newcommand{\sym}{sym}
\newcommand{\halfline}{{\vspace{3pt}}}
\newcommand{\tab}{{\hspace{1cm}}}
\newcommand{\ball}[2]{B_{#1}(#2)}
\newcommand{\llbracket}{[}
\newcommand{\rrbracket}{]}
\newcommand{\less}[1]{<_{#1}}
\newcommand{\greater}[1]{>_{#1}}
\newcommand{\leeq}[1]{{\leq}_{#1}}
\newcommand{\supr}[1]{\mathrm{sup}_{#1}}
\newcommand{\RR}{\mathbb{R}}
\newcommand{\QQ}{\mathbb{Q}}
\newcommand{\ZZ}{\mathbb{Z}}
\newcommand{\NN}{\mathbb{N}}

\begin{document}

\title{An SAD3 Formalization of the Basic Number Systems for Walter Rudin's
\it{Principles of Mathematical Analysis}}

\author{Peter Koepke}

\date{August 25, 2018}

\maketitle


\section{Introduction}

The SAD3 proof-checking system checks the logical correctness of texts written in an input language which is acceptable and readable as common mathematical language. Texts should resemble the style of undergraduate textbooks. To test the viability of this approach we are currently formalizing some initial chapters of the {\it Principles of Mathematical Analysis} by Walter Rudin \cite{Rudin}. SAD3 accepts texts in a native ForTheL (Formula Theory Language) format. We are also testing a {\LaTeX} add-on which accepts texts in {\LaTeX} format and transforms them into ForTheL.

The present formalization deals with pages 1 to 9 of \cite{Rudin}. Like in the book, the ordered field $\mathbb{R}$ of real numbers is postulated. We construe the structures of integer and rational numbers as substructures of $\mathbb{R}$:
$$\NN \subseteq \ZZ \subseteq \QQ \subseteq \RR.$$
This has the advantage that the real addition and multiplication can be used for those substructures and we do not have to talk about ordered sets and fields in general. A large part of the material of Rudin appears in the formalization in some form. We have turned the axioms about general linear orders and fields into axioms for the real numbers.



We first did the formalization in a ForTheL file {\tt Numbers03.ftl}. After this file was accepted by SAD3 it was rewritten to a {\LaTeX} file
{\tt Numbers02.tex}. The {\LaTeX} file may contain further text that is typeset but not checked. After typesetting the ForTheL content is indicated by a vertical line.  In this way this whole document is a single SAD3 document, whose ForTheL content is proofchecked. Towards the end of the formalization we put the original Rudin text side-by-side with the formalization for comparison.



\section{The Formalization}
\subsection{Some Set-Theoretic Terminology}

Rudin introduces this on page 3.

\begin{forthel}
[set/-s] [element/-s] [belong/-s] [subset/-s] [relation/-s] [number/-s]

Signature. A real number is a notion.
Definition. REL is the set of real numbers. 
Axiom. every real numbers is element of REL.






Let S,T,A,B,C denote sets.

Definition DefSubset.   A subset of S is a set T
such that every element of T is a element of S.

Definition DefEmpty.    S is empty iff S has no elements.

Signature. A relation is a notion.

Signature. An order is a relation.

Signature.Let < be a relation.

Signature. Let x,y be elements of REL. x<y is an atom.

Let x,y,z denote real numbers.

Axiom. x<y or y<x or x = y.

Axiom. Then not ((x<y and y<x) or (x<y and y=x) or (x=y and y<x)).
Axiom. If x<y and y<z then x<z.



Definition upperBound.
Assume A is a subset of REL.
An upper bound of A is an element b of REL such that ( y<b or y = b ) for every element y of A.

Definition lowerBound.
Assume A is a subset of REL.
A lower bound of A is an element b of REL such that ( b<y or y = b ) for every element y of A.

Definition Supremum.
Assume A is a subset of REL.
Let s be an element of REL.
s is supremum of A  iff s is an upper bound of A 
and for every element x of REL if x<s then x is not an upper bound of A .

Definition Infimum.
Assume A is a subset of REL.
Let s be an element of REL.
s is infimum of A  iff s is an lower bound of A 
and for every element x of REL if s<x then x is not an lower bound of A .

Definition BoundedBelow.
Assume A is a subset of REL.
A is bounded below  iff 
there exists an element b of REL such that b is a lower bound of A .

Definition BoundedAbove.
Assume A is a subset of REL.
A is bounded above  iff 
there exists an element b of REL such that b is an upper bound of A .

Definition leastupperboundproperty.
Assume R is a subset of REL.
R is lub iff for every subset A of R
if (A is bounded above and not empty) then (there exists an element s of R such that s is supremum of A ).

Proposition.

Assume REL is lub.
Assume B is a subset of REL and not empty and bounded below.
Let L = {a in REL | a is lower bound of B}.
Then there exists an element a of REL such that a is supremum of L and a is infimum of B .
Proof.
L is not empty.
Every element x of B is an upper bound of L.
Then L is bounded above.
Take an element a of REL such that a is supremum of L.
a is an element of L.
qed.




\end{forthel}

\subsection{The Field of Real Numbers}

Rudin introduces $\RR$ in Theorem 1.19 and refers to the ordered field 
axioms in 1.5, 1.6, 1.12. Our propositions correspond to Rudin's 1.14 - 1.16 on fields. 

\begin{forthel}

Signature. A field is a notion.

Let x,y,z denote real numbers.

Signature Add. x + y is real number.
Axiom A1. Let x, y be real numbers. Then x + y is a real number.
Axiom A2. x + y = y + x.
Axiom A3. (x + y) + z = x + (y + z). 
Signature A4. 0 is element of REL such that for every real number x x + 0 = x.
Signature A5. -x is element of REL such that x+(-x) = 0.

Signature M1. x * y is a real number.
Axiom M2. x * y = y * x.
Axiom M3. (x * y) * z = x * (y * z).
Signature M4. 1 is element of REL such that for every real number x x * 1 = x.
Axiom M42. 0 is not equal to 1.
Signature M5. Assume not x = 0. inv(x) is a real number such that inv(x)*x = 1.

Axiom D. x*(y + z) = (x*y) + (x*z).



Proposition P114a.
If x + y = x + z then y = z.
Proof.

Assume x + y = x + z.
y  = 0 + y
= ((-x)+x) + y
= -x + (x+y)
= -x + (x+z)
= ((-x)+x) + z
= 0 + z
= z.
qed.

Proposition P114b.
If x + y = x then y = 0.
Proposition P114c. 
If x + y = 0 then y = -x.
Proof.
Assume x + y = 0.
Take z = -x. Then x + z = 0.
Then x + z = x + y. Then y = -x (by P114a).
qed.
Proposition P114d. 
Let x be elements of REL. -(-x) = x.
Proof.
Then (-x)+x = 0.
Then -(-x) = x (by P114c).
qed.


Proposition P115a.
Let x,y,z be elements of REL. If x is not equal to 0 and x*y = x*z then y = z.
Proposition.
Let x,y be elements of REL. If x is not equal to 0 and x*y = x then y = 1.
Proposition.
Let x,y be elements of REL. If x is not equal to 0 and x*y = 1 then y = inv(x).
Proposition.
Let x be elements of REL. If x is not equal to 0 then inv(inv(x)) = x.



Proposition P116a.
Let x be elements of REL. 0*x = 0.
Proof.
Then (0*x) + (0*x)
= (0+0)*x
= 0*x.
Then 0*x = 0 (by P114b).
qed.
Proposition P116b. 
If x is not equal to 0 and y is not equal to 0 then x*y is not equal to 0.
Proof.

Let x is not equal to 0 and y is not equal to 0.
Assume the contrary.

Then 1 = (inv(x)*inv(y))*(x*y)
= (inv(x)*inv(y))*0
= 0 .
Contradiction.
qed.

Lemma L1.
(-x)*y = -(x*y).
Proof.

((-x)*y) + (x*y) .= (y*(-x)) + (y*x) (by M2)
.= y*(-x+x) (by D) 
.= (-x+x)*y (by M2)
.= 0*y (by A2, A5)
.= 0 (by P116a, M2).

qed. 
Proposition P116c.
(-x)*y = -(x*y) = x*(-y).
Proof.
Then(-x)*y = -(x*y) (by L1).

Then -(x*y) = -(y*x) = (-y)*x = x*(-y).

qed. 

Proposition P116d.
(-x)*(-y)= x*y.


Let x - y stand for x + (-y).


Let x > y stand for y < x.
Let x <= y stand for x<y or x=y.
Let x >= y stand for y <= x.



\end{forthel}

\subsection{The Ordered Real Field}
Rudin introduces ordered fields in 1.17, corresponding to the subsequent two axioms. Our propositions formalize Rudin's 1.18.

\begin{forthel}
	Axiom A1. If y<z then x+y<x+z.
	Axiom A2. If x>0 and y>0 then x*y>0.
	
	Proposition P118a. If x>0 then -x<0. 
	Proof.
	Assume x>0. Then 0= -x + x > -x + 0.
	Then -x<0.
	qed.
	Proposition P118a2. If x<0 then -x>0. 
	Proof.
	Assume x<0. Then 0= -x + x < -x + 0.
	Then -x>0.
	qed.
	Proposition P118b. If x>0 and y<z then x*y<x*z.
	Proof.
	Assume x>0 and y<z. 
	Let us show that z-y > y-y = 0.
	y<z.
	then (-y)+y<(-y)+z (by A1).
	
	end.
	Then x*(z-y)>0.
	Then x*z = (x*(z-y))+(x*y).
	
	Take a = x*(z-y) and b = x*y.
	Then a>0.
	Then x*z = a + b > 0+b = (x*y).
	
	Then x*z > x*y.
	qed.
	Proposition P118c. If x<0 and y<z then x*y>x*z.
	Proof.
	Assume x<0 and y<z.
	Then -x>0 (by P118a2).
	Then z-y>0. 
	Let us show that z-y>0.
	y<z.
	then (-y)+y<(-y)+z (by A1).
	then z-y>y-y=0.
	end.
	Then -(x*(z-y)) = (-x)*(z-y) (by P116c).
	Then (-x)*(z-y) > 0.
	Then x*(z-y) < 0.
	Then (x*z) - (x*y) < 0.
	
	Take a = (x*z) and b = - (x*y).
	Then a+b<0.
	Then a < -b.
	
	Then x*z < x*y.
	qed.
	
	Proposition P118d. If not x = 0 then x*x > 0.
	Proof.
	Case x>0. Then x*x > 0.
	end.
	Case x<0. Then -x>0. Then x*x = (-x)*(-x)>0.
	end.
	
	qed.
	Lemma. 1>0. 
	Proof.
	Then 1 = 1*1 > 0 (by P118d, M42).
	qed.
	
	Proposition P118e1. If 0<y then 0 < inv(y).
	Proof.
	(1) Assume the contrary.
	Then y*inv(y) <= 0.
	Then y*inv(y) = 1 > 0.
	Contradiction.
	
	qed.
	Proposition P118e2. If 0<x<y then 0 < inv(y)<inv(x).
	Proof. 
	Assume 0<x<y.Then 0<inv(y) and 0<inv(x).
	Then inv(x)*inv(y)>0.
	Take a = inv(y)*inv(x).	
	Then x<y. Then x*a<y*a.
	Then x*(inv(x)*inv(y))< y*(inv(x)*inv(y)).
	Then (x*inv(x))*inv(y)< (y*inv(y))*inv(x).
	Then inv(y) < inv(x).
	
	qed.
	

\end{forthel}

\subsection{Upper and Lower Bounds}
This corresponds to Rudin's 1.7. and 1.8. We also study lower bounds and their relation to upper bounds, because lower bounds appear more natural in the context of induction.


\subsection{Rational Numbers}
Integer and rational numbers are not axiomatized or constructed in Rudin, but simple assumed. We need the following formalizations to make the text self-contained.

\begin{forthel}

\begin{signature} A \emph{rational number} is a real number.
Let $p,q,r$ stand for rational numbers.\end{signature}

\begin{definition} $\QQ$ is the set of rational numbers.
\end{definition}


\begin{lemma} $\QQ \subseteq \RR$.\end{lemma}

\begin{axiom} $p + q$, $p \cdot q$, $0$, $-p$, $1$ are 
rational numbers.\end{axiom}

\begin{axiom} Assume $p \neq 0$. $1/p$ is a rational number.
\end{axiom}

\end{forthel}




\subsection{Positive Integers}
The natural numbers $1,2,3,..$ are defined as positive integers, to use the terminology from Rudin.
\begin{forthel}
	Signature. A natural number is a real number.
	
	Definition NatSet.
	NAT is the set of natural numbers.
	
	Axiom. 0 is natural number.
	Let n, m denote a natural number.
	Axiom. n+1 is a natural number.
	Axiom. n >=0.
	
	Signature. a natural number is a real number.
	
	

	
	Proposition Lis2. Let n be a natural number and not n=0. If n*x<m then x<m*inv(n).
	Proof.
	Assume n*x<m.
	n > 0.
	inv(n) > 0 (by P118e1).
	inv(n)*(n*x)<inv(n)*m (by P118b).
	(x*n)*inv(n)<m*inv(n).
	x*(n*inv(n))<m*inv(n).
	x*1<m*inv(n).
	x<m*inv(n).
	qed.
	Proposition Lis3. Let n be a natural number and not n=0. if m<n*y then m*inv(n)<y.
	Proof.
	Assume m<n*y.
	n > 0.
	inv(n) > 0 (by P118e1).
	inv(n)*m<inv(n)*(n*y) (by P118b).
	m*inv(n)<(n*inv(n))*y.
	m*inv(n)<1*y.
	m*inv(n)<y.
	qed.
\end{forthel}

\subsection{Archimedian properties}
Finally we have formalized what Rudin assumes for the proof of his 1.20. We can now contrast the formalization with Rudin's original.

\newpage
The next two theorems correspond to Theorems 1.20(a) and (b) of Rudin. We start with the original introductory text of Rudin and then display the two texts in parallel to illustrate the closeness of the formalization to the original.

\vspace{1cm}

The next theorem could be extracted from this construction [of the real numbers] with very little extra effort. However, we prefer to derive it from Theorem 1.19 since this provides a good illustration of what one can do with the least-upper-bound property.

\begin{multicols}{2}



\begin{theorem}
(a) If $x\in \mathbb{R}$, $y\in \mathbb{R}$, and $x>0$, then there is a positive integer $n$ such that
$$n x > y.$$
\end{theorem}
\begin{proof}
Let $A$ be the set of all $n x$, where $n$ runs through the positive integers. If $(a)$ were false, then $y$ would be an upper bound of $A$. But then $A$ has a \emph{least} upper bound in $\mathbb{R}$.
Put $\alpha=\sup A$. Since $x>0$, $\alpha-x<\alpha$, and $\alpha-x$ is not an upper bound of $A$. Hence $\alpha-x<m x$ for some positive integer $m$. But then $\alpha<(m+1) x \in A$, which is impossible, since $\alpha$ is an upper bound of $A$.
\end{proof}

\end{multicols}
\newpage
\begin{multicols}{2}



\columnbreak

\begin{theorem}
If $x\in\mathbb{R}$, $y\in\mathbb{R}$, and $x<y$, then there exists
a $p\in\mathbb{Q}$ such that $x<p<y$.
\end{theorem}
\begin{proof}
Since $x < y$, we have $y-x>0$, and $(a)$ furnishes a positive integer $n$ such that
$$m (y-x) > 1.$$
Apply $(a)$ again, to obtain positive integers $m_1$ and $m_2$ such that
$m_1 > n x$, $m_2 > -n x$. Then
$$-m_2 <n x < m_1.$$
Hence there is an integer $m$ (with $-m_2 \leq m \leq m_1$) such that
$$m-1 \leq n x < m.$$
If we combine these inequalities, we obtain
$$n x < m \leq 1 + n x < n y.$$
Since $n > 0$, it follows that
$$x < \frac{m}{n} < y.$$
This proves $(b)$, with $p = m/n$.
\end{proof}

\end{multicols}

\section{Remarks}
\subsection{Structures}

Rudin introduces ordered sets and fields axiomatically and then postulates the ordered field of reals. Indeed the reals are constructed from the rationals numbers in an Appendix to chapter 1.

Since ForTheL does not provide general structures and mechanisms to express that a structure satisfies some abstract axioms, I have instead postulated the structure $\mathbb{R}$ of the real numbers right away. The axioms of ordered sets and fields are then only stated for this particular structure.

Instead of building up number systems "from below" we define the sets of rational, integer and positive integer numbers as subsets of $\mathbb{R}$. This has the advantage that we can use the real addition and multiplication also for those number systems and they become substructures without substructure mechanisms.

Future versions of SAD3 should allow for elegant handling of structures. This also involves a liberal use of operator symbols like $+$ in several additive structures, without mentioning the structure, e.g., as a subscript. This "overloading" would require some (Prolog-like?) derivations of notions for variables and terms.

\subsection{Text Comparisons}

We cover most of the material on pages 3 - 9 of Rudin. By the different organisation, the two presentations only align on page 9 for some archimedean properties. We illustrate this alignment in a two-column synoptic view of the formalization and the original. One could go through the text statement by statement and discuss similarities and differences. One could also put in further work to minimize differences. To some degree this could take part on the level of the formalization, but eventually the ForTheL language and its interpretation in SAD3 would have to be extended.
I shall only mention a few points.






\begin{thebibliography}{1}

\bibitem{Rudin}
  Walter Rudin,
  \textit{Principles of mathematical analysis},
  McGraw-Hill,
  1976.

\end{thebibliography}
  


\end{document}
