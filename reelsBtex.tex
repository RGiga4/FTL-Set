\documentclass{article}
\usepackage[english]{babel}
\usepackage{enumerate, latexsym, amssymb, amsmath}
\usepackage{framed, multicol}
\newenvironment{forthel}{\begin{leftbar}}{\end{leftbar}}

%%%%%%%%%% Start TeXmacs macros
\newcommand{\tmaffiliation}[1]{\\ #1}
\newcommand{\tmem}[1]{{\em #1\/}}
\newenvironment{enumeratenumeric}{\begin{enumerate}[1.] }{\end{enumerate}}
\newenvironment{proof}{\noindent\textbf{Proof\ }}{\hspace*{\fill}$\Box$\medskip}
\newenvironment{quoteenv}{\begin{quote} }{\end{quote}}
\newtheorem{axiom}{Axiom}
\newtheorem{lemma}{Lemma}
\newtheorem{theorem}{Theorem}
\newtheorem{definition}{Definition}
\newtheorem{signature}{Signature}
\newtheorem{proposition}{Proposition}
%%%%%%%%%% End TeXmacs macros

\newcommand{\event}{UITP 2018}
\newcommand{\dom}{Dom}
\newcommand{\fun}{aFunction}
\newcommand{\sym}{sym}
\newcommand{\halfline}{{\vspace{3pt}}}
\newcommand{\tab}{{\hspace{1cm}}}
\newcommand{\ball}[2]{B_{#1}(#2)}
\newcommand{\llbracket}{[}
\newcommand{\rrbracket}{]}
\newcommand{\less}[1]{<_{#1}}
\newcommand{\greater}[1]{>_{#1}}
\newcommand{\leeq}[1]{{\leq}_{#1}}
\newcommand{\supr}[1]{\mathrm{sup}_{#1}}
\newcommand{\RR}{\mathbb{R}}
\newcommand{\QQ}{\mathbb{Q}}
\newcommand{\ZZ}{\mathbb{Z}}
\newcommand{\NN}{\mathbb{N}}
\begin{document}

\title{The Reel Field part B}

\maketitle

Reels-B wurde sehr oft überarbeitet da es die Eigenschaft hat spontan nicht mehr verifiziert zu werde, da man wieder eine Kleinigkeit ge\"andert hat.\\

Diese Version benutzt die starke Aussage Lis1 die in Reels-A gezeigt wurde.\\
Aussage Lis1 wurde nicht extra im Rudin gezeigt sondern nur indirekt benutzt.\\

\begin{forthel}
[set/-s] [element/-s] [number/-s] [integer/-s]

\begin{signature} A real number is a notion.

\end{signature}

\begin{definition} $\mathbb{R}$Set.

\end{definition}
$\mathbb{R}$ is the set of real numbers.

Let x,y,z,v, a,b denote real numbers.



\begin{signature} Add. x + y is real number.

\end{signature}
\begin{axiom} A2. x+y = y+x.

\end{axiom}
\begin{axiom} A3. (x+y)+z = x+(y+z).

\end{axiom}
\begin{signature} 0 is a real number such that for every real number x x + 0 = x.

\end{signature}
\begin{signature} -x is a real number such that -x + x = 0.

\end{signature}

Let x - y stand for x + (-y).

\begin{signature} Add. x $\cdot$ y is real number.

\end{signature}
\begin{axiom} x$\cdot$y = y$\cdot$x.

\end{axiom}
\begin{axiom} (x$\cdot$y)$\cdot$z = x$\cdot$(y$\cdot$z).

\end{axiom}
\begin{signature} 1 is a real number such that for every real number x x$\cdot$1 = x.

\end{signature}
\begin{signature} inv(x) is a real number.

\end{signature}
\begin{axiom} If (not x = 0) then inv(x)$\cdot$x = 1.

\end{axiom}
\begin{axiom} 1 is not equal to 0.

\end{axiom}

\begin{axiom} x$\cdot$(y+z)=(x$\cdot$y)+(x$\cdot$z).

\end{axiom}


\begin{axiom} P114c. If x+y = 0 then y = -x.

\end{axiom}
\begin{axiom} P114d. -(-x) = x.

\end{axiom}

\begin{axiom} P116c. (-x)$\cdot$y = -(x$\cdot$y) = x$\cdot$(-y).

\end{axiom}




\begin{signature} x $<$ y is an atom.

\end{signature}
Let x $>$ y stand for y $<$ x.\\
Let x $\leq$ y stand for x$<$y or x=y.\\
Let x $\geq$ y stand for y $\leq$ x.\\

\begin{axiom} Then x$<$y or y$<$x or x = y.

\end{axiom}
\begin{axiom} Then not ((x$<$y and y$<$x) or (x$<$y and y=x) or (x=y and y$<$x)).

\end{axiom}
\begin{axiom} If x$<$y and y$<$z then x$<$z.

\end{axiom}





\begin{axiom} A11. If y$<$z then x+y$<$x+z.

\end{axiom}
\begin{axiom} A21. If x$>$0 and y$>$0 then x$\cdot$y$>$0.

\end{axiom}

\begin{axiom} P118a. If x$>$0 then -x$<$0.

\end{axiom}
\begin{axiom} P118a2. If x$<$0 then -x$>$0.

\end{axiom}




\begin{signature} A natural number is a real number.

\end{signature}

\begin{definition} NatSet.

\end{definition}
$\mathbb{N}$ is the set of natural numbers.

\begin{axiom} 0 is natural number.

\end{axiom}
Let n denote a natural number.
\begin{axiom} n+1 is a natural number.

\end{axiom}

\begin{axiom} n $\geq$0.

\end{axiom}

\begin{signature} A integer is a real number.

\end{signature}

\begin{definition} $\mathbb{Z}$et.

\end{definition}
$\mathbb{Z}$ is the set of integer.
Let i,j denote integer.


\begin{signature} A rational number is a real number.

\end{signature}

\begin{definition} $\mathbb{Q}$et.

\end{definition}
$\mathbb{Q}$ is the set of rational numbers.

\begin{axiom} Let m be an element of $\mathbb{Z}$ and n be an element of $\mathbb{N}$. then m$\cdot$inv(n) is a rational number.

\end{axiom}

\begin{signature} A natural number is an integer.

\end{signature}


\begin{axiom} If x-y = z then x =z+y. 

\end{axiom}
\begin{axiom} If x$>$-y then y$>$-x. 

\end{axiom}


\begin{axiom} P120a. If x$>$0 and y is element of $\mathbb{R}$ then exists n  n$\cdot$x$>$y.

\end{axiom}


Let q denote an element of $\mathbb{Q}$.
Let m, m1, m2, mt denote integer.


\begin{axiom} Lis1. for every x exists m m-1$\leq$x$<$m.

\end{axiom}
\begin{axiom} Lis2. if n$\cdot$x$<$m then x$<$m$\cdot$inv(n).

\end{axiom}
\begin{axiom} Lis3. if m$<$n$\cdot$y then m$\cdot$inv(n)$<$y.

\end{axiom}



\begin{theorem}
 P120b. if x $<$ y then exists q   x$<$q$<$y.
\end{theorem}\begin{proof}\\
Assume x $<$ y.\\
Take z = y-x.Then z $>$0.\\
Take n  such that n$\cdot$z$>$1 (by P120a).\\
Then n is not 0.\\
Take v = n$\cdot$x. then 1$>$0.\\
Take m such that m-1$\leq$v$<$m.\\
Then n$\cdot$x$<$m$\leq$1+(n$\cdot$x).\\
Then n$\cdot$z = n$\cdot$(y-x)=(n$\cdot$y)-(n$\cdot$x).\\
Then (n$\cdot$y)-(n$\cdot$x)$>$1.\\

Let us show that 1+(n$\cdot$x)$<$n$\cdot$y.\\
Take a = (n$\cdot$y) and b = (n$\cdot$x).\\
a-b$>$1.\\
b+(a+(-b))$>$b+1 (by A11).\\
(a+(-b))+b$>$1+b (by A2).\\
a+((-b)+b)$>$1+b (by A3).\\
a$>$1+b.\\
1+(n$\cdot$x)$<$n$\cdot$y.\\
end.\\

Then n$\cdot$x$<$m$<$n$\cdot$y.\\
Then x$<$m$\cdot$inv(n)$<$y.\\

\end{proof}












\end{forthel}
\end{document}